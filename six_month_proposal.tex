\documentclass{article}

\usepackage{graphicx}
\usepackage[space]{grffile}
\usepackage{latexsym}
\usepackage{amsfonts,amsmath,amssymb}
\usepackage{url}
\usepackage[utf8]{inputenc}
\usepackage{fancyref}
\usepackage{hyperref}
\hypersetup{colorlinks=false,pdfborder={0 0 0},}
\usepackage{textcomp}
\usepackage{longtable}
%\usepackage{multirow,booktabs}



\begin{document}

\title{Proposal}

\author{Nicholas Davies\\ University of Canterbury }

\date{\today}

\bibliographystyle{plain}

\maketitle


\section{Growth stresses }
Growth stress here refers to the internal stress field of a tree which
results from the strain induced at the periphery during growth by cell
maturation. Most commonly there is a stress gradient over the stem from tension
at the periphery to compression in at the pith in the longitudinal direction,
in the transverse direction compression is observed. Reaction wood significantly
distorts this pattern in trees which are off axis or under some other loading.

At harvesting growth stresses can ruin structural and veneer logs due to the
resulting splitting and warping. End splits, heart checks, and ring shakes all
reduce the value of the timber in a stem. When the stem is felled or cross cut,
growth stresses are released around the saw cuts causing shortening at the
periphery and extension in the centre. The dimension change is maximum at the
saw cut, reducing as distance from the cut increases. When the
contraction/extension force exceeds the plastic limit of the stem splitting
occurs. Prolonged compression at the centre of the stem during growth can exceed
the elastic limit of the wood, resulting in internal defects such as brittle heart.

Within mills during processing growth stresses cause a number of issues leading
to reductions in value recovery. Because growth stresses are released when the
stem is sectioned via sawing (plain, quarter etc.) the resulting shape change
can cause the saws to jam. Value losses at this stage of processing comes
from the need to saw boards multiple times in order to release the stresses
while still allowing for the final board dimensions to be retrieved, often
this can result in boards of smaller than optimal dimensions. The
resulting economic loss comes from the final yield being as low as 30%.

Selection for low growth stress producing families should significantly reduce
the occurrence of internal defects, increasing value recovery.

When a typical fibre cell develops it divides from a cambial file, produces a
primary cell wall and expands. Once expansion is complete the secondary
cell wall forms, initially cellulose and hemicellulose are deposited towards
the cell lumen (from the primary wall). The cell wall consists of a number of
layers, of which some or all may be present, the Compound middle laminar (CML),
the secondary cell wall layers; S1,S2,S3 and the gelatinous layer (G layer).
Inside of the CML are the S layers are laid in order toward the lumen. Finlay
in tension wood, after the formation of at least S1 a G layer can be formed. At
some time after the formation of the cellulose structure lignification occurs,
starting at the CML and progressing inwards, this may start in the CML before
the cellulose is completely laid down in the secondary wall. At some point during
the process of the formation of the secondary wall a contraction or extension (in
the case of compression wood) of the cell occurs. Because each cell is connected
to its neighbours when shape change occurs the inner cells resist the change, and
consequently are under a state of compression while the newly formed cells
cannot fully contract and are in a state of tension. When reaction wood is
considered the development of the stress field from the contraction or
extension of cells can result in substantual reorientation of the stem or branches.

It is thought that growth stresses provide trees with a competitive
advantage as they can reorient their stems and canopy for optimal light
interception, or increase their mechanical stability reducing breakage or
uprooting. Even in straight stems growth stresses increase mechanical stability
by prestressing the periphery of the stem as wood is stronger in tension than
compression. Transverse compression has been suggested to help under drought
conditions when turgor pressures are high.

Currently there are three competing hypothesis (none of which have any direct
experimental evidence) to explain the molecular origin of growth stresses.

The Lignin swelling hypothesis states that after the cellulose and hemicellulose
fibril aggregates (FA) have been laid down in the cell wall, lignification
pushes these fibril aggregates apart, shortening the distance a fibril can
cover. The result is a longitudinal contraction and tangential expansion if the
MicroFibril Angle (MFA) is below ~40 degrees. Longitudinal expansion occurs
(as in compression wood) for larger MFAs.

The cellulose contraction hypothesis argues cellulose contracts during
crystallisation, or more recently that hemicellulose interacts which the
cellulose fibrils causing a contraction or expansion.

Unifying the two hypotheses presented above, utilising lignin swelling to
explain compression and normal wood while cellulose contraction to explained
tension wood, Yamamoto et al. developed the aptly named unified hypothesis.

\end{document}