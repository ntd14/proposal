\documentclass{article}

\usepackage{graphicx}
\usepackage[space]{grffile}
\usepackage{latexsym}
\usepackage{amsfonts,amsmath,amssymb}
\usepackage{url}
\usepackage[utf8]{inputenc}
\usepackage{fancyref}
\usepackage{hyperref}
\hypersetup{colorlinks=false,pdfborder={0 0 0},}
\usepackage{textcomp}
\usepackage{longtable}
%\usepackage{multirow,booktabs}



\begin{document}

\title{Proposal}

\author{Nicholas Davies\\ University of Canterbury }

\date{\today}

\bibliographystyle{plain}

\maketitle


\section{Growth stresses }
Growth stress here refers to the internal stress field of a tree which
results from the strain induced at the periphery during growth by cell
maturation. Most commonly there is a stress gradient over the stem from tension
at the periphery to compression in at the pith in the longitudinal direction,
in the transverse direction compression is observed. Reaction wood significantly
distorts this pattern in trees which are off axis or under some other loading.

At harvesting growth stresses can ruin structural and veneer logs due to the
resulting splitting and warping. End splits, heart checks, and ring shakes all
reduce the value of the timber in a stem. When the stem is felled or cross cut,
growth stresses are released around the saw cuts causing shortening at the
periphery and extension in the centre. The dimension change is maximum at the
saw cut, reducing as distance from the cut increases. When the
contraction/extension force exceeds the plastic limit of the stem splitting
occurs. Prolonged compression at the centre of the stem during growth can exceed
the elastic limit of the wood, resulting in internal defects such as brittle heart.

Within mills during processing growth stresses cause a number of issues leading
to reductions in value recovery. Because growth stresses are released when the
stem is sectioned via sawing (plain, quarter etc.) the resulting shape change
can cause the saws to jam. Value losses at this stage of processing comes
from the need to saw boards multiple times in order to release the stresses
while still allowing for the final board dimensions to be retrieved, often
this can result in boards of smaller than optimal dimensions. The
resulting economic loss comes from the final yield being as low as 30%.

Selection for low growth stress producing families should significantly reduce
the occurrence of internal defects, increasing value recovery.

When a typical fibre cell develops it divides from a cambial file, produces a
primary cell wall and expands. Once expansion is complete the secondary
cell wall forms, initially cellulose and hemicellulose are deposited towards
the cell lumen (from the primary wall). The cell wall consists of a number of
layers, of which some or all may be present, the Compound middle laminar (CML),
the secondary cell wall layers; S1,S2,S3 and the gelatinous layer (G layer).
Inside of the CML are the S layers are laid in order toward the lumen. Finlay
in tension wood, after the formation of at least S1 a G layer can be formed. At
some time after the formation of the cellulose structure lignification occurs,
starting at the CML and progressing inwards, this may start in the CML before
the cellulose is completely laid down in the secondary wall. At some point during
the process of the formation of the secondary wall a contraction or extension (in
the case of compression wood) of the cell occurs. Because each cell is connected
to its neighbours when shape change occurs the inner cells resist the change, and
consequently are under a state of compression while the newly formed cells
cannot fully contract and are in a state of tension. When reaction wood is
considered the development of the stress field from the contraction or
extension of cells can result in substantual reorientation of the stem or branches.

It is thought that growth stresses provide trees with a competitive
advantage as they can reorient their stems and canopy for optimal light
interception, or increase their mechanical stability reducing breakage or
uprooting. Even in straight stems growth stresses increase mechanical stability
by prestressing the periphery of the stem as wood is stronger in tension than
compression. Transverse compression has been suggested to help under drought
conditions when turgor pressures are high.

Currently there are three competing hypothesis (none of which have any direct
experimental evidence) to explain the molecular origin of growth stresses.

The Lignin swelling hypothesis states that after the cellulose and hemicellulose
fibril aggregates (FA) have been laid down in the cell wall, lignification
pushes these fibril aggregates apart, shortening the distance a fibril can
cover. The result is a longitudinal contraction and tangential expansion if the
MicroFibril Angle (MFA) is below ~40 degrees. Longitudinal expansion occurs
(as in compression wood) for larger MFAs.

The cellulose contraction hypothesis argues cellulose contracts during
crystallisation, or more recently that hemicellulose interacts which the
cellulose fibrils causing a contraction or expansion.

Unifying the two hypotheses presented above, utilising lignin swelling to
explain compression and normal wood while cellulose contraction to explained
tension wood, Yamamoto et al. developed the aptly named unified hypothesis.

\section{Proposed work}
\subsection{Theoretical work}
Over the years there have been a number of attempts to mathematically model
cells (usually fibres or tracheids) from cell wall constituents (Mark 1967,
Koponen et al. 1989, Harrington et al. 1998, Yamamoto and Kojima 2002, Kojima
and Yamamoto 2004 are a few example) however very few efforts have used these
techniques to investigate the formation of growth stresses (archer 1987,
Yamamoto 1998, Guitard et al. 1999).

Currently the most advanced model for how growth stresses develop within the
cell wall was presented by Almeras et. al. (2005) using the unified hypothesis
(Okuyama et. al. 1986, Okuyama et. al. 1994, Yammamoto et. al. 1991, Yammamoto
et. al. 1992 and Yammamoto 1998) utilising both the lignin swelling and
cellulose contraction hypotheses. For details see section ---.

Proposed model of the cell:
Modelling of a generic single cell with variable cell wall parameters to
investigate the required geometry and constituents to create maximum
longitudinal and tangential extension and contraction via the lignin swelling
hypothesis. The single cell model should have the capacity to put limits on the
magnitude of stress generation the lignin swelling hypothesis is theoretically
capable of under different constituent and geometric makeups.

Because the proposed experiments (see section --) induce tension wood in species
both with and without G-layers an experimental upper limit of the stress
generation the lignin swelling hypothesis is capable of should be reached and
compared to the theoretical one derived above.

It is expected that the base model and parameters will be similar to those
utilised to describe lignin swelling by Almeras  et. al. (2005) and Yammamoto
(1998). Cell wall layer radii, thickness, S2 layer MFA, moduli of the CMF
bundles and matrix will all be included. Additional variables will be included
as necessary. It is intended to add the standard deviation of the MFA within the
cell wall layers, as in Harrington (1998), pore size (or conversely fibril
aggregate size) (Fahlen 2005, Chang 2014, Salmen 2012, Kim 2012) and cell wall
constituents (Baba 2009, Donaldson et al. 2001) and layer properties/geometries
(Bergander 2002, Grozdits 1982, Almeras 2005, Yammamoto 1995, 1998, Chang 2014,
salmen 2002) to form a model, conceptually similar to the qualitative
architecture presented by Mellerowicz et al. 2012, salmen 2009 and others --try
to find original--. Boundary conditions will be initially derived from those
presented by Almeras et. al. (2005) and further modified for increased realism
and/or usability of later models.

One of the major differences between the model presented here and in previous
literature is the inclusion of fibres intertwining macrofibrils. Recently Chang
et al. (2014) measured the mesopore size and shape within tension wood and
opersite wood of poplar during cell wall maturation. With this recent
advancement reasonable assumptions around how regularly fibrils interact with
other fibrils outside of their host macrofibril can be made. It is thought that
these mesopores occur between joining fibrils connecting the macrofibrils into
the larger structure that is the forming cell wall. If the deposition of lignin
into the mesopores forcing the fibrils apart is the mechanism by which growth
stresses develop the quantity of pores and pore sizes are important parameters
to investigate as they will largely effect the ability of the mechanism to cause
stress.

Proposed model of stem:
Because of the nature of the experimental work it is required to be undertaken
at a macroscopic scale, while the proposed theoretical model is at a cellular
(micro-meter) scale. The scale difference between the two methods causes an
issue in that they are not directly comparable (as a sample of wood is not
homogeneous). In order to overcome the scale dependency it is proposed a second
theoretical model be produced which will operate at a macroscopic scale with the
purpose of simulating the experiments undertaken. By parameterizing with the
single cell model (which has been parameterized with the experimentally derived
cell anatomy and geometry) approximations to the actual sample being tested
should be able to be made and compared to the experimental outcomes. This
proofing will make sure that the results the single cell model is providing are
realistic.

\subsection{Experimental work}
Currently neither lignin swelling or cellulose contraction (described in section
---) have any direct experimental evidence. The tension which cellulose is under
on the stem periphery has been directly measured using x-ray diffraction showing
a strain reduction of 0.2\% in cellulose when the stress is released (Clair
2006).

Experimental evidence of the G-layer providing contraction within tension wood
has been presented by Goswami (2008). Longitudinal extension and tangential
contraction were observed when the G-layer was enzymatically removed from
tension wood poplar samples. The S2 layer was reported to have a high MFA (36
degrees) as has been reported previously and for other G-layer producing species
--refs--. Goswami (2008) suggested lateral swelling of the G-layer caused the
contraction.

The primary goal of the set of experiments which will be presented within this
chapter is to attempt to identify which cell wall constituents are controlling
stress generation and how they are controlling stress generation under different
conditions. In order to evaluate stress generation mechanisms a number of
experimental techniques have been identified.

Basic cell wall anatomy and geometry needs to be investigated for the NZDFI
species involved in this project. Where possible literature values will be used
to approximate values for model parameterization.

The following properties are required, however will only be sort from
experimental techniques when it is deemed there is a significant advantage over
available literature values.

The cell wall anatomy of different wood types (tension, normal and opposite)
needs to be investigated for the various NZDFI species (principally E.
bosistoana). The anatomy study will consist of investigating which species
produce a G-layer (microscopy with staining) and the cell wall structure
associated with its production (Electron Microscopy). The cellulose, lignin and
other constituents will be determined for tension, normal and opposite wood
(Acid hydrolysis combined with NMR studies) along with the MFA and the MFA
standard deviation in all three wood types (x-ray diffraction). Fibre diameter,
length and lumen size will also be obtained (microscopy). Within tension wood
the removal of the G-layer (in G-layer producing species) will be needed in
order to determine the secondary cell wall properties of tension wood (enzymatic
removal).

Note that growth stresses for a large number of samples will be collected during
the breeding work, however because of the time consuming nature of the
experimental works presented here only a small number of specimens will be
tested as needed.

In order to produce the three types of wood required two different growth
manipulation techniques are suggested:

Technique one; Young stems (less than three month old growth from coppice) will
be restrained to a loop, similar to Jacobs' loops (Jacobs 1945) and allowed to
grow for approximately one year, with regular adjustments of the restraints to
make sure the cambium is not damaged. From the same plants a second leader will
be selected and restrained to a straight pole to provide normal wood of the same
genetics.

Technique two; Straight one year old stems (from coppice, and seedlings of a
mixture of camadulensis, tricarpa and quadrangulata) will be bent and restrained
and allowed to growth for a further 6-12 months, with regular adjusting of the
restraints to avoid cambium damage. Normal wood samples can be collected from
these stems from wood produced away from the bend site. These plants will be
selected from camaldulensis (reported to produce S1-G tension wood (Baba 1996)),
quadrangualata and tricarpa depending on the suitability of the plants
available, and there ability to produce a G-layer will be investigated with
microscopy with staining --stain ref--.

The following experiment is proposed in order to investigate the proportion of
the stem reorientation that is due to the G-layer. During growth tension wood
production is induced by forcing curvature into the living stem, as described
above. By introducing an enzyme treatment to the plant while it is still
transpiring to degrade the G-layer and reverse any straightening that was caused
by the G-layer. With the G-layer removed the remaining stress can be released
via planking or splitting.

\subsection{Breading}
Because growth stresses cause a number of issues for harvesting and milling
timber, tree breeding programs can and have be used in order to select for
genetics which reduce these effects. There is no reason to expect  breading for
growth stresses differs significantly from (conventionally) breeding trees for
any other trait. Over the last few decades many advances have been made in
experimental and statistical techniques which rapidly improve the time and
accuracy of conventional breeding.

It is suspected that the most efficient way to minimise the issues growth
stresses cause during the production of timber is through appropriate genetic
selection. Eucalyptus species, in particular E. bosistoana are showing promise
within the NZDFI trials to produce high value naturally durable structural
timber. In order to see the yield efficiency required to make this product
profitable, growth stresses need to be reduced to minimise the effects discussed
in section ---. While within the NZDFI project there are a number of other
concerns for breeders (such as durability, form and growth rate) growth stresses
also need to be considered. Using conventional breeding methods discussed below,
growth stresses will be minimised within the NZDFI genetics. Currently a number
of trials have been established or will soon be established, these include:

Permanent sample plots (whole forests) located throughout New Zealand (primarily
in the Marlborough Nelson area). These plots are for profit forestry plantations
ranging in ages up to 8? years old. Because the plots are not solely research
plots, limited testing can  be undertaken on the trees. These trials consist of
the species --need list-- set up as alpha latause trials. Some of the genetic
material is duplicated in other research specific experiments described below.

All trials at Harewood are set out as randomised individual trials. Principally
this work will be concerned with E. bosistoana of which there  are two trials.
One with xx replicates of xx families, planted in 2011 and coppiced in 2013
(check these), due to be harvested in spring/summer 2015. The other E.
bosistoana trial has 10 replicates of 20 families, was planted in (2010?) and
harvested for the first time in 2012, the plants were then coppiced and
harvested again in December 2014. Four families representing the highest and
lowest growth stress generating genetics were coppiced for a second time and
will be due for harvest in 2016. Preliminary results from the 2012 and 2014
harvests show reasonably high heritability of growth strain generated family
rankings. The same data was collected from E. argophloia plants planted in 2010?
measured and coppiced in 2012, with final measurements completed in 2014.  In
ground plantings of E. bosistoana, E. camaldulensis, and--what else-- have been
plated in 2014. Note most of the plantings required to get the material for the
studies outlined in section --- are also grown at this site.

Due to the success of the previous trials NZDFI is currently setting up a xxx
plant trial with xxx replicates of xxx families of E. bosistoana, xxx replicates
of argophloia and possibly globoidea. The trials are set up as alpha lattices
and harvest is expected in late 2016 or 2017. The intention is to have as much
overlap as possible with the existing genetics within the current and previous
trials.

Note the term family is used here to mean the same mother, but not necessarily
the same father. If seeds are collected at different times even from the same
tree variability exists due to possibly of different set of fathers. Also some
Eucalyptus self propagate, but it is unknown which ones or what proportion of
seeds are self propagated within the NZDFI genetic material. Any effects of self
propagation are ignored.

Within the structure of the breeding program there is the ability to
statistically check within family genetics results from young stems and mature
stems, although the trees are grown under different environmental conditions.

In order to select for low growth stress producing families experimental tests
need to be undertaken on each plant. The main test to be used to determine the
extent of growth stresses within the breeding population is the split test (also
known as the Pairing Test) as described in Chauhan (2010) The test involves
taking a significantly long section of stem and cutting along the pith to create
a radial split. The diameter of the stem is taken before testing and the width
of the opening measured immediately after splitting. Once the opening is
measured the stem is cut to across the grain to give two samples (one from each
side of the split). Density is measured by measuring the mass (using balances)
and volume using the displacement method on each of the pieces. Acoustics are
also taken using wood-spec to calculate the dynamic modulus, and hence the
stress can be derived Chauhan (2010). Other properties such as bark thickness
are also recorded for other purposes. Due to the size of the Woodville trial it
may be the case that less tests are carried out. Decisions on the essential
tests will be made near the time of harvesting. Note throughout testing the
samples are kept in a green state.

Although the particular statistical techniques best suited to the data set
obtained from the trial results can’t be known before the data is collected, it
is expected that typical breeding statistical techniques will be applicable. The
two objectives to be achieved for the breeding trials are to estimate genetic
parameters, (in particular heritability of growth stresses) and to predict
breeding values for families. Mixed model methods are commonly used within tree
and animal breeding for these purposes.


\section{Objectives}
The proposed research has two main objectives; add information to the
debate on how growth stresses form within trees and to increase the quality of
the breading stock within the NZDFI project for the production of high value
timber.

\section{Costs}
Travel expenses to 8th Plant Biomechanics Conference - Japan, November
30-December 4th 2015

Travel expenses for use of equipment which cant be sourced at UC,
potently for use of NMR, SEM, ATM.

Travel expenses to Woodville and Marlborough as required.

Bluefern - generally free, unless a very high workload is needed. Unlikely to be
the case. Potently the use of other HPC facilities if they are better suited to
the task.

Sundry items including lab materials and chemicals. Operational
expenses for loop and bending trials at Harewood.

\section{Timeline}

\begin{tabular}{ l l }
\hline
  March 2015 & Set up loop and bending trials \\
   & Proposal due 1 Apr 2015 \\
   \hline
  April 2015 & Harewood analysis \\
  \hline
  May-July 2015 & Harvesting loops and bending trials \\
  & Anatomy study\\
  & Enzyme experiment\\
  \hline
  August 2015 & One year report due 1st September 2015\\
  \hline
  September-October 2015&Write up anatomy and enzyme studies\\
  &Cellular modeling\\
  \hline
  November 2015&Marlborough strain tests\\
  \hline
  December 2015 & Japan \\
  &Re-establish loop trials\\
  & Harewood harvest\\
  \hline
  January 2016&Harewood analysis\\
  &Marlborough analysis\\
  \hline
  February-Apr 2016 & Modelling\\
  \hline
  May-July 2016&Harvesting loops\\
  &Further anatomy study from loops\\
  &Update exp paper\\
  \hline
  August-November 2016&Finish Modeling\\
  &Write up modelling\\
  \hline
  December 2016&Woodville harvest\\
  &Harewood harvest\\
  \hline
  January-February 2017&Woodville analysis\\
  &Harewood analysis\\
  &Analise Marlborough, Harewood and Woodville together\\
  \hline
  March-October 2017&Complete remaining tasks and papers\\
  &submit thesis\\
  \hline
\end{tabular}


\section{Projected publications}
On heritability of growth stresses in Eucalyptus bosistona. Using data from
Woodville, Harewood and Marlborough. --Is this a paper or report to NZDFI?--

Anatomy study from loops, reporting G-layer, MFAs etc --probably letter or tech
note --

On G-layer enzyme removal effect on looped samples

Presenting modeling work

\end{document}