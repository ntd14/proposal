\documentclass{article}

\usepackage{graphicx}
\usepackage[space]{grffile}
\usepackage{latexsym}
\usepackage{amsfonts,amsmath,amssymb}
\usepackage{url}
\usepackage[utf8]{inputenc}
\usepackage{fancyref}
\usepackage{hyperref}
\hypersetup{colorlinks=false,pdfborder={0 0 0},}
\usepackage{textcomp}
\usepackage{longtable}
%\usepackage{multirow,booktabs}



\begin{document}

\title{proposal}

\author{Nicholas Davies\\ University of Canterbury }

\date{\today}

\bibliographystyle{plain}

\maketitle

\section{Introduction to wood structure and formation}
As trees grow they produce wood in order to become taller and wider. Becoming
taller and increasing canopy size is an effective way to out compete the other
trees and plants for light. With increasing height and width comes increasing
weight, wind drag and internal pressures (for water transport), which requires
either enough redundant strength in the existing structure (such as young monocotyledons) or
for the tree to strengthen its structure as it increases its size. In
dicotyledons and gymnosperms this occurs in two ways, apical and cambial growth
on branches, roots and the stem(s).

Softwoods have a simpler micro structure than hardwoods, consisting mainly of
axially elongated pointed cells named tracheids which serve as both mechanical
support structures and water conduits. Although varying with species, softwoods
may also contain radially orientated tracheids, radially or axially orientated
parenchyma cells and other cell types. Tracheids are the dominant form of cells
within the stems and branches.

Hardwoods contain a more complex micro structure with a number
of different cell types. Fibres provide structural support as their primary
function, while similar to softwood tracheids they differ in some key aspects,
being shorter in the longitudinal direction, more rounded in the transverse
outline, tend to have smaller lumens and have little role in sap ascent. However
the ends do taper to points as in softwood tracheids. Libriform fibres tend to
be longer than fibre trachiads, have thicker walls and are solely for support.
Fibre trachiads function in both conduction and support, as in softwoods,
however their appearance in wood with vessels suggests that they function
primarily for support, and perhaps are an intermediate evolutionary feature
between the softwood trachiad and the libriform fibre. Septate fibres devide
their cell lumens into chambers without crossing the primary cell wall. Septate
fibres are produced in the late stages of division just prior to the death of
the cytoplasm, and appear to resemble axial parenchyma cells, and have been
hypothesised to store starches, oils and resins.

Vessels are the main conduits for sap ascent. Vessels are comprised of multiple
vessel elements being joined at the ends to form long conduits, which can extend short
distances (often less than 200mm) or can be as long as the height of the tree.
These elements are connected through pores or perforations in perforation
plates at the end walls of the cells. The arrangement of vessels into groups is
species dependent and usually described as ring porous (the vessels congregate in early wood)
or diffuse porous (vessels are distributed throughout both early and late wood).

Further cell types also exist, such as vasicentric tracheids which have profuse
side wall pitting exhibiting deformation from the expansion of the surrounding
vessels. Axial parenchyma cells are generally abundant and tend to exist in
vertical files and are expected to play a role in the development of heartwood.

Rays are formed from radially orientated cells often tracheids or parenchyma.
Hardwoods typically contain multisteriate parenchyma rays, but there are a
number of species with unisteriate or a combination of ray sizes, comparatively softwoods rearly
contain multisteriate rays. Parenchyma ray cells are living within sap wood,
however during the transition to heartwood die and are used for storage of
extractives. Rays also provide a mechanical advantage...

Note that things like pits etc are not discussed here.

compression and tension wood in stems and branches
In order to reorentat stems and branches of (most) trees produce reaction wood
which provides a force in order to reorentat the tissue. Typicly this
reorentation is toward the light or upwards as is defiend by the negative
gravitrposim hypothesises. Other reasons for reorentation such as reduceing wind
drag have also be sugested. In softwoods this reoretion is caused by the
production of compression wood. Compression wood forms on the out side of the
stem or branch and (expands? so that it is under compression? causeing a
restoring force). Hardwoods on the other hand produce tension wood on the inside
of the desired curve which (contracts?). -- relationship to GS--


detailed hardwood anatomy, euc focus
breifly mention tension wood and Glayer

detailed hardwood fibre anatomy
Primarily, at different resolutions this work focuses on the fibre tracheids as
they are the structual cells expected to be responcable for growth stresses in
normal and reaction wood within hardwoods. The fibre tracheids consist of a
number of cell wall layers depending on the species, the particualar cell and
its primary function. Normal wood fibres within Eucalyptus species (CHECK THIS)
consist of a middle lamaner (conecting the fibre to the sourding cells) a
primary cell wall and a secondary cell wall consisting of S1, S2 and S3 layers
(produced in coronalogical order so the exact composition will change depending
on the cells developmental stage). The S2 layer is the largest layer and
consists of cellullose macrofibrils wraped helicly around the cells longitudinal
axis. This cellulose is contained within a matrix of hemicelluloses (examples)
and lignin. --how does this provide structure--
When tension wood is formed in order to reorentate the stem or a branch
sometimes a G-layer is formed. Notable this occours within the Eucalupt species
--example-- while Nitens does not form a G-layer. --why is this important--

In order for the living cambrial cells to produce wood, each cell must go
through its own birth? growth and death. Because the cambriam (and apical
merastem) are continually deviding it alows for the tree to be a dynamic
structure changing its form to become better adapted to its current enviromental
setting even though large portions (ie the wood) are dead. The transition form
devision through elongation and development to death is expected to play a role
in the development of growth stresses within the stem.

\subsection{Basic cell division}
Dicotyledons and gymnosperms grow in two main ways, upward apical growth and
outward cambial growth.

Note monocotyledons (for example palms) do not produce secondary growth and instead diameter
forms as part of primary growth.

As the cambium is forming fusiform and ray initials are created.
(how are the initials created)
Fusiform initials are short radially and tangetialy with tapered
ends. From the cambial initials cells to the inside create the vertical elements
of xylem (tracheids, vessels, fibers, parenchyma, etc.), while cells outside become phloem.
Ray initials produce horizontal elements (rays).

Cambial cells divide in two ways, periclainal and anticlinal.
Periclainal cell division occurs to the inner and outer of the cambial layers.
As the cell division to the inside occurs the volume of secondary xylem that is being
formed increases the tangential stress on vascular cambium resulting in an
extention of the cambial circumference. Although over time many plants show an
increase in the longitudinal and tangential dimensions of the cambial initials it is
likely that this expansion is mainly facilitated by anticlinal division followed
by the expansion of the daughter cells next to the pedant.

\subsection{Cell formation and elongation}

\textbf{cell elongation/shape change}
Once the primary wall has formed and ..has happened.. rapid elongation occours.
secondary cell walls are produced (and posable the G layer)
GS form as part of this -- discussed in detail later--
\subsection{Cell death}
\textbf{cell death > final cell shape change and chemical constituants}

Note GS
\subsection{Cells and wood in the context of a whole tree}
Wood as a materal within the tree has a number of functions..water
transport..structure..nutrant transport..

The growth stresses that form as part of cell formation are throught to provide
a superiour mechanical structure. Because of the continual formation of new
cells providing growth stresses on the perifery of the stem the older wood which
has completed its formation and cell death must be contracted further with each
new layer of cells atempting to contract. The result of this is the older wood
near the centre of the stem becomes compressed while the newer cells can not
contract the the extent that would leave them in their lowest energy state
remain in tension, until the bond between the old wood and new is seperated
releasing the forces restricting this contraction (and extention in the centre)

rection wood..

relate back to first couple paragraphs


\section{Introduction to growth stresses}

ref to above for cell elongation and death

early work in 20s and related models/theories

lignin swelling

cellulose contraction

hemicelluose theories

yamamotos recent model

issues with current understanding


\subsection{Why growth stresses exist}

hardwoods v softwoods

speculation from various authors

mechanical hypotheses

\subsection{Intro to the issues growth stresses cause }

for harvesting

within mills

\section{Theoretical and experimental understanding of growth stresses}

\subsection{Background of breading}

field techniques

laboratory techniques

stat techniques

mention tradeoff with durability etc

\subsubsection{Beading work in this thesis}
What we actually have:

Harewood trial:
dec 2014 has bosistoana and argophloia copiced from old planting that mon has GS
data from. --from data can show (kind of as could argue same enviromental
effects caused it) genetic relationship.
New Harewood trial, 2016 harvest, will
have a number of species potential to copice bos again if needed.

Woodvile, 2016/2017 harvest will have Bosistoana, argophloia and possibly globoidea.
May or may not be the same families as the various drylands trials.

NOTE family means same mother, not same father.
If collected at different times even from the same tree variability exists due
to possibly of different set of fathers. Also some self propagate, but we don't
know which ones or what proportion, so ignore this.

Progeny trials are alpha latauses, harewood is a standard randomised individual
trial.

Contact Ruth McConnachie: rgcmcconnochie@xtra.co.nz for DFI details.

slit tests
Pairing Test and Longitudinal Growth Strain: Establishing the Association 2008
is the earlyest paper I can find on the split/paring test. note kens papers from
mon

surface tests

Potentually use NIR
http://www.afs-journal.org/articles/forest/pdf/2002/05/05.pdf
Has some useful info on wave lengths associated with bonds ic cellulose

Non-destructive evaluation of surface longitudinal growth strain
on Sugi (Cryptomeria japonica) green logs using near-infraredspectroscopy

statistics

Progeny trials are alpha latauses, harewood is a standard randomised indervidual
trial.

PLSR etc for NIR work

normal breading stats?

\subsection{Background of chemistry work}

lignin swelling

cellulose contraction

what has been done in the past?
that xray syncotron experement etc

\subsubsection{Chemistry work in this thesis}

Do all of the DFI species have a G-layer?
Maybe include some Nitens tests if they don't.
check MFA and SD for S\_2 in tension, normal and compression/opposite wood
Get cellulose lignin and hemicellulose(s) contests for tension normal and
compression/opposite wood Split hemicelluloses where possible, eg xyloglucan
etc.
Torsion tests on individual cells, again for tension, normal and compression. Maybe remove G-Layer
in tension wood and compare to normal and compression wood of similar MFA and compounds etc.

Could we somehow measure growth stress release on a single cell?
Ideally, grow disordered cells invitro, and separate them from the parent cell
as soon as possible, then record when in their formation they undergo what
dimension changes. Is there some non-destructive test to check what is going on in the cell? or if we have multiple cells in the same conditions maybe we can destructivly test some during the growth phase, under the assumption they are all growing at the same time.
OR
remove the cambial layer leaving top and bottom of cell attached to the stem on a large sample,
then somehow remove the connection to the cells behind it, then release the top and measure the contraction.


\subsection{Background of modeling}

yammamotos most resent attempt

possible different methods > FEM, DEM, molecular dynamics, gemomentry of stem
and cells

\subsubsection{Modeling in this thesis}
cells as partials in relaxed state

apply body force, ie the growth stress field

get original/non cut stick back

take take groups of repressive cells and use composite theory and position dependent body force
(growth strain field) from the sub domain above

introduce time dependence to see how the stress field develops during
maturation, composite scale still -- each cell can have its own clock so that it
has a maturation rate to change its field variables.

take individual cells at macromolecular level and try to produce stress field
above during a time dependent maturation function

Molecular dynamics simulations to work out the molecular mechanisms developing the growth stresses

Using the MD sims paramertorize a cell model

Using the cell model develop a time dependent field function

from the field function create representative cell blocks

put the cell blocks together into a stick

cut the stick > do we get out what we put it?

\section{Intentions}
to improve breeding stock for NZ dryland forestry with respect to eucs being used for structural timber

to increase understanding of growth stress formation particularly in eucalyptus
by chemical analysis and computer modeling

\section{Objectives}

to create a mathematical model and computer simulation of a piece of cambium forming growth

stresses at the macromolecular level

to investigate the chemical causes of GSs by chemical analysis >> how?

to improve breeding stock for eucs wrt growth stresses from field and lab testing to select appropriate families.

\section{Costs}

\section{Timeline}

\end{document}