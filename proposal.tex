\documentclass{article}

\usepackage{graphicx}
\usepackage[space]{grffile}
\usepackage{latexsym}
\usepackage{amsfonts,amsmath,amssymb}
\usepackage{url}
\usepackage[utf8]{inputenc}
\usepackage{fancyref}
\usepackage{hyperref}
\hypersetup{colorlinks=false,pdfborder={0 0 0},}
\usepackage{textcomp}
\usepackage{longtable}
%\usepackage{multirow,booktabs}



\begin{document}

\title{proposal}

\author{Nicholas Davies\\ University of Canterbury }

\date{\today}

\bibliographystyle{plain}

\maketitle

\section{Introduction to wood structure and formation}
As trees grow they produce wood in order to become taller and wider. Becoming
taller and increasing canopy size is an effective way to out compete the other
trees and plants for light. With increasing height and width comes increasing
weight, wind drag and internal pressures (for water transport), which requires
either enough redundant strength in the existing structure (such as young monocotyledons) or
for the tree to strengthen its structure as it increases its size. In
dicotyledons and gymnosperms this occurs in two ways, apical and cambial growth
on branches, roots and the stem(s).

Softwoods have a simpler micro structure than hardwoods, consisting mainly of
axially elongated pointed cells named tracheids which serve as both mechanical
support structures and water conduits. Although varying with species, softwoods
may also contain radially orientated tracheids, radially or axially orientated
parenchyma cells and other cell types. Tracheids are the dominant form of cells
within the stems and branches.

Hardwoods contain a more complex micro structure with a number
of different cell types. Fibres provide structural support as their primary
function, while similar to softwood tracheids they differ in some key aspects,
being shorter in the longitudinal direction, more rounded in the transverse
outline, tend to have smaller lumens and have little role in sap ascent. However
the ends do taper to points as in softwood tracheids. Libriform fibres tend to
be longer than fibre trachiads, have thicker walls and are solely for support.
Fibre trachiads function in both conduction and support, as in softwoods,
however their appearance in wood with vessels suggests that they function
primarily for support, and perhaps are an intermediate evolutionary feature
between the softwood trachiad and the libriform fibre. Septate fibres devide
their cell lumens into chambers without crossing the primary cell wall. Septate
fibres are produced in the late stages of division just prior to the death of
the cytoplasm, and appear to resemble axial parenchyma cells, and have been
hypothesised to store starches, oils and resins.

Vessels are the main conduits for sap ascent. Vessels are comprised of multiple
vessel elements being joined at the ends to form long conduits, which can extend short
distances (often less than 200mm) or can be as long as the height of the tree.
These elements are connected through pores or perforations in perforation
plates at the end walls of the cells. The arrangement of vessels into groups is
species dependent and usually described as ring porous (the vessels congregate in early wood)
or diffuse porous (vessels are distributed throughout both early and late wood).

Rays are formed from radially orientated cells often tracheids or parenchyma.
Hardwoods typically contain multisteriate parenchyma rays, but there are a
number of species with unisteriate or a combination of ray sizes, comparatively softwoods rearly
contain multisteriate rays. Parenchyma ray cells are living within sap wood,
however during the transition to heartwood die and are used for storage of
extractives. Rays also provide a mechanical advantage by diverting the
axial force flow reducing buckling and shear stresses between fibres.

Further cell types also exist, such as vasicentric tracheids which have profuse
side wall pitting exhibiting deformation from the expansion of the surrounding
vessels. Axial parenchyma cells are generally abundant and tend to exist in
vertical files and are expected to play a role in the development of heartwood.
More detailed wood anatomy and has little bearing on this project and is
discussed in a number of wood anatomy texts.

In order to reorentat stems and branches of (most) trees produce reaction wood
which provides a force in order to reorentat the tissue. Typicly this
reorentation is toward the light or upwards as is defiend by the negative
gravitrposim hypothesises. Other reasons for reorentation such as reduceing wind
drag have also be sugested. In softwoods this reoretion is caused by the
production of compression wood. Compression wood forms on the out side of the
stem or branch and (expands? so that it is under compression? causeing a
restoring force). Hardwoods on the other hand produce tension wood on the inside
of the desired curve which (contracts?) resulting in a curve forming.
Tradtionly the galaterness layer (G-layer), a layer primerally consisting of low
angle cellulose fibrils on the inside of the fibre tracheids, is credited with forming growth
stresses within the tension wood. However some hardwoods produce tension wood
without producing a G-layer such as E Nitens.

Primarily, at different resolutions this work focuses on the fibre tracheids as
they are the structual cells expected to be responcable for growth stresses in
normal and reaction wood within hardwoods. The fibre tracheids consist of a
number of cell wall layers depending on the species, the particualar cell and
its primary function. Normal wood fibres within Eucalyptus species (CHECK THIS)
consist of a middle laminar (connecting the fibre to the sounding cells) a
primary cell wall and a secondary cell wall consisting of S1, S2 and S3 layers
(produced in coronalogical order so the exact composition will change depending
on the cells developmental stage). The S2 layer is the largest layer and
consists of cellulose macrofibrils wrapped helically around the cells
longitudinal axis. This cellulose is contained within a matrix of hemicelluloses (examples)
and lignin giving the cell wall properties of a fibre reninforced matrix. --how
does this provide structure--

In order for the living cambrial cells to produce wood, each cell must go
through devision from its perant cell, growth and death. Because the
cambriam (and apical merastem) are continually deviding it alows for the tree to be a dynamic
structure changing its form to become better adapted to its current environmental
setting even though large portions (ie the wood) are dead. The transition form
division through elongation and development to death is expected to play a role
in the development of growth stresses within the stem.

\subsection{Cell division, formation, elongation and death}
Dicotyledons and gymnosperms grow in two main ways, upward apical growth and
outward cambial growth. Note monocotyledons (for example palms) do not produce
secondary growth and instead diameter forms as part of primary growth.

As the cambium is forming, fusiform and ray initials are created from the
aplical shoot cells. Fusiform initials are short radially and tangetialy with
tapered ends. From the cambial initials, cells to the inside create the vertical
elements of xylem (tracheids, vessels, fibers, parenchyma, etc.), while cells outside become phloem.
Ray initials produce horizontal elements (rays).

Cambial cells divide in two ways, periclainal and anticlinal.
Periclainal cell division occurs to the inner and outer of the cambial layers.
As the cell division to the inside occurs the volume of secondary xylem that is being
formed increases the tangential stress on vascular cambium resulting in an
extention of the cambial circumference. Although over time many plants show an
increase in the longitudinal and tangential dimensions of the cambial initials it is
likely that this expansion is mainly facilitated by anticlinal division followed
by the expansion of the daughter cells next to the pedant.

During primary wall formation rapid elongation occurs. When the cells devdie
from their perants they remain fixed to their nabiours via the middle lamina.
The intenal hydrostatic (turgor) pressure causes cell expansion. The osmotic flow of
water from the outside the cell to the inside (due to a lower solute concentration
outside the cell than in) which is constrained by the primary cell wall, the
primary cell wall becomes under increasing tension as more water flows into
the cell. Because the centre of the cell has restricted movement, in order for
elongation (to disipate the increasing tensile forces generated from the
inflow of water) to occour the cell turns the biosythesis of cell wall
constituants to produce tip growth. Growth at the tips of the cells allows for
the cells to remain a cosntnat thickness, so no streching is needed during the
elongation phase, as has been sugested previously. The expantion of the cells is
suspected to be controld via modulation of the primary cell wall rather than via
turgor pressure. -- note that primary wall has randomly orentated MFs embeded
in hemicellulose and pectic compounds and becomes lignified after S layer added
, ML is non lignifed, note often compound middle laminer is used to describe
the ML and P at once as are hard to distinguish--- Once the cell has reached its
full size biosynthises of the S1 starts.
Typicly the S1 layer is thin and comprises of very high angle microfibrials,
within the layer many laminates are found. Within each laminate the MFs are
closly aligned, however between each laminate they can (but do not nessassery)
differ greatly, or even reverse the direction of the helix the MFs form around
the cell, although lower right to upper left orentation tends to be favered.
Close to the S2 layer the MFA decreases repidly.
The S2 layer bound to the inside of S1 is typicly much thicker and has more
verticly orentated micorfibrils compeared to the primary, S1 and S3 layers, these MFs circle the
cell axis from lower left to upper right. S2 contains the majorty of the lignin
within the cell. In some cases, most commonly in late wood a thin S3 layer is
also produced with high MFA, reversing the direction of the MF helixs to lower
right to upper left.

Finally if tension wood is being produced a Gelatonus layer
may be produed on the inside of the inner most wall (S2 or S3). The G-layer has
near verticly orentated microfibrils and very little lignification. It is suspected that the
G-layer plays an important role in the generation of reorentation stresses.

At some point during the formation of the seconday cell wall, or soon after the
cell shrings verticly and expands tangentially. Because of the connectivness
between cells this results in growth stresses forming within the stem, this
phenomonan is descussed in greater detail in ----. After the seconday wall
formation cell `death` occours as part of the transition from sap wood into
heartwood. While the hollow, dead cells play an importnat role in water
transport and mechanical support of the tree, over time any residual nutrant
that can be used by living cells--- heatwood stuff----

What is the deal with Rays----

\subsection{Cells and wood in the context of a whole tree}
Wood as a materal within the tree has three major functions to achieve; water
transport, nutrant transport and mechanical struture. Softwoods achieve
water transport and mechancial struture within trachieds, while parenchima cells
are used for nutrant transport. Hardwoods have evolved a more complicated
internal structure of vessels and fibre tracheids in order to separates out the
functions of water transport and mechanical support respectively.

--advantages and disavantages of this--

The growth stresses that form as part of cell formation are throught to provide
a superiour mechanical structure. Because of the continual formation of new
cells providing growth stresses on the periphery of the stem the older wood
which has completed its formation and cell death must be contracted further with each
new layer of cells attempting to contract. The result of this is the older wood
near the centre of the stem becomes compressed while the newer cells can not
contract to the extent that would leave them in their lowest energy state
remain in tension, until the bond between the old wood and new is separated
releasing the forces restricting this contraction (and extention in the centre)

Reaction wood as described above provides the ability for the stem to reorentate
in order to be best adapted to its environment at any given time.

These properties of wood allow for an adaptive organism to survive..

\section{History of work on growth stresses}

It is suspected that growth stresses develop within trachieds during the
formation the secondary cell wall, although the exact timing and mechanism for
developing growth stresses is still of much debate. The most current ttheoryis
a hybrid of the older cellulose contraction and lignin swelling hypothesis.

A breif discussion of work relating to growth stresses prior to 1965 is given
below, Archer (growth stresses book intro) provides a full review of suggested theories
up until 1965.

Wood workers have unintentionally known of growth stresses within trees
for centries. Usually refered to as `a pull towards the sap` when cuting boards good
craftsmen would section the log in such a way as to get a stright board once it
is removed from the log (and the growth stresses released). Most work early on in
the study of growth stresses surrounded investigating how/why boards changed
shape when cut from an intact stem.

Martley (1928) was possibly the first to study growth stresses in a scientific
manner. Initally he argued that the curvature of planks sawn from logs was due
to the current growth not being able to support the dead weight of the tree until
lignification was complete. As a result the centre is under compression while
the periphery had zero stress. However calculations showed that the self weight
was not sufficient to cause the observed longitudinal dimension changes of the
timber.

After Martley's work a small number of authors investigated growth stresses
through the 30's and 40's. Jacobs, although testing 34 hardwood species, focused
mainly on Eucalyptus and in 1938 argued that (longtudinal) tension successively
develops in the outer layers of the stem as it grows, and as a consequence of of
the tension, compression must form in the centre of the stem. Jacobs later used
E. gigantea to descibre a strain gradient developing during growth.
Experementally Jacobs made use of strip planking, measuring the deflection of
the board after removal from the log, and the length change when the planks were
foced back straight. He showed that wood tends to shrink in the longitudinal
direction at the periphery while extend near the pith (indicating in the log
the planks are under compression in the centre and tension at the extremities).

Further Jacobs put foward a number of hypothesis to explane how the growth
stresses were forming. First arguing that it is very unlikly that dead cells
(wood) could extend within the core in order to create the observed stress
gradiant. Instead sugesting the causes of; weight of the tree, surface tension
and sap stream forces, cellulose and colloidal complexes, lignin intercellular
substances and the primary or secondary cell wall. Although without any evidence
did not claim any of these to be the major cause.

Stresses relating to reaction wood received more attention through the 30s and
40s for both soft and hardwoods. Jacobs 1945 stated that the reorientation of
stems is caused by a modification to the already existing stress gradient
throughout the stem. One option he presented was simply that the eccentric
growth causes larger number of cell sheves to be added to the upper side of the
curve each providing the same amount of contraction force, this results in a
angle correction even with identical cells. Sap tension is also considered, but
more importantly Jacobs notes the posability of tensions being formed within the
cell walls of tension wood.
Munch 1938 specualted that the addition of matter into the cell wall could cause
compression wood. ..
Jacobs 1945 also found that it was commanly the case that the amount of
compression wood developed and the stem angle recovery had a poor relationship.
He sugested maybe it was infact the normal strain pattern in tension which
correct the lean, the compression wood mearly acted as a pivot, not contributing
a tensile force on the lower side of the stem.

Boyed 1950 Developed a new expemental techneque in order to investigate the
stress profile further. By cutting a slit longitudinaly in the centre of the
log, attaching strain gauges onto the wood inside the slit and sucseivly
shortening the log from both ends he obtained direct extention measurments from
inside the stem. --found that the crossover point is is about 1/3 rad of the log
from the perifphery--

Most commanly growth stresses were investigated from the longitudinal direction,
however cells also change dimention in the transverse direction, this leads to a
more complicated three dimetional stress feild developing within even a straight
stem.
koehler 1933 showed that a saw cut radially through a disk has a tendancy to
close near the perifery sugesting that the periferal cells are under tangential
compression with the inner cells under radial tension. He sugested this was the
cause of shakes in standing timber.
jacobs 1945 removed inner circals from disks of a number of speceis and found
when an inner portion is removed the disks cercunfrance incresses. Jacobs again
argued that strain in the sap stream along with cells being wider tangentally
than radially led to the observed lateral stresses. Although he also mentions
the posability of secondary thickening from the deposition of lignin as a
posabe contributing factor.
boyed 1950a developed and experement whereby he removed a wedge from a disk and
meaured the radial expantion, showing the disks were infact under radial
tension. Further aditinal species were found to be in agreement with the results
of jacobs 1945 when the inner circils were removed from disks. Boyd also shows
that the longitudinal stresses maifesting as transverse stresses via poisson
ratios are only aproximatly one tenth that of the measured stresses.

Boyd 1950c provides an indepth rebutel of the available theries at the time,
arriving at the conclusion the the cell wall development must control the shape
change which results in growth stresses. Further he postulates that cellulose
is primarily responsible with lignin and carbohydrates also playing important
rolls when stresses are formed in normal, compression and tesnion wood.

wardrop 1965 commented that a tensile stress generated in the cellulose
transitioning into a crystaline state could be the explination for cells
contracting during the formation of the secondary wall. Cellulose contraction
alighned well with the observation of the G-layer (which has  a very low
MFA) being comman in a number of tension wood producing species, and also gave
the ability for low MFA normal wood to contract. Bamber 1978 further argued
cellulose contraction claiming turgor pressure in normal wood cells remained
high enough that the cells did not contract before the lignin was deposited,
once/during lignin deposition the cellulose became crystaline and shrunk,
causing the cell to become shorter, the mechanisum for tension wood is
essentually the same. Compression wood on the other had was explaned by the
cellulose being layed down and then the turgor pressure decreasing, causing the
cell to contract before lignin was deposited. In turn the cellulose was under
compression, resulting in the tendency for the compression wood cells to
expand.

Boyd (1972) presented (or rather poplerised) the alternative (more widly
accsepted) hypothesis of lignin swelling (first conceved by Munch 1938). Tensile
stress is gained in cells of low MFA by lignin deposition into the cell wall,
pushing the cellulose fibrils appart, which in tern shrinks the longitudinal
length of the cell and incresses the tangential width. When MFA is high, the
opposite occurs, lengthening the cell and reducing its tangental width. This
shape change is not readily apparent in compression wood (characterised as short fat
trachaids) until the release of the stress acting on the CW, where by the cells
become longer and skinnier.

Around the same time two other lesser known hypothesis were presented,  strains
due to changs in water content Hejnowicz 1967, argued that the stresses in
compression wood are related to the inhibition of water by the cell walls,
which results in swelling, because the expansion of compression wood is equal to
the shrinkage due to drying. --paper disproving this--

brodzki 1972 hypothesised strains due to 1,3-linked glucan (laricinan)
deposition within the helical checks of the S2 cell wall layer could be the
most significant factor in longitudinal growth stress generation. Boyd 1978
refuted this idea arguing (along with other issues) that the laricinan would
expand into the cell luman not casuing any stresses in the cell wall, unless a
(non-observed) constraining meadian restricted the expansion.

---- Gills 1973

Through the late 70's and 80's archer produced a number of papers in two series,
`on the distribution of growth stresses' --refs-- mainly concerning the
mathamtical treatment of the stress feilds within trees. --- and `on the
orign of grwoth streses' ---refs--- primarally concerned with the underlying
mechanisums generating growth stresses.

The `on the dirstrobition of growth stresses' series presented a comprehensive
mathematical framework for the treatment of the stress feild within living
trees. Advancing on Kublers work Archer introduced orthotropic solution which
allowed for each new growth increment to alter the stress distribution within
the stem in a self equlibrating fashion. The other advancement made was the
increased acuracy from the crossover point from compression to tension now being
goverened by the moduli in both the radial and circumferential directions.
Archer went on to develop a numerical approximation to the stress fields
generated by asymmetric growth strains and inclind grains, allowing for
variation within growth stresses. Finely he used the developed methods to
present solutions for a number of hardwood species.

`on the origin of growth stresses' Archer investigated the mechanisums behind
growth stress generation.

More recently theories regarding the nature of hemicelluloses and their bonding
have been used in an atempt to remove some of the issues associated with the
cellulose contraction hypothesis. One major issue of callulose contraction is
that in its initial form it was argued that the crystallisation process of cellulose
shortend its length. --ref-- showed that when cellulose crystallised it became
longer as the chains increased order. Two theories have been advanced to combat
the issue of lengthinging during crystallisation in order to retain an updated
version of the cellulose contraction hypothesis.

--- argues that at the edge of the cellulose fibrils the cellulose becomes
dissordered and is concequently able to bond with hemicelluloses, which have a
slightly shorter repeate length than the cellulose crystel. These hemicelluloses
bondend to the outside of the fibril cause the fibil to be compressed in the
cystaline centre, while under tension on the surface. An intersting concequnce
is the contraction of the cellulose due to the hemicellulose bonding should be
dependent on the area/volume to circunfrance/suface area ratio. A potentual way
to test this hypothesis is duscussed in section ---

The seccond theory put foward in an attempt to correct the issues souronding
cellulose lengthenging during crystalisation is from --- who argues that
hemicelluloses form within the fibrils and push them appart causeing the
cellulose fibrils to contract. Interestingly mechanicly this is very similar to
the lignin swelling hypothosis. By in causing the MFs to no longer run stright,
instead they have to use some of their length to devate passed a culster of
hemicelluloses concequently shortening the over all distance the fibril can
cover. One side effect of having these devations is fibrils should not have a
consistant cross sectional area over their whole length, where the
hemicelluloses have been deposited should result in an increased cross section.
potential way to test this?--

The most resent attempts made to describe the formation of growth stresses have
been made by yamamoto and his team. They argue a combination of both lignin
swelling and cellulose contraction is needed, called the unified hypothesis. It
should be noted that --- and others sugested that both theories were likely to
contribute to growth stress generation. By unifing the hypothesises they follow
the current thinking of a number of authors in other areas of wood science that
both tension and compression wood are not distictly different types of wood and
are instead extreme versions of normal wood. ---more from yammamoto---


--- lots more in here from 80s-now ---
Note Muller et al 2006 found low hemicellulose content in G-layer
timell 1969 higher conc of lignin in s2 layer when G-fibres present

There are currently a number of outstanding issues assocated with any (or all)
of the current hypothises/theories. When and how do the stresses get
generated is still of much debate, over the last couple of decades it has become
fairly widly accsepted that the generation of the stresses occours during or
imediatly after the deposition of the seconday cell wall. Most commanly either
the G-Layer or the S2 layer are considered responcable. What the mechanisum(s)
is within the cell wall has been hypothosised about at great length (as
discussed above), however no theory presented so far is without contry
extpermental evidence.

Another outstanding issue, common to many biological problems is why do
particular trates vary so much between indervidualls and species? One of the
more debated topics around growth stress generation is whether the generation
mechanisums for stress in reactionwood are extreme versions of the same
mechanisums in normal wood. The G-layer is not found in normal wood, however on
rear ocastions .. .. lignin swelling could potentually fit this criteria for
normal and compression wood, however modification of boyds theory would be
needed due to the depence of a MFA as some wood with lower than about 40 degree
MFA still produces compressive forces. Boyds theory combinded with excessive
mild compression wood formation in corewood still alows for the same tensile
generation mechanisums to be used by older cambriams, as long as the MFA is
suited to the task.

It is farly well acsepted (although almost by default) that growth stresses
exist because they provide a mechanical advantage for servival. However to
quantify the mechanical advantage with so much varablity between inderviduals,
and no known way of controling growth stress generation this is very difficult.

Growth stresses studies have been largly confined to model, or comman species
however there are a number of species which apear to form intermediates or
`strange` forms of reaction wood. For example hebe is a angiosperm which apears
to form compression wood rather than tension wood.

\subsection{Why growth stresses exist}
Hardwoods typicly have much larger growth stress magnitudes than softwoods.
--why-- is this true 'xylem cell development'?--- some have claimed conifers
have compression throughout the stem when young, not until old that they follow
the same trend as hardwoods explanes the low/negative GS sometimes reported in
young conifers.

Prehapse the leading argument for the reasion of growth stresses existance is
the mechanical hypothesis. The mechanical hypthesis argues that a number of wood
properties, inclusing the development of growth stresses evolved in order to
provide increase mechancial stability to trees in order to increase their
servival. The mechanical hypotheisis as applied to growth stresses argues that
because wood is stronger in tension than compression by preloading the outer
edge of the stem in tension it increaseing the non-destructive bending radius on
the inside of the curve when a force is applied causing the stem to bend.
--dosnt explain why hardwoods have larger GS than softwoods, or why young
softwoods exibit compression.-- This hypothesis struggles to explane the
differences between hard and softwoods, particually at young ages. If mechancial
stability is infact the driver for growth stress generation at young ages, why
do young angiosperms and genosperms produce esentually opersite solutions?.

If the reason conifers have compressive forces when young is excessive
compression wood to enable reorentation, maybe this idicates that normal wood is
more colosly realated to tension wood than compression wood. What forms of mild
tension wood are known of?

speculation from various authors
Typicly when atempting to determain the reasons for why wood properties exist
one of four hypothesis are used; mechanical, hydrolic, time dependent and a
combonation of the previous three. Inital speculation as the the reason for
growth stresses existance came from Martley (1928) who breifly entertained the
mechancial hypothesis based on self weight. Jacobs (1945) sugested they were a
biproduct of sap tension, which he later retracted Jacobs (196?) when sap
pressures were recalculated at a much lower value than the genrally beleived
values at the time. .. Growth stresses undobutable have an effect on the
mechancial stability of trees, although it is consevable that the effect may be
biproduct of another driver.


\subsection{Issues growth stresses cause }

At harvesting growth stresses are released by the saw cut (and crosscutting etc)
and can ruin structural and veneer logs due to the resulting spliting and
warping. Growth stresses, particularly reaction growth stresses increase the
danger for the faller by effects such as saws binding and `barber chairing`.

When the stem is felled or cross cut, growth stresses are released around the
saw cuts causing shortening at the perifery and extension in the centre. The
dimention change is maximum at the saw cut, reducing as distance from the cut
increases. When the contraction/extention force exceeds the yield limit of the
stem splitting occurs. The cracks tend to propagate in the radial direction by
cell wall pealing, although the cracks tend to only be a few centimeters deep
they significantly reduce the value recovery for both structural and pealer
logs.

Prolonged compression at the centre of the stem during growth can exceed the
elastic limit of the wood, resulting in internal defects such as brittle heart.
When the stem is felled these defects have already occoured and hence there is
no way to prevent them during felling, however selection for low growth stress
producing families should significantly reduce the occerance of internal
defects.

Within mills during processing growth stresses cause a number of issues leading
to reductions in value recovery and efficantcy. Because growth stresses are
releaced when the stem is sectioned via quarter sawing or --- the resulting
shape change can cause the saws to jam. The main value loss at this stage of
processing come from the need to saw boards multiple times in order to release
the stresses while still alowing for the final board dimentions to be retreved.
Increaseing the number of times the boards are sawn to get their end dimentions
gives not only poor saw use efficanty but the major economic loss comes from the
final yield being less than 30%.


\section{Proposed theoretical and experimental work}

\subsection{A proposed modification to the lignin swelling hypotheses}
The lignin swelling hypotheses argues the deposition of lignin into the
secondary cell wall forces the cellulose fibrils away from each other, because
cellulose is very stiff when it bows becuase of the lignin pushing the fibrils
apart the cell changes shape based on the MFA. One of the main arguments for the
use of the cellulose contraction hypothesis is that the G-layer is mainly
crystaline cellulose and hence is not effected by lignin swelling. However if
the outer of the cell is constrained transversly by high MFA fibrils (as in the
P and S1 layers) and surrounding cells, when lignin is deposited in the S1, S2
and/or S3 layers causing swelling cell expansion will occour toward the cell
lumen (as long as the MFA is conducive to transverse swelling). To create the
maximum amount of contraction from the G-layer there will be an optimum MFA
within the secondary wall (most commonly the S2 layer) dependent on cell
geometries, around 40 degrees. Further because the G-layer is contracting the
cell, it requires the rest of the cell to be as flexable as possible (without
compromising the cells other properties). Flexible cells commonly exhibit higher
MFAs than stiff cells within the S2 layer. By having a non-stiff structure the
G-layer can cause more contraction on the individual and surrounding cells while
under less bowing from lignin swelling. Therefore the optimum MFA of the secondary
wall (excluding the G-layer) will provide the G-layer with the maximum ability
to contract the cell when MFA is in the mid range, i.e. non-stiff and maximum
transverse swelling for minimal longitudinal dimention change.

The qualatative basis of the lignin swelling hypthosis in its currnet form
remains unmodified and can still acount for normal and compression wood growth
stresses.



\subsection{Theoretical work}

yammamotos most resent attempt

possible different methods > FEM, DEM, molecular dynamics, gemomentry of stem
and cells

% cells as partials in relaxed state
%
% apply body force, ie the growth stress field
%
% get original/non cut stick back
%
% take take groups of repressive cells and use composite theory and position dependent body force
% (growth strain field) from the sub domain above
%
% introduce time dependence to see how the stress field develops during
% maturation, composite scale still -- each cell can have its own clock so that it
% has a maturation rate to change its field variables.
%
% take individual cells at macromolecular level and try to produce stress field
% above during a time dependent maturation function
%
% Molecular dynamics simulations to work out the molecular mechanisms developing the growth stresses
%
% Using the MD sims paramertorize a cell model
%
% Using the cell model develop a time dependent field function
%
% from the field function create representative cell blocks
%
% put the cell blocks together into a stick
%
% cut the stick > do we get out what we put it?


Proposed model of cell:
Modelling of a generic single cell with verable cell wall properties to
investigate the required geometry and constituents to create maximum
longitudinal and tangential extension and contraction via the lignin swelling
hypothesis. The single cell model should have the capacity to put limits on what
stress generation the lignin swelling hypothesis is theoretically capable of.

Because the proposed experiments induce servear tension wood in species both
with and without G-layers and experemental upper limit of the lignin swelling
hypothesis should be reached and compared to the theoretical one derived above.
By including the G-layer (assuming the experimental work shows the G-layer is a
contributing factor to the production of growth stresses in tension wood) within
the model (by implementing the hypothesis above), and comparing the required
chemical make up and cell geometries with the servear tesnion wood
experimentally investigated, light should be shed on the liklyhood of the lignin
swelling hypothesis being extenable to include G-layer type tension wood.

Time permitting if experements find that the lignin swelling models dont account
for the strains present a previssor for cellulose contraction could be included.
Between the experemental results and the cellular models, calculations as to how
much contraction is needed should be posable, giving an initial starting point to
look more in depth into cellulose contraction mechanisms.

Proposed model of stem:
Because of the nature of the experimental work it is required to be undertaken
at a macroscopic scale while the proposed theoretical model is at a cellular
(nano to micro meter scale). The scale difference between the two methods causes
an issue in that they are not directly comparable (as a sample of wood is not
homogenious). In order to overcome the scalar dependecy it is proposed a second
theoretical model be produed which will operate at a macroscopic scale with the
perpous of simulating the experiments undertaken. By parametrised with the
single cell model (which has been parametrised with the expermentally derived
cell anatomy and geometry) aproximations to the actual sample being tested
should be able to be made and compared to the experimental outcomes. This
proofing is required to make sure that the results of the single cell model is
providing are realistic.

---Potentual methods to build these models---

\subsection{Experemental work}

lignin swelling

cellulose contraction

what has been done in the past?
that xray syncotron experiment etc

The primary goal of the set of experiments which will be presented within this
chapter is to attempt to identify which cell wall constituents are controlling
stress generation and how they are controlling stress generation under different
conditions. In order to evaluate stress generation mechanisms a number of
experimental techneques have been identified.

Basic cell wall anatomy and geometry needs to be investigated for the NZDFI
species involved in this project. The cell wall anatomy under different wood
types (tension, normal and opersite) needs to be investigated for the verious
species (pinciply E.B.). The anatomy study will consist of investigating which
species produce a G-layer and what cell wall structure is assocated with its
production. The cellulose, lignin and hemicellulose contents will be determined
for tension, normal and opposite wood along with the MFA and the MFA standard
devation for a number of samples in all three wood types. Fibre width, length
and luman size will also be obtained. Within tension wood the removal of the
G-layer (in G-layer producing species) will be needed in order to determain the
seconday cell wall properties of tension wood.

The cell wall constituant study results will be used to make comparisons between
the growth stresses produced by the stems and the different properties within
the cell walls. With the anatomy results collected from tension, normal and
opersite wood comparisons can be made not only between these within trees but
also between trees. By comparing tesnion wood with the G-layer removed with
normal wood with similar properties some insight into the role of the G-layer
should be gained. Note that growth stresses for a large number of samples will
be collected during the breading work.

In order to produce the three types of wood required three different growth
minipulation techneques are sugested:

Techneque one; from young (less than three month old growth from coppice) will
be retained to an arch, similar to Jacobs loops and allowed to grow for
aproximatly 1 year, with regular adjustments of the restraints to make sure the
cambriam is not damaged by them. From the same plants a second leader will
be selected and restraiend to a stright pole to provide normal wood of the same
genetics. Currently there are 12 E.B.? plants set asside for this project.

Techneque two; from young (less than three month old growth from coppice) two
stems will be selected, one grown upright and the other staked and put on
as servear lean as posable. 8 E.B. plants have been set asside for this.

Techneque three; xxx stright one year old stems (from copice of a mixture of
camadilencia, tricarpa and quadrangularta) will be bent and restrained and
allowed to growth for a futher 6-12 months, with regular adjusting of the
restraints to avoid cambium damage. Normal wood samples can be collected from
these stems from wood produced before minipulation and away from the bend site.
These plants will be selected from camaldulensis (reported to produce S1-G
tension wood), quadrangualata and tricarpa depending on the suitability of the
plants available.

The other set of experements proposed is to investigate the extenet of an effect
the G-layer has on tension generation within tension wood, and how the G-layer
generates these tensions.

Proposed experement one:
Taking samples with G-layers and applying a vacumme pump to suck callase into
the fibres and vessels to degrade the G-layer releaseing the strain which the
G-layer is applying to the samples will cause a shape change. By comparing the
inital and final shape change the strain the G-layer was imposing on the samples
can be obtained. Further the rest of the Growth stresses can be realeased using
more traditional techneques such as split tests or planking, by releasing the
reamining strain the proporton of stress assocated with the G-layer and other
cell wall components (assumed to be the S2 layer) can be determined.

Proposed experement two:
Release the stress with a split or planking test, then remove the G
layer using the same method as above. When the tension caused by the G-layer is
released relaxation back towards the the inital state should be observed. The
proportaion of G-layer induced strain and S2 induced strain will then be
evedent.

Proposed experement three:
During growth induce tension wood production by forceing curvature into the
living stem. By introducing callase to the plant while it is still transpiring
should degrade the G-layer and revered any straigthining that was caused by the
G-layer.

Any of the three proposed experements, if they work will provide the proportions
of the strain in tesnion wood which can be atributed to the G-layer.

Any experements where by we could show the mechanisum of the G-layer?
% Take pure cellulose measure the repeate length of the cellulose units, then add
% hemicellulose to it and measure the repeate length, does it contract. Idea being
% that hemicellulose bind to the dissorded cellulose on the outside of the fibril
% which have a smaller chain length, resulting in the fibril being under tension
% on the outside and compression on the inside. outside hemi under tension, inside
% crystiline cellulose under compression. Could also calculate this by working out
% the force required to shorten a hypothetical fibril of crystyline cellulose,
% then trying to find a hemicellulose bonding scheme which would provide it.
% This could be checked by taking different diameter fibrils, resulting in
% different surface to volume/area ratios should result in different contractions.

Paper claiming camaldulensis has G-layer, note that it is a S1-G cell
Chemical and anatomical characterization of the tension wood of Eucalyptus
camaldulensis L. Mokuzai Gakkaishi

% Could we somehow measure growth stress release on a single cell?
% Ideally, grow disordered cells invitro, and separate them from the parent cell
% as soon as possible, then record when in their formation they undergo what
% dimension changes. Is there some non-destructive test to check what is going on
% in the cell? or if we have multiple cells in the same conditions maybe we can
% destructivly test some during the growth phase, under the assumption they are
% all growing at the same time.
% OR
% remove the cambial layer leaving top and bottom of cell attached to the stem on
% a large sample, then somehow remove the connection to the cells behind it, then
% release the top and measure the contraction.



% looks like
% H c c c c H
% H c c c c H
% H c c c c H
% where the H-H bonds are shorter than the c-c bonds (H and c stand for hemi and
% crystyel)
%
% other comman theory is that hemicelluloses form inside the cyrstel and push the
% chains apart shortening the length
%
% c c c c
% c c c c
% c c c c
% becomes
%   c c c c
% c c H c c
% c c H c c

% to experementally test lignin swelling is there a way to remove only the lignin
% from the cell wall? then see how much the wood/cell extended, then cut it again
% and see if it extends more. --not yet--



\subsection{Breading}
Because growth stresses cause a number of issues for harvesting and milling
timber tree breading programs can be used in order to select for genetics which
reduce these effects. --previous breading for GS-- There is no reasion to
expect  breading for growth stresses differes significanly from (convetanally)
breading trees for any other trait, which is process which has been developed
over centries. Over the last few decades many advances have been made in
experemental and statistical techneques which rapidly improve the time and
acuracy of conventinal breading.

It is suspected that the most efficent way to minamise the issues growth
stresses cause druing the production of timber is through aproprate genetic
selection. Ecualiptus species, in particular bosistona are showing promise
within the NZDFI trials to produce high value natrally durable structural
timber. In order to see the yield efficanicies required to make this
product profitable growth stresses need to be reduced to minamise the
effects discussed in section ---. While within the NZDFI project there are a
number of other concerns for breaders (such as durability, form and growth
rate) growth stresses also need to be concidered. Using conventional breading
methods discussed below growth stresses will be minamised within the NZDFI
genetics. Currently a number of trials have been established or will soon be
established, these include:

Permenent sample plots (whole forests) located in ----. These plots are for
profit forestry plantations ranging in age from --- to ---. Because the plots
are not specificly research plots limited testing can be undertaken on the
trees. These trials consist of the species ---- set up as alpha latause trials.
Some of the genetic material is duplicated in other research specific
experements described below.

The Harewood trial:
All trials at Harewood are set out as randomised individual trials.
Principly this work will be concerenced with E. bosistoana of which there are
two trials. One with xx replicates of xx families, planted in (when?) and
copiced in (when), due to be harvested in spring/summer 2015. The other E.B.
trial has xx replicates of xx families, was planted in (when?) and harvested for
the first time in 2012, the plants were then copiced and harvested again in
december 2014. Four families representing the highest and lowest growth stress
generating genetics were copiced for a second time and will be due for harvest
in 2016. Preliminary results from the 2012 and 2014 harvests show reasonably high
heredatibility of growth strains, particually within family rankings. The same
data was collected from --- E. argophloia plants planted in --- meausred and
copiced in 2012, with final measurments completed in 2014.  In ground plantings
of ----- have been plated in 2014 and will continue in 2015 with bosistana. Note
most of the plantings required to get the material for the studies outlined in
section --- are also grown at this site.

Woodvile trial:
Due to the success of the previous trials NZDFI is currently setting up a xxx
plant trial with xxx replicates of xxx families of E.B, xxx replicates of ----
argophloia and possibly globoidea. The trials are set up as alpha latauses and
harvest is expected in late 2016 or 2017. The intention is to have as much
overlap as possible with the existing genetics within the current and previous
trials. ---clones?---


Note the term family is used here to mean the same mother, but not necessarily
the same father. If seads are collected at different times even from the same
tree variability exists due to possibly of different set of fathers. Also some
eucaplitus self propagate, but it is unknown which ones or what proportion of
seats are self propagated within the NZDFI genetic material. Self propogation
can be ignored ---why---

One of the criticisms from breeders about the experimental approach used here is
the assumption `What happens in young trees will happen in older trees'. The
technique has been used previously ---- and shown to work well. Within the
structure of the breading program there is the ability to statistically check
within family genetics results from young stems and mature stems. ---- need to
reword this--- Growth stress measurements are a good candidate for this type of
selection as the expected biological processes underlying growth stress
generation are age independent (bar the dependence of some influencing
factors assocated with the tipical radial pattern of density, MFA etc). --- more
here---

Experemental tests for breading:
In order to select for low growth stress produing families expermental tests
need to be undertaken on each plant. The main test to be used to determian the
extent of growth stresses within the breading population is the split test
(also known as the Pairing Test) as described in ---
Establishing the Association 2008 is the earlyest paper I can find on the split/paring test. note kens papers from
mon
The test involves taking a significantly long section of stem and using a saw
cutting along the pith to create a radial split. The diameter of the stem is
taken before testing and the width of the opening measured imeditly after
spliting. Once the opening is measured the stem is cut to across the
grain to give two smaples (one from each side of the split). Density is
measured by taking measuring the mass (using balances) and volume using the
displacment method on each of the pieces. Acoustics are also taken using woodspec
to calculate the dynamic modulus, and hence the stress can be derived. Other
properties such as bark thickness are also record for other purposes. Due to
the size of the Woodville trial it may be the case that less tests are caried
out. Decisions on the essentual tests will be made near the time of harvesting.
Note throughout testing the samples are kept in a green state.

Potental for the use of modern technologies such as NIR, particually portable NIR
to investigate older trees without destructively testing them may be of use. NIR
has shown promise in predicting longitudinal surface strain in Sugi green logs.
Non-destructive evaluation of surface longitudinal growth strain
on Sugi (Cryptomeria japonica) green logs using near-infraredspectroscopy.
Surface tests such as described by ---- will be requried to either calibrate an
NIR predictive model, or if NIR is unfesable to aquire surface strain values on
older specimans.

statistical methods:
Although the particular statistical techneques best suited to the data set
obtained from the trial results cant be known before the data is collected, it
is expected that ----

normal breading stats

PLSR etc for NIR work
If NIR does prove to be a valuable tool for nondestructive testing of surface
growth strains a predictive model will need to be built from a training set. It
is common practise with the use of NIR to use multiple regression typed
statistical models. In particular Partial Least Squares Regressions are
commanly used. While the most appropriate method to use can not be known at this stage is
is not expected to be particularly exotic, and will be chosen from well tested
and reviewed methods.

Contact Ruth McConnachie: rgcmcconnochie@xtra.co.nz for DFI details.

\section{Intentions}

\section{Objectives}

\section{Costs}

\section{Timeline}

\end{document}