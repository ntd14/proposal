\documentclass{article}

\usepackage{graphicx}
\usepackage[space]{grffile}
\usepackage{latexsym}
\usepackage{amsfonts,amsmath,amssymb}
\usepackage{url}
\usepackage[utf8]{inputenc}
\usepackage{fancyref}
\usepackage{hyperref}
\hypersetup{colorlinks=false,pdfborder={0 0 0},}
\usepackage{textcomp}
\usepackage{longtable}
%\usepackage{multirow,booktabs}



\begin{document}

\title{proposal}

\author{Nicholas Davies\\ University of Canterbury }

\date{\today}

\bibliographystyle{plain}

\maketitle




\section{Intro to wood structure and formation}
\textbf{basic wood structure of hardwoods, breifly tuch on differnces with softwoods}

\textbf{basic cell devsion}
Dicotyledons and gynosperms grow in to main ways, upward apical growth and outward cambial growth.

Note monocotyledons (for example palms) do not produce secondary growth and instead diamiter
forms as part of primary growth.

As the cambiam is forming fusiform and ray initals are created.
(how are the initals created)
Fusiform initals are short radialy and tangetialy with tapered
ends. From the cambial initals cells to the inside create the vertial elements
of xylem (tracheids, vessels, fibers, parenchyma, etc.), while cells outside become phloem.
Ray initals produce horozontal elements (rays).

Cambial cells devide in two ways, periclainal and anticlinal.
Periclainal cell devision ocours to the inner and outer of the cambil layers.
As the cell devision to the inside ocours the volume of seconday xylem that is being
formed incresses the tangential stress on vascular cambiam resulting in an
extention of the cambial cercomfrence. Although over time many plants show an
increase in the longitudinal and tangential dimentions of the cambial initals it is
likely that this expantion is mainly ficilited by anticlinal devision followed by the
expantion of the daughter cells next to the perant.

\textbf{cell formation}

\textbf{cell elongation/shape change}

\textbf{cell death > final cell shape change and chemical constituants}

\textbf{cells/wood in context of wood and whole tree}




\section{intro to what growth stresses are}

ref to above for cell elongation and death

early work in 20s and related models/theories

lignin swelling

cellulose contraction

hemicelluose theories

yamamotos recent model

issues with current understanding



\section{why growth stresses exist}

hardwoods v softwoods

speculation from verious authors

mechanical hypotheseis

\section{intro to the issues growth stresses cause }

for havesting

within mills

\section{background of breading}

feild techneques

labratory techneques

stat techneques

mention tradeoff with durability etc

\subsubsection{beading work in this thesis}
What we acually have:

Harewood trial:
dec 2014 has bosistoana copiced from old planting that mon has GS data from.
New Harewood trial, 2016 harvest, will have a number of species potentual to copice bos again if needed.

Woodvile, 2016/2017 harvest will have Bosistoana, argophloia and possably globoidea.
May or may not be the same families as the verious drylands trials.

NOTE family means same mother, not same father.
If collected at different times even from the same tree varability exists due
to posibility of different set of fathers. Also some selfpropogate, but we dont
know which ones or what proportion, so ignore this.

Proginy trials are alpha latauses, harewood is a standard randomised indervidual trial.

Contact Ruth McConnachie: rgcmcconnochie@xtra.co.nz for DFI details.

slit tests
Pairing Test and Longitudinal Growth Strain: Establishing the Association 2008
is the earlyest paper I can find on the split/paring test.

surface tests

Potentually use NIR
http://www.afs-journal.org/articles/forest/pdf/2002/05/05.pdf
Has some usefull info on wave lengths assocated with bonds ic cellulose

Non-destructive evaluation of surface longitudinal growth strain
on Sugi (Cryptomeria japonica) green logs using near-infraredspectroscopy

statistics

\section{background of chem work}

probably need to talk to clemens about his ideas on testing lignin swelling

how could we test cellulose contraction? -- been done to some extent, maybe copy there method using cells of verious stages of development

what has been done in the past?

somthing with NIR, MRI, PET or some other imaging scaning

\subsubsection{chem in this thesis}

Do all of the DFI species have a G-layer?
Maybe include some Nitens tests if they do.
check MFA and SD for S\_2 in tesion, normal and compression/opersite wood
Get cellulose lignin and hemicellulose(s) contests for testion normal and compression/opersite wood
    Split hemicelluloses where posable, eg xyloglucan etc.
Tortion tests on individual cells, again for tesion, normal and compression. Maybe remove G-Layer in tesion wood and compare to normal and compression wood of similar MFA and compounds etc.

Could we somehow measure growth stress release on a single cell?
Ideally, grow disordered cells invitro, and seperate them from the perant cell as soon as posable, then record when in their formation they undergo what dimention changes. Is there some non-destructive test to check what is going on in the cell? or if we have multiple cells in the same conditions maybe we can destructivly test some during the growth phase, under the assumption they are all growing at the same time.
OR
remove the cambrail layer leaving top and bottom of cell attached to the stem on a large sameple, then somehow remove the connection to the cells behind it, then release the top and measure the contraction.


\section{background of modeling}

yammamotos most resent attempt

posable different methods > FEM, DEM, molecular dynamics, gemomentry of stem and cells

\subsubsection{modeling in this thesis}
cells as particals in relasxed state

apply body force, ie the growth stress feild

get origonal/non cut stick back

take take groups of represtive cells and use composite theory and position dependnet body force (growth strain feild) from teh subdomain above

introduce time depedence to fues at how the stress field develops during maturation, composite scale still

take individual calls at macromolecular level and try to prodiuce stress field above during a time dependent maturation function

Molecular dynamics simulations to work out the molecular mechanisums developing the growth stresses

Using the MD sims paramertorize a cell model

Using the cell model develop a time dependent field function

from the feild function create representative cell blocks

put the cell blocks togeather into a stick

cut the stick > do we get out what we put it?

\section{intensions}
to improve breeding stock for NZ dryland forestry with respect to eucs being used for structural timber

to increase understanding of growth stress formation particually in eucallyapts by chemical analysis and computer modeling

\section{objectives}

to create a mathematical model and computer simulation of a piece of cambrium forming growth

stresses at the macromollecular level

to investigate the chemical causes of GSs by chemical annalysis >> how?

to improve breeding stock for eucs wrt growth stresses from feild and lab testing to select aproprate famielies.

\end{document}

