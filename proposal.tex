\documentclass{article}

\usepackage{graphicx}
\usepackage[space]{grffile}
\usepackage{latexsym}
\usepackage{amsfonts,amsmath,amssymb}
\usepackage{url}
\usepackage[utf8]{inputenc}
\usepackage{fancyref}
\usepackage{hyperref}
\hypersetup{colorlinks=false,pdfborder={0 0 0},}
\usepackage{textcomp}
\usepackage{longtable}
%\usepackage{multirow,booktabs}



\begin{document}

\title{proposal}

\author{Nicholas Davies\\ University of Canterbury }

\date{\today}

\bibliographystyle{plain}

\maketitle

\section{Introduction to wood structure and formation}
As trees grow they produce wood in order to become taller and wider. Becoming
taller and increasing canopy size is an effective way to out compete the other
trees and plants for light. With increasing height and width comes increasing
weight, wind drag and internal pressures (for water transport), which requires
either enough redundant strength in the existing structure (such as young monocotyledons) or
for the tree to strengthen its structure as it increases its size. In
dicotyledons and gymnosperms this occurs in two ways, apical and cambial growth
on branches, roots and the stem(s).

Softwoods have a simpler micro structure than hardwoods, consisting mainly of
axially elongated pointed cells named tracheids which serve as both mechanical
support structures and water conduits. Although varying with species, softwoods
may also contain radially orientated tracheids, radially or axially orientated
parenchyma cells and other cell types. Tracheids are the dominant form of cells
within the stems and branches.

Hardwoods contain a more complex micro structure with a number
of different cell types. Fibres provide structural support as their primary
function, while similar to softwood tracheids they differ in some key aspects,
being shorter in the longitudinal direction, more rounded in the transverse
outline, tend to have smaller lumens and have little role in sap ascent. However
the ends do taper to points as in softwood tracheids. Libriform fibres tend to
be longer than fibre trachiads, have thicker walls and are solely for support.
Fibre trachiads function in both conduction and support, as in softwoods,
however their appearance in wood with vessels suggests that they function
primarily for support, and perhaps are an intermediate evolutionary feature
between the softwood trachiad and the libriform fibre. Septate fibres devide
their cell lumens into chambers without crossing the primary cell wall. Septate
fibres are produced in the late stages of division just prior to the death of
the cytoplasm, and appear to resemble axial parenchyma cells, and have been
hypothesised to store starches, oils and resins.

Vessels are the main conduits for sap ascent. Vessels are comprised of multiple
vessel elements being joined at the ends to form long conduits, which can extend short
distances (often less than 200mm) or can be as long as the height of the tree.
These elements are connected through pores or perforations in perforation
plates at the end walls of the cells. The arrangement of vessels into groups is
species dependent and usually described as ring porous (the vessels congregate in early wood)
or diffuse porous (vessels are distributed throughout both early and late wood).

Rays are formed from radially orientated cells often tracheids or parenchyma.
Hardwoods typically contain multisteriate parenchyma rays, but there are a
number of species with unisteriate or a combination of ray sizes, comparatively softwoods rearly
contain multisteriate rays. Parenchyma ray cells are living within sap wood,
however during the transition to heartwood die and are used for storage of
extractives. Rays also provide a mechanical advantage by diverting the
axial force flow reducing buckling and shear stresses between fibres.

Further cell types also exist, such as vasicentric tracheids which have profuse
side wall pitting exhibiting deformation from the expansion of the surrounding
vessels. Axial parenchyma cells are generally abundant and tend to exist in
vertical files and are expected to play a role in the development of heartwood.
More detailed wood anatomy and has little bearing on this project and is
discussed in a number of wood anatomy texts.

In order to reorentat stems and branches of (most) trees produce reaction wood
which provides a force in order to reorentat the tissue. Typicly this
reorentation is toward the light or upwards as is defiend by the negative
gravitrposim hypothesises. Other reasons for reorentation such as reduceing wind
drag have also be sugested. In softwoods this reoretion is caused by the
production of compression wood. Compression wood forms on the out side of the
stem or branch and (expands? so that it is under compression? causeing a
restoring force). Hardwoods on the other hand produce tension wood on the inside
of the desired curve which (contracts?) resulting in a curve forming.
Tradtionly the galaterness layer (G-layer), a layer primerally consisting of low
angle cellulose fibrils on the inside of the fibre tracheids, is credited with forming growth
stresses within the tension wood. However some hardwoods produce tension wood
without producing a G-layer such as E Nitens.

Primarily, at different resolutions this work focuses on the fibre tracheids as
they are the structual cells expected to be responcable for growth stresses in
normal and reaction wood within hardwoods. The fibre tracheids consist of a
number of cell wall layers depending on the species, the particualar cell and
its primary function. Normal wood fibres within Eucalyptus species (CHECK THIS)
consist of a middle laminar (connecting the fibre to the sounding cells) a
primary cell wall and a secondary cell wall consisting of S1, S2 and S3 layers
(produced in coronalogical order so the exact composition will change depending
on the cells developmental stage). The S2 layer is the largest layer and
consists of cellulose macrofibrils wrapped helically around the cells
longitudinal axis. This cellulose is contained within a matrix of hemicelluloses (examples)
and lignin giving the cell wall properties of a fibre reninforced matrix. --how
does this provide structure--

In order for the living cambrial cells to produce wood, each cell must go
through devision from its perant cell, growth and death. Because the
cambriam (and apical merastem) are continually deviding it alows for the tree to be a dynamic
structure changing its form to become better adapted to its current environmental
setting even though large portions (ie the wood) are dead. The transition form
division through elongation and development to death is expected to play a role
in the development of growth stresses within the stem.

\subsection{Cell division, formation, elongation and death}
Dicotyledons and gymnosperms grow in two main ways, upward apical growth and
outward cambial growth. Note monocotyledons (for example palms) do not produce
secondary growth and instead diameter forms as part of primary growth.

As the cambium is forming, fusiform and ray initials are created from the
aplical shoot cells. Fusiform initials are short radially and tangetialy with
tapered ends. From the cambial initials, cells to the inside create the vertical
elements of xylem (tracheids, vessels, fibers, parenchyma, etc.), while cells outside become phloem.
Ray initials produce horizontal elements (rays).

Cambial cells divide in two ways, periclainal and anticlinal.
Periclainal cell division occurs to the inner and outer of the cambial layers.
As the cell division to the inside occurs the volume of secondary xylem that is being
formed increases the tangential stress on vascular cambium resulting in an
extention of the cambial circumference. Although over time many plants show an
increase in the longitudinal and tangential dimensions of the cambial initials it is
likely that this expansion is mainly facilitated by anticlinal division followed
by the expansion of the daughter cells next to the pedant.

During primary wall formation rapid elongation occurs. When the cells devdie
from their perants they remain fixed to their nabiours via the middle lamina.
The intenal hydrostatic (turgor) pressure causes cell expansion. The osmotic flow of
water from the outside the cell to the inside (due to a lower solute concentration
outside the cell than in) which is constrained by the primary cell wall, the
primary cell wall becomes under increasing tension as more water flows into
the cell. Because the centre of the cell has restricted movement, in order for
elongation (to disipate the increasing tensile forces generated from the
inflow of water) to occour the cell turns the biosythesis of cell wall
constituants to produce tip growth. Growth at the tips of the cells allows for
the cells to remain a cosntnat thickness, so no streching is needed during the
elongation phase, as has been sugested previously. The expantion of the cells is
suspected to be controld via modulation of the primary cell wall rather than via
turgor pressure. -- note that primary wall has randomly orentated MFs embeded
in hemicellulose and pectic compounds and becomes lignified after S layer added
, ML is non lignifed, note often compound middle laminer is used to describe
the ML and P at once as are hard to distinguish--- Once the cell has reached its
full size biosynthises of the S1 starts.
Typicly the S1 layer is thin and comprises of very high angle microfibrials,
within the layer many laminates are found. Within each laminate the MFs are
closly aligned, however between each laminate they can (but do not nessassery)
differ greatly, or even reverse the direction of the helix the MFs form around
the cell, although lower right to upper left orentation tends to be favered.
Close to the S2 layer the MFA decreases repidly.
The S2 layer bound to the inside of S1 is typicly much thicker and has more
verticly orentated micorfibrils compeared to the primary, S1 and S3 layers, these MFs circle the
cell axis from lower left to upper right. S2 contains the majorty of the lignin
within the cell. In some cases, most commonly in late wood a thin S3 layer is
also produced with high MFA, reversing the direction of the MF helixs to lower
right to upper left.

Finally if tension wood is being produced a Gelatonus layer
may be produed on the inside of the inner most wall (S2 or S3). The G-layer has
near verticly orentated microfibrils and very little lignification. It is suspected that the
G-layer plays an important role in the generation of reorentation stresses.

At some point during the formation of the seconday cell wall, or soon after the
cell shrings verticly and expands tangentially. Because of the connectivness
between cells this results in growth stresses forming within the stem, this
phenomonan is descussed in greater detail in ----. After the seconday wall
formation cell `death` occours as part of the transition from sap wood into
heartwood. While the hollow, dead cells play an importnat role in water
transport and mechanical support of the tree, over time any residual nutrant
that can be used by living cells--- heatwood stuff----

What is the deal with Rays----

\subsection{Cells and wood in the context of a whole tree}
Wood as a materal within the tree has three major functions to achieve; water
transport, nutrant transport and mechanical struture. Softwoods achieve
water transport and mechancial struture within trachieds, while parenchima cells
are used for nutrant transport. Hardwoods have evolved a more complicated
internal structure of vessels and fibre tracheids in order to separates out the
functions of water transport and mechanical support respectively.

--advantages and disavantages of this--

The growth stresses that form as part of cell formation are throught to provide
a superiour mechanical structure. Because of the continual formation of new
cells providing growth stresses on the periphery of the stem the older wood
which has completed its formation and cell death must be contracted further with each
new layer of cells attempting to contract. The result of this is the older wood
near the centre of the stem becomes compressed while the newer cells can not
contract to the extent that would leave them in their lowest energy state
remain in tension, until the bond between the old wood and new is separated
releasing the forces restricting this contraction (and extention in the centre)

Reaction wood as described above provides the ability for the stem to reorentate
in order to be best adapted to its environment at any given time.

These properties of wood allow for an adaptive organism to survive..

\section{History of work on growth stresses}

It is suspected that growth stresses develop within trachieds during the
formation the secondary cell wall, although the exact timing and mechanism for
developing growth stresses is still of much debate. The most current ttheoryis
a hybrid of the older cellulose contraction and lignin swelling hypothesis.

A breif discussion of work relating to growth stresses prior to 1965 is given
below, Archer (growth stresses book intro) provides a full review of suggested theories
up until 1965.

Wood workers have unintentionally known of growth stresses within trees
for centries. Usually refered to as `a pull towards the sap` when cuting boards good
craftsmen would section the log in such a way as to get a stright board once it
is removed from the log (and the growth stresses released). Most work early on in
the study of growth stresses surrounded investigating how/why boards changed
shape when cut from an intact stem.

Martley (1928) was possibly the first to study growth stresses in a scientific
manner. Initally he argued that the curvature of planks sawn from logs was due
to the current growth not being able to support the dead weight of the tree until
lignification was complete. As a result the centre is under compression while
the periphery had zero stress. However calculations showed that the self weight
was not sufficient to cause the observed longitudinal dimension changes of the
timber.

After Martley's work a small number of authors investigated growth stresses
through the 30's and 40's. Jacobs, although testing 34 hardwood species, focused
mainly on Eucalyptus and in 1938 argued that (longtudinal) tension successively
develops in the outer layers of the stem as it grows, and as a consequence of of
the tension, compression must form in the centre of the stem. Jacobs later used
E. gigantea to descibre a strain gradient developing during growth.
Experementally Jacobs made use of strip planking, measuring the deflection of
the board after removal from the log, and the length change when the planks were
foced back straight. He showed that wood tends to shrink in the longitudinal
direction at the periphery while extend near the pith (indicating in the log
the planks are under compression in the centre and tension at the extremities).

Further Jacobs put foward a number of hypothesis to explane how the growth
stresses were forming. First arguing that it is very unlikly that dead cells
(wood) could extend within the core in order to create the observed stress
gradiant. Instead sugesting the causes of; weight of the tree, surface tension
and sap stream forces, cellulose and colloidal complexes, lignin intercellular
substances and the primary or secondary cell wall. Although without any evidence
did not claim any of these to be the major cause.

Stresses relating to reaction wood received more attention through the 30s and
40s for both soft and hardwoods. Jacobs 1945 stated that the reorientation of
stems is caused by a modification to the already existing stress gradient
throughout the stem. One option he presented was simply that the eccentric
growth causes larger number of cell sheves to be added to the upper side of the
curve each providing the same amount of contraction force, this results in a
angle correction even with identical cells. Sap tension is also considered, but
more importantly Jacobs notes the posability of tensions being formed within the
cell walls of tension wood.
Munch 1938 specualted that the addition of matter into the cell wall could cause
compression wood. ..
Jacobs 1945 also found that it was commanly the case that the amount of
compression wood developed and the stem angle recovery had a poor relationship.
He sugested maybe it was infact the normal strain pattern in tension which
correct the lean, the compression wood mearly acted as a pivot, not contributing
a tensile force on the lower side of the stem.

Boyed 1950 Developed a new expemental techneque in order to investigate the
stress profile further. By cutting a slit longitudinaly in the centre of the
log, attaching strain gauges onto the wood inside the slit and sucseivly
shortening the log from both ends he obtained direct extention measurments from
inside the stem. --found that the crossover point is is about 1/3 rad of the log
from the perifphery--

Most commanly growth stresses were investigated from the longitudinal direction,
however cells also change dimention in the transverse direction, this leads to a
more complicated three dimetional stress feild developing within even a straight
stem.
koehler 1933 showed that a saw cut radially through a disk has a tendancy to
close near the perifery sugesting that the periferal cells are under tangential
compression with the inner cells under radial tension. He sugested this was the
cause of shakes in standing timber.
jacobs 1945 removed inner circals from disks of a number of speceis and found
when an inner portion is removed the disks cercunfrance incresses. Jacobs again
argued that strain in the sap stream along with cells being wider tangentally
than radially led to the observed lateral stresses. Although he also mentions
the posability of secondary thickening from the deposition of lignin as a
posabe contributing factor.
boyed 1950a developed and experement whereby he removed a wedge from a disk and
meaured the radial expantion, showing the disks were infact under radial
tension. Further aditinal species were found to be in agreement with the results
of jacobs 1945 when the inner circils were removed from disks. Boyd also shows
that the longitudinal stresses maifesting as transverse stresses via poisson
ratios are only aproximatly one tenth that of the measured stresses.

Boyd 1950c provides an indepth rebutel of the available theries at the time,
arriving at the conclusion the the cell wall development must control the shape
change which results in growth stresses. Further he postulates that cellulose
is primarily responsible with lignin and carbohydrates also playing important
rolls when stresses are formed in normal, compression and tesnion wood.

wardrop 1965 commented that a tensile stress generated in the cellulose
transitioning into a crystaline state could be the explination for cells
contracting during the formation of the secondary wall. Cellulose contraction
alighned well with the observation of the G-layer (which has  a very low
MFA) being comman in a number of tension wood producing species, and also gave
the ability for low MFA normal wood to contract. Bamber 1978 further argued
cellulose contraction claiming turgor pressure in normal wood cells remained
high enough that the cells did not contract before the lignin was deposited,
once/during lignin deposition the cellulose became crystaline and shrunk,
causing the cell to become shorter, the mechanisum for tension wood is
essentually the same. Compression wood on the other had was explaned by the
cellulose being layed down and then the turgor pressure decreasing, causing the
cell to contract before lignin was deposited. In turn the cellulose was under
compression, resulting in the tendency for the compression wood cells to
expand.

Boyd (1972) presented (or rather poplerised) the alternative (more widly
accsepted) hypothesis of lignin swelling (first conceved by Munch 1938). Tensile
stress is gained in cells of low MFA by lignin deposition into the cell wall,
pushing the cellulose fibrils appart, which in tern shrinks the longitudinal
length of the cell and incresses the tangential width. When MFA is high, the
opposite occurs, lengthening the cell and reducing its tangental width. This
shape change is not readily apparent in compression wood (characterised as short fat
trachaids) until the release of the stress acting on the CW, where by the cells
become longer and skinnier.

Around the same time two other lesser known hypothesis were presented,  strains
due to changs in water content Hejnowicz 1967, argued that the stresses in
compression wood are related to the inhibition of water by the cell walls,
which results in swelling, because the expansion of compression wood is equal to
the shrinkage due to drying. --paper disproving this--

brodzki 1972 hypothesised strains due to 1,3-linked glucan (laricinan)
deposition within the helical checks of the S2 cell wall layer could be the
most significant factor in longitudinal growth stress generation. Boyd 1978
refuted this idea arguing (along with other issues) that the laricinan would
expand into the cell luman not casuing any stresses in the cell wall, unless a
(non-observed) constraining meadian restricted the expansion.

---- Gills 1973

Through the late 70's and 80's archer produced a number of papers in two series,
`on the distribution of growth stresses' --refs-- mainly concerning the
mathamtical treatment of the stress feilds within trees. --- and `on the
orign of grwoth streses' ---refs--- primarally concerned with the underlying
mechanisums generating growth stresses.

The `on the dirstrobition of growth stresses' series presented a comprehensive
mathematical framework for the treatment of the stress feild within living
trees. Advancing on Kublers work Archer introduced orthotropic solution which
allowed for each new growth increment to alter the stress distribution within
the stem in a self equlibrating fashion. The other advancement made was the
increased acuracy from the crossover point from compression to tension now being
goverened by the moduli in both the radial and circumferential directions.
Archer went on to develop a numerical approximation to the stress fields
generated by asymmetric growth strains and inclind grains, allowing for
variation within growth stresses. Finely he used the developed methods to
present solutions for a number of hardwood species.

`on the origin of growth stresses' Archer investigated the mechanisms behind
growth stress generation.

More recently theories regarding the nature of hemicelluloses and their bonding
have been used in an atempt to remove some of the issues associated with the
cellulose contraction hypothesis. One major issue of callulose contraction is
that in its initial form it was argued that the crystallisation process of cellulose
shortend its length. --ref-- showed that when cellulose crystallised it became
longer as the chains increased order. Two theories have been advanced to combat
the issue of lengthinging during crystallisation in order to retain an updated
version of the cellulose contraction hypothesis.

--- argues that at the edge of the cellulose fibrils the cellulose becomes
dissordered and is concequently able to bond with hemicelluloses, which have a
slightly shorter repeate length than the cellulose crystel. These hemicelluloses
bondend to the outside of the fibril cause the fibil to be compressed in the
cystaline centre, while under tension on the surface. An intersting concequnce
is the contraction of the cellulose due to the hemicellulose bonding should be
dependent on the area/volume to circunfrance/suface area ratio. A potentual way
to test this hypothesis is duscussed in section ---

The seccond theory put foward in an attempt to correct the issues souronding
cellulose lengthenging during crystalisation is from --- who argues that
hemicelluloses form within the fibrils and push them appart causeing the
cellulose fibrils to contract. Interestingly mechanicly this is very similar to
the lignin swelling hypothosis. By in causing the MFs to no longer run stright,
instead they have to use some of their length to devate passed a culster of
hemicelluloses concequently shortening the over all distance the fibril can
cover. One side effect of having these devations is fibrils should not have a
consistant cross sectional area over their whole length, where the
hemicelluloses have been deposited should result in an increased cross section.
potential way to test this?--

The most resent attempts made to describe the formation of growth stresses have
been made by yamamoto and his team. They argue a combination of both lignin
swelling and cellulose contraction is needed, called the unified hypothesis. It
should be noted that --- and others sugested that both theories were likely to
contribute to growth stress generation. By unifing the hypothesises they follow
the current thinking of a number of authors in other areas of wood science that
both tension and compression wood are not distictly different types of wood and
are instead extreme versions of normal wood. ---more from yammamoto---


--- lots more in here from 80s-now ---
Note Muller et al 2006 found low hemicellulose content in G-layer
timell 1969 higher conc of lignin in s2 layer when G-fibres present

The generation of longitudinal maturation stress in wood is not dependent on
diurnal changes in diameter of trunk --- new info on water pressure hyp

There are currently a number of outstanding issues associated with all
of the current hypotheses/theories. When and how do the stresses get
generated is still of much debate, over the last couple of decades it has become
fairly widely accepted that the generation of the stresses occurs during or
immediately after the deposition of the secondary cell wall. Most commonly either
the G-Layer or the S2 layer are considered responcable. What the mechanism(s)
is within the cell wall has been hypothesised about at great length (as
discussed above), however no theory presented so far is without country
experimental evidence.

Another outstanding issue, common to many biological problems is why do
particular trates vary so much between individual and species? One of the
more debated topics around growth stress generation is whether the generation
mechanisms for stress in reaction wood are extreme versions of the same
mechanisms in normal wood. The G-layer is not found in normal wood, however not
all tension wood producing species produce G-layers. Lignin swelling could
potentially fit this criteria for normal and compression wood, however
modification of Boyds theory would be needed address the dependence of a MFA
as some wood with lower than 40 degree MFA still produces compressive forces,
and there has been reported to be little lignin within the G-layer, which is
suspected to be responcable or at lest partly responcable for tension
generation. Boyds theory combined with excessive mild compression wood
formation in core wood still allows for the same tensile generation mechanisms
to be used by older cambriams, as long as the MFA is suited to the task.

It is fairly well accepted (although almost by default) that growth stresses
exist because they provide a mechanical advantage for survival. However to
quantify the mechanical advantage with so much variability between inderviduals,
and no known way of controlling growth stress generation this is very difficult.

Growth stresses studies have been largely confined to model, or common species
however there are a number of species which appear to form intermediates or
`strange` forms of reaction wood. For example Hebe is a angiosperm which appears
to form compression wood rather than tension wood.

\subsection{Why growth stresses exist}
Hardwoods typical have much larger growth stress magnitudes than softwoods.
--why-- is this true 'xylem cell development'?--- Some young conifers have been
reported to have larger compressive stress at the stem than at the pith,
this may be attributed to the abundance of compression wood in juvenile
conifers observed by some. Once older they follow the same radial stress profile
as hardwoods.

The commonly accepted argument for the reason of growth stresses existence is
the mechanical hypothesis. The mechanical hypothesis argues that a number of wood
properties, including the development of growth stresses evolved in order to
provide increase mechanical stability of trees in order to increase their
survival. The mechanical hypothesis as applied to growth stresses argues that
because wood is stronger in tension than compression by preloading the outer
edge of the stem in tension it increases the non-destructive bending radius on
the inside of the curve when a force is applied causing the stem to bend.
The growth stresses produced by reaction wood allow for the tree to correct its
center of gravity, orientate in such a way as to minimise external loads such as
wind and position it's self for optimum light interception. All of these
increase competitiveness.

---speculation from various authors
Typically when attempting to determine the reasons for why wood properties exist
one of four hypothesis are used; mechanical, hydraulic, time dependent and a
combination of the previous three. Initial speculation as the the reason for
growth stresses existence came from Martley (1928) who briefly entertained the
mechanical hypothesis based on self weight. Jacobs (1945) suggested they were a
byproduct of sap tension, which he later retracted Jacobs (196?) when sap
pressures were recalculated at a much lower value than the generally believed
values at the time. .. Growth stresses indubitable have an effect on the
mechanical stability of trees, although it is conceivable that the effect may be
byproduct of another driver.


\subsection{Issues growth stresses cause }

At harvesting growth stresses are released by the saw cut (and crosscutting etc)
and can ruin structural and veneer logs due to the resulting splitting and
warping. Growth stresses, particularly reaction growth stresses increase the
danger for the faller by effects such as saws binding and `barber chairing`.

When the stem is felled or cross cut, growth stresses are released around the
saw cuts causing shortening at the periphery and extension in the centre. The
dimension change is maximum at the saw cut, reducing as distance from the cut
increases. When the contraction/extension force exceeds the plastic limit of the
stem splitting occurs. The cracks tend to propagate in the radial direction by
cell wall pealing, although the cracks tend to only be a few centimeters deep
they significantly reduce the value recovery for both structural and peeler logs.

Prolonged compression at the centre of the stem during growth can exceed the
elastic limit of the wood, resulting in internal defects such as brittle heart.
When the stem is felled these defects have already occurred and hence there is
no way to prevent them during felling, however selection for low growth stress
producing families should significantly reduce the occurrence of internal
defects.

Within mills during processing growth stresses cause a number of issues leading
to reductions in value recovery. Because growth stresses are released when the
stem is sectioned via sawing (plain, quarter etc.) the resulting shape change
can cause the saws to jam. The main value loss at this stage of processing comes
from the need to saw boards multiple times in order to release the stresses
while still allowing for the final board dimensions to be retrieved.
Increasing the number of times the boards are sawn to get their end dimensions
gives not only poor saw use efficiency but the major economic loss comes from the
final yield being as low as 30%.


\section{Proposed theoretical and experimental work}

\subsection{A proposed modification to the lignin swelling hypotheses}
The lignin swelling hypotheses (Boyd 1950) argues the deposition of lignin into
the secondary cell wall forces the cellulose fibrils away from each other, because
cellulose is very stiff when it bows because of the lignin pushing the fibrils
apart the cell changes shape based on the MFA. One of the main arguments for the
use of the cellulose contraction hypothesis is that the G-layer is mainly
crystalline cellulose and hence is not effected by lignin swelling. However if
the outer of the cell is constrained transversely by high MFA fibrils (as in the
P and S1 layers) and surrounding cells, when lignin is deposited in the S1, S2
and/or S3 layers causing swelling, cell expansion will occur toward the cell
lumen (as long as the MFA is conducive to transverse swelling). If the G-layer
is already constructed when this swelling into the lumen takes place it will
cause a bowing of the cellulose fibrils within the G-layer and consequently
contraction of the cell.

To create the maximum amount of contraction from the G-layer there will be an
optimum MFA within the secondary wall (most commonly the S2 layer) dependent on cell
geometries, around 40 degrees. Further because the G-layer is contracting the
cell, it requires the rest of the cell to be as flexible as possible (without
compromising the cells other properties). Flexible cells commonly exhibit higher
MFAs than stiff cells within the S2 layer. By having a non-stiff structure the
G-layer can cause more contraction on the individual and surrounding cells while
under less bowing from lignin swelling. Therefore the optimum MFA of the secondary
wall (excluding the G-layer) will provide the G-layer with the maximum ability
to contract the cell when MFA is in the mid range, i.e. non-stiff and maximum
transverse swelling for minimal longitudinal dimension change.

The qualitative basis of the lignin swelling hypothesis in its current form
remains unmodified and can still account for normal and compression wood growth
stresses.



\subsection{Theoretical work}
Over the years there have been a number of attempts to mathematically model
cells (usually fibres or tracheids) from cell wall constituents (Mark
1967, Koponen et al. 1989, Harrington et al. 1998, Yamamoto and Kojima 2002,
Kojima and Yamamoto 2004 are a few example) however very few efforts have used
these techniques to investigate the formation of growth stresses (archer 1987,
Yamamoto 1998, Guitard et al. 1999).

Currently the most advanced model for how growth stresses develop within the
cell wall was presented by Almeras et. al. (2005) using the unified hypothesis
(Okuyama et. al. 1986, Okuyama et. al. 1994, Yammamoto et. al. 1991, Yammamoto
et. al. 1992 and Yammamoto 1998) utilising both the lignin swelling and
cellulose contraction hypotheses. For details see section ---.

Proposed model of the cell:
Modelling of a generic single cell with variable cell wall properties to
investigate the required geometry and constituents to create maximum
longitudinal and tangential extension and contraction via the lignin swelling
hypothesis. The single cell model should have the capacity to put limits on what
stress generation the lignin swelling hypothesis is theoretically capable of.

Because the proposed experiments (see section --) induce tension wood in species
both with and without G-layers an experimental upper limit of the lignin
swelling hypothesis should be reached and compared to the theoretical one
derived above. By including the G-layer (assuming the experimental work shows
the G-layer is a contributing factor to the production of growth stresses in
tension wood) within the model (by implementing the hypothesis above), and
comparing the required chemical make up and cell geometries with the
tension wood experimentally investigated, light should be shed on the likelihood
of the lignin swelling hypothesis being extendable to include G-layer type
tension wood.

It is expected that the base model and parameters will be similar to
those utilised to describe lignin swelling by Almeras  et. al. (2005) and
Yammamoto (1998). Cell wall layer radii, thickness, S2 layer MFA, moduli of
the CMF bundles and matrix will all be included. Additional variables will
be included as necessary. It is intended to add the standard deviation of the
MFA within the S2 layer, as in Harrington (1998), MFA for the G-layer when
present will also be included with S2 interaction. Because the
maturation process is time dependent, time dependent functions will be
considered allowing for partial and concurrent cellular development. Boundary
conditions will be initially derived from those presented by Almeras et. al.
(2005) and further modified for increased realism and/or usability of later
models.

Time permitting if experiments find that the lignin swelling models don't account
for the strains present a proviso for cellulose contraction could be included.
Between the experimental results and the cellular models, calculations as to how
much contraction is needed should be possible, giving an initial starting point
to look more in depth into cellulose contraction mechanisms.

Proposed model of stem:
Because of the nature of the experimental work it is required to be undertaken
at a macroscopic scale, while the proposed theoretical model is at a cellular
(nano to micro meter) scale. The scale difference between the two methods
causes an issue in that they are not directly comparable (as a sample of wood is not
homogeneous). In order to overcome the scale dependency it is proposed a second
theoretical model be produced which will operate at a macroscopic scale with the
purpose of simulating the experiments undertaken. By parametrised with the
single cell model (which has been parametrised with the experimentally derived
cell anatomy and geometry) approximations to the actual sample being tested
should be able to be made and compared to the experimental outcomes. This
proofing is required to make sure that the results of the single cell model is
providing are realistic.

\subsection{Experimental work}

Currently neither lignin swelling or cellulose contraction have any direct
experimental evidence. The tension which cellulose is under on the stem
periphery has been directly measured using x-ray diffraction showing the
repeat length of 1.035nm in tension wood decreasing to 1.033nm when the stress
is released (Clair 2006).

Experimental evidence of the G-layer providing contraction within tension wood
has been presented by Goswami (2008). Longitudinal extension and tangential
contraction was observed when the G-layer was enzymatically removed from
tension wood poplar samples. The S2 layer was reported to have a high MFA (36
degrees) as has been reported previously and for other G-layer producing species
--refs--. Goswami (2008) suggested lateral swelling of the G-layer caused the
contraction.

The primary goal of the set of experiments which will be presented within this
chapter (when used to parametrize the mathematical modeling work) is to attempt
to identify which cell wall constituents are controlling stress generation and
how they are controlling stress generation under different conditions. In order
to evaluate stress generation mechanisms a number of experimental techniques
have been identified.

Basic cell wall anatomy and geometry needs to be investigated for the NZDFI
species involved in this project. Where possible literature values will be used
to approximate these values for model parametrization. The following properties
are required, however will only be sort from experimental techniques when it is
deemed there is a significant advantage over available literature values.

The cell wall anatomy under different wood types (tension, normal and opposite)
needs to be investigated for the various NZDFI species (principally E.
bosistoana).
The anatomy study will consist of investigating which species produce a G-layer
(microscopy with staining) and what cell wall structure is associated with its
production (Electron Microscopy). The cellulose, lignin and other constituent
contents will be determined for tension, normal and opposite wood (Wood
nitration combined with NMR studies) along with the MFA and the MFA standard
deviation in all three wood types (x-ray diffraction). Fibre diameter, length
and lumen size will also be obtained (microscopy). Within tension wood the
removal of the G-layer (in G-layer producing species) will be needed in order to
determine the secondary cell wall properties of tension wood (enzymatic removal).

The cell wall constituent study results will be used to make comparisons between
the growth stresses produced by the stems and the different properties within
the cell walls. With the anatomy results collected from tension, normal and
opposite wood comparisons can be made not only within trees but
also between trees. By comparing tension wood with the G-layer removed with
normal wood with similar properties some insight into the role of the G-layer
should be gained. Note that growth stresses for a large number of samples will
be collected during the breeding work, however because of the time consuming
nature of the experimental works presented here only a small number of specimens
will be tested as needed.

In order to produce the three types of wood required two different growth
manipulation techniques are suggested:

Technique one; Young stems (less than three month old growth from coppice) will
be restrained to a loop, similar to Jacobs loops (Jacobs 1945) and allowed to
grow for approximately one year, with regular adjustments of the restraints to make sure
the cambium is not damaged. From the same plants a second leader will
be selected and restrained to a straight pole to provide normal wood of the same
genetics.

Technique two; Straight one year old stems (from coppice, and seedlings of a
mixture of camadilencia, tricarpa and quadrangularta) will be bent and restrained and
allowed to growth for a further 6-12 months, with regular adjusting of the
restraints to avoid cambium damage. Normal wood samples can be collected from
these stems from wood produced before manipulation and away from the bend site.
These plants will be selected from camaldulensis (reported to produce S1-G
tension wood (Baba 1996)), quadrangualata and tricarpa depending on the
suitability of the plants available.

The other set of experiments proposed is to investigate the extent of an effect
the G-layer has on tension generation within tension wood, and how the G-layer
generates these tensions.

Proposed experiment one:
Taking samples with G-layers and applying a vacuum pump to suck an enzyme
treatment into the fibres and vessels to degrade the G-layer releasing the
strain which the G-layer is applying to the samples will cause a shape change.
By comparing the initial and final shape change the strain the G-layer was
imposing on the samples can be obtained. Further the rest of the growth stresses
can be released using more traditional techniques such as split tests or
planking, by releasing the remaining strain the proportion of stress associated
with the G-layer and other cell wall components (assumed to be the S2 layer) can
be determined.

Proposed experiment two:
Release the stress with a split or planking test, then remove the G
layer using the same method as above. When the tension caused by the G-layer is
released relaxation back towards the the initial state should be observed. The
proportion of G-layer induced strain and S2 induced strain will then be
evident.

Proposed experiment three:
During growth induce tension wood production by forcing curvature into the
living stem. By introducing an enzyme treatment to the plant while it is still
transpiring should degrade the G-layer and reverse any straightening that was
caused by the G-layer.

Any of the three proposed experiments, if they work will provide the proportions
of the strain in tension wood which can be attributed to the G-layer.

% Paper claiming camaldulensis has G-layer, note that it is a S1-G cell
% Chemical and anatomical characterization of the tension wood of Eucalyptus
% camaldulensis L. Mokuzai Gakkaishi

\subsection{Breading}
Because growth stresses cause a number of issues for harvesting and milling
timber, tree breading programs can and have be used in order to select for
genetics which reduce these effects. --previous breading for GS-- There is no reason to
expect  breading for growth stresses differs significantly from (conventionally)
breading trees for any other trait, which is process which has been developed
over centuries. Over the last few decades many advances have been made in
experimental and statistical techniques which rapidly improve the time and
accuracy of conventional breading.

It is suspected that the most efficient way to minimise the issues growth
stresses cause during the production of timber is through appropriate genetic
selection. Eucalyptus species, in particular bosistona are showing promise
within the NZDFI trials to produce high value naturally durable structural
timber. In order to see the yield efficiency required to make this
product profitable growth stresses need to be reduced to minimise the
effects discussed in section ---. While within the NZDFI project there are a
number of other concerns for breeders (such as durability, form and growth
rate) growth stresses also need to be considered. Using conventional breading
methods discussed below growth stresses will be minimised within the NZDFI
genetics. Currently a number of trials have been established or will soon be
established, these include:

Permanent sample plots (whole forests) are located throughout New Zealand
(primarily in the Marlborough Nelson area).
These plots are for profit forestry plantations ranging in ages up to 8 years
old.
Because the plots are not specifically research plots limited testing can be
undertaken on the trees. These trials consist of the species ---- set up as
alpha latause trials. Some of the genetic material is duplicated in other
research specific experiments described below.


The Harewood trial:
All trials at Harewood are set out as randomised individual trials.
Principally this work will be concerned with E. bosistoana of which there are
two trials. One with xx replicates of xx families, planted in (when?) and
coppiced in (when), due to be harvested in spring/summer 2015. The other E. bosistoana
trial has xx replicates of xx families, was planted in (when?) and harvested for
the first time in 2012, the plants were then coppiced and harvested again in
December 2014. Four families representing the highest and lowest growth stress
generating genetics were coppiced for a second time and will be due for harvest
in 2016. Preliminary results from the 2012 and 2014 harvests show reasonably high
heritability  of growth strains, particularly within family rankings. The same
data was collected from --- E. argophloia plants planted in --- measured and
coppiced in 2012, with final measurements completed in 2014.  In ground
plantings of bosistana, --- have been plated in 2014 and will continue in 2015
with bosistana. Note most of the plantings required to get the material for the
studies outlined in section --- are also grown at this site.

Woodvile trial:
Due to the success of the previous trials NZDFI is currently setting up a xxx
plant trial with xxx replicates of xxx families of bosistana, xxx replicates of
argophloia and possibly globoidea. The trials are set up as alpha lattices and
harvest is expected in late 2016 or 2017. The intention is to have as much
overlap as possible with the existing genetics within the current and previous
trials.


Note the term family is used here to mean the same mother, but not necessarily
the same father. If seeds are collected at different times even from the same
tree variability exists due to possibly of different set of fathers. Also some
Eucalyptus self propagate, but it is unknown which ones or what proportion of
seats are self propagated within the NZDFI genetic material. Self propagation
is ignored.

Within the structure of the breading program there is the ability to
statistically check within family genetics results from young stems and mature
stems. Growth stress measurements are a good candidate for this type of
selection as the expected biological processes underlying growth stress
generation are age independent (bar the dependence of some influencing factors
associated with the typical radial pattern of density, MFA etc).

Experimental tests for breading:
In order to select for low growth stress producing families experimental tests
need to be undertaken on each plant. The main test to be used to determine the
extent of growth stresses within the breading population is the split test
(also known as the Pairing Test) as described in Chauhan (2010)
The test involves taking a significantly long section of stem and using a saw
cutting along the pith to create a radial split. The diameter of the stem is
taken before testing and the width of the opening measured immediately after
splitting. Once the opening is measured the stem is cut to across the
grain to give two samples (one from each side of the split). Density is
measured by taking measuring the mass (using balances) and volume using the
displacement method on each of the pieces. Acoustics are also taken using
wood-spec to calculate the dynamic modulus, and hence the stress can be derived. Other
properties such as bark thickness are also record for other purposes. Due to
the size of the Woodville trial it may be the case that less tests are carried
out. Decisions on the essential tests will be made near the time of harvesting.
Note throughout testing the samples are kept in a green state.

Statistical methods:
Although the particular statistical techniques best suited to the data set
obtained from the trial results cant be known before the data is collected, it
is expected that typical breading statistical techniques will be
applicable. The two objectives to be achieved for the breading trials are to
estimate genetic parameters, (in particular heritability of growth stresses)
and to predict breading values for families. Mixed model methods are commonly
used within tree and animal breading.

Potential for the use of modern technologies such as portable NIR
to investigate older trees (in the Marlborough and Nelson plots) without
destructively testing them may be of use.
NIR has shown promise in predicting longitudinal surface strain in Sugi green logs.
Non-destructive evaluation of surface longitudinal growth strain
on Sugi (Cryptomeria japonica) green logs using near-infrared-spectroscopy.
Surface tests such as described by --ref-- will be required to either calibrate
an NIR predictive model, or if NIR is unfeasible to acquire surface strain values on
older specimens.

Partially destructive testing may be appropriate on a limited number of older
trees. Core samples will be taken for other works, however at this stage there
are no methods for the measurement of growth stresses (directly or indirectly)
from core samples. Surface strain tests using strain gauges as have been used by
a number of authors --refs-- could be undertaken if necessary.

%Contact Ruth McConnachie: rgcmcconnochie@xtra.co.nz for DFI details.

\section{Intentions}

\section{Objectives}

\section{Costs}

\section{Timeline}

\end{document}