\documentclass{article}

\usepackage{graphicx}
\usepackage[space]{grffile}
\usepackage{latexsym}
\usepackage{amsfonts,amsmath,amssymb}
\usepackage{url}
\usepackage[utf8]{inputenc}
\usepackage{fancyref}
\usepackage{hyperref}
\hypersetup{colorlinks=false,pdfborder={0 0 0},}
\usepackage{textcomp}
\usepackage{longtable}
%\usepackage{multirow,booktabs}
\usepackage{parskip}
\setlength{\parindent}{0pt}
\usepackage{setspace}
 \doublespacing

\begin{document}
\title{Growth stresses in Eucalyptus}

\author{Nicholas Davies\\ University of Canterbury }

\date{\today}

\bibliographystyle{plain}

\maketitle

\section{Summary}
Growth stresses form in most tree species, in Eucalyptus (and many other
species) these internal stresses can be of a considerable magnitude, especially
in reaction wood. When a stem or branch is cut, some of the stress is released
in the form of strain, resulting in a number of timber defects which reduce log
value.

It has been shown previously that these stresses are induced during secondary
cell wall formation, however the mechanism responsible is unknown. Two
hypotheses exist to explain the stress generation; 1) deposition of lignin
between the fibril aggregates causing the cell to expand or contract axially
depending on the orientation of the fibril (MFA), or 2) a mechanism resulting in
a length change of the cellulose itself causes cellular contraction or expansion
depending on the MFA. A combination of the two has also been proposed.

The proposed research will increase our understanding of the mechanisms behind
growth stress formation through the development of a mathematical model taking
into account a realistic cell wall supramolecular architecture to simulate
maturation. The geometric parameters of the cell wall molecular architecture and
chemical composition will be studied using x-ray diffraction, atomic force
microscopy and wet chemistry techniques. Experimentation will also be used to
investigate the role of the gelatinous (G) cell wall layer in the formation of
extreme tension wood.

The New Zealand Dryland Forest Initiative (NZDFI) is currently breeding
\textit{Eucalyptus bosistoana} as a high value timber crop alternative to
\textit{Pinus radiata}. One of the goals of the breeding trials is to select for
minimal growth stresses so that the timber does not lose value when harvesting
and milling due to the release of internal stresses. The proposed research will
select families at young ages which are producing the lowest possible growth
stresses improving the \textit{E. bosistoana} breeding stock.

\section{Introduction}
As trees grow they produce wood in order to become taller, wider or orientate
stems, branches and roots. Becoming taller or reorienting stems and branches can
be an effective way to outcompete the other trees and plants for light --ref--.
With increasing height, width or off axis stems comes increasing gravitational
force, wind drag and internal pressures (for water transport), which requires
either enough redundant strength in the existing structure (such as in
monocotyledons) or for the tree to strengthen its structure as it increases its
size. In dicotyledons and gymnosperms size increase and reorientation occurs in
two ways, apical and cambial growth on branches, roots and the stem(s) --ref--.

\subsection{Wood structure and formation}
Softwoods have a simpler microstructure than hardwoods, consisting mainly of
axially elongated tracheids which provide both mechanical support and water
transport --ref--. Tracheids are the dominant form of cells within the stems and
branches --ref--. Hardwoods contain a more complex micro structure with a number
of different cell types. Fibres, similar to softwood tracheids provide
structural support however it is their primary function, with vessels providing
conduction --ref--.

Vessels are the main conduits for sap ascent --ref--. Vessels are comprised of
multiple vessel elements being joined at the ends to form long conduits, which
can extend short distances (often less than 200mm) or can be as long as the
height of the tree --ref--.

Rays are formed from radially orientated cells often tracheids or parenchyma
--ref--. Rays provide a mechanical advantage by diverting the axial force flow
reducing buckling and shear stresses between fibres --ref--.

Many further cell types and functions exist, but more detailed wood anatomy and
has little bearing on this project and is discussed in a number of wood anatomy
texts --refs--.

--Growth stresses in normal wood--

In order to reorientate stems and branches of (most) trees, reaction wood is
produced which provides a force in order to reorientate the tissue --ref--.
Typically this reorientation is toward the light or upwards as is defined by the
negative gravitropism hypothesises --ref--. Other reasons for reorientation such
as reducing wind drag have also be suggested --ref--. In softwoods this
reorientation is caused by the production of compression wood. Compression wood
forms on the outside of the stem or branch and expands longitudinally. Hardwoods
on the other hand produce tension wood on the inside of the desired curve which
contracts longitudinally resulting in a curve forming. Traditionally the
gelatinous layer (G-layer), a layer primarily consisting of low angle cellulose
fibrils on the inside of the fibres, is credited with forming growth stresses
within the tension wood --ref--. However some hardwoods produce tension wood
without producing a G-layer such as \textit{Eucalyptus Nitens} --ref--.

Primarily this work focuses on fibres as they are the structural cells expected
to be responsible for growth stresses in normal and reaction wood within
hardwoods --ref--. Fibres consist of a number of cell wall layers depending on
the species, the particular cell and its primary function. Normal wood fibres
within Eucalyptus species consist of a middle laminar (connecting the fibre to
the sounding cells) a primary cell wall and a secondary cell wall consisting of
S1, S2 and S3 layers (produced in chronological order so the exact composition
will change depending on the cells developmental stage)--ref--. The S2 layer is
the largest layer and consists of cellulose macrofibrils wrapped helically
around the cells longitudinal axis --ref--. Cellulose is contained within a
matrix of hemicelluloses (examples) and lignins giving the cell wall properties
of a fibre reinforced matrix --ref--.

In order for the living cambial cells to produce wood, each cell must go through
division from its parent cell, growth and death. Because the cambium (and apical
meristem) are continually dividing it allows for the tree to be a dynamic
structure changing its form to become better adapted to its current
environmental setting even though large portions (ie the wood) are dead. The
transition from division through elongation and development to death is expected
to play a role in the development of growth stresses within the stem.

\subsection{Cell division, formation, elongation and death}
Dicotyledons and gymnosperms grow in two main ways, upward apical growth and
outward cambial growth.  As the cambium is forming, fusiform and ray initials
are created from the apical shoot cells --ref--. From the cambial initials,
cells to the inside create the vertical elements of xylem (tracheids, vessels,
fibers, parenchyma, etc.), while cells outside become phloem--ref--. Ray
initials produce horizontal elements (rays)--ref--.

Cambial cells divide in two ways, periclinal and anticlinal. Periclinal cell
division occurs to the inner and outer of the cambial layers. As the cell
division to the inside occurs the volume of secondary xylem that is being formed
increases the tangential stress on vascular cambium resulting in an extension of
the cambial circumference--ref--. --Anticlinal?--

During primary wall formation rapid elongation occurs. When the cells divide
from their parents they remain fixed to their neighbors via the middle lamina
--ref--. The internal hydrostatic (turgor) pressure causes cell expansion
--ref--. The osmotic flow of water from the outside the cell to the inside which
is constrained by the primary cell wall causes increasing tension --ref--.
Because the centre of the cell has restricted movement, in order for elongation
(to dissipate the increasing tensile forces generated from the inflow of water)
to occur the cell turns the biosynthesis of cell wall constituents to produce
tip growth --ref--. Growth at the tips of the cells allows for the cells to
remain a constant thickness, so no stretching is needed during the elongation
phase.

-- note that primary wall has randomly orientated MFs embedded
in hemicellulose and pectic compounds and becomes lignified after S layer added
, ML is non lignified, note often compound middle laminar is used to describe
the ML and P at once as are hard to distinguish--- Once the cell has reached its
full size biosynthesis of the S1 starts.

Typically the S1 layer is thin and comprises of microfibrils winding around the
cell axis a a very high angle (MFA), within the layer many laminates are found.
Within each laminate the MFs are closely aligned, however between each laminate
they can (but do not necessarily) differ greatly, or even reverse the direction
of the helix the MFs form around the cell, although lower right to upper left
orientation tends to be favored --ref--. Close to the S2 layer the MFA decreases
rapidly. The S2 layer bound to the inside of S1 is typically much thicker and
has more vertically oriented microfibrils compared to the primary, S1 and S3
layers, these MFs circle the cell axis from lower left to upper right--ref--. In
some cases, most commonly in late wood a thin S3 layer is also produced with
high MFA, reversing the direction of the MF helices to lower right to upper
left--ref--.

Finally if tension wood is being produced a Gelatinous layer (G-layer)
may be produced on the inside of the innermost wall (S1, S2 or S3)--ref--. The
G-layer has near vertically oriented microfibrils and very little lignification.
It is suspected that the G-layer plays an important role in the generation of
reorientation stresses--ref--.

At some point during the formation of the secondary cell wall, or soon after the
cell shrinks vertically and expands tangentially --ref-- (with the exception of
some young softwoods, where the opposite occurs, it is expected this is due to
the abundance of compression wood present --ref--). Because of the connectedness
between cells growth stresses form within the stem, this phenomenon is discussed
in greater detail in section ----. After the secondary wall formation cell death
occurs as part of the transition from sapwood to heartwood --ref--.

\subsection{Cells and wood in the context of a whole tree}
Wood as a material within the tree has three major functions to achieve; water
transport, nutrient transport and mechanical structure --ref--. Softwoods
achieve water transport and mechanical structure within tracheids, while
parenchyma cells are used for nutrient transport. Hardwoods have evolved a more
complicated internal structure of vessels and fibres in order to separates out
the functions of water transport and mechanical support respectively.

Growth stresses form as part of cell formation and are thought to provide a
superior mechanical structure. The continual formation of new cells contracting
on the periphery of the stem causes the older wood which has completed formation
to contracted further with each new layer of cells. Older wood near the centre
of the stem becomes compressed while the newer cells can not fully contract and
remain in tension, until the bond between the old wood and new is separated
releasing the forces restricting this contraction (and extension in the centre).
Growth stresses in normal wood increase the mechanical stability of the stem,
while in reaction wood they provides the ability for the stem to reorient in
order to be best adapted to its environment at any given time. These properties
of wood allow for an adaptive organism to survive in a changing environment,
however they also cause significant issues and value loss when harvesting and
milling the timber they produce.


\section{History of work on growth stresses}
It is suspected that growth stresses develop within tracheids during the
formation the secondary cell wall, although the exact timing and mechanism for
developing growth stresses is still of much debate. The most current theory is
a hybrid of the older cellulose contraction and lignin swelling hypothesis.

Wood workers have unintentionally known of growth stresses within trees
for centuries. Usually referred to as `a pull towards the sap` when cutting boards good
craftsmen would section the log in such a way as to get a straight board once it
is removed from the log (and the growth stresses released). Most work early on in
the study of growth stresses surrounded investigating how/why boards changed
shape when cut from an intact stem.

Martley (1928) was possibly the first to study growth stresses in a scientific
manner. Initially he argued that the curvature of planks sawed from logs was due
to the current growth not being able to support the dead weight of the tree until
lignification was complete. As a result the centre is under compression while
the periphery had zero stress. However calculations showed that the self weight
was not sufficient to cause the observed longitudinal dimension changes of the
timber.

After Martley's work a small number of authors investigated growth stresses
through the 30's and 40's. Jacobs, although testing 34 hardwood species, focused
mainly on Eucalyptus and in 1938 argued that (longitudinal) tension successively
develops in the outer layers of the stem as it grows, and as a consequence of
the tension, compression must form in the centre of the stem. Jacobs later used
E. gigantea to decipher the strain gradient developing during growth.
Experimentally Jacobs made use of strip planking, measuring the deflection of
the board after removal from the log, and the length change when the planks were
forced back straight. He showed that wood tends to shrink in the longitudinal
direction at the periphery while extend near the pith (indicating in the log
the planks are under compression in the centre and tension at the extremities).

Further Jacobs put forward a number of hypothesis to explain how the growth
stresses were forming. First arguing that it is very unlikely that dead cells
(wood) could extend within the core in order to create the observed stress
gradient. Instead suggesting the causes of; weight of the tree, surface tension
and sap stream forces, cellulose and colloidal complexes, lignin intercellular
substances and the primary or secondary cell wall. Although without any evidence
did not claim any of these to be the major cause.

Stresses relating to reaction wood received more attention through the 30s and
40s for both soft and hardwoods. Jacobs (1945) stated that the reorientation of
stems is caused by a modification to the already existing stress gradient
throughout the stem. One option he presented was simply that the eccentric
growth causes larger number of cell sheaves to be added to the upper side of the
curve each providing the same amount of contraction force, this results in a
angle correction even with identical cells. Sap tension was also considered, but
more importantly Jacobs notes the possibility of tensions being formed within the
cell walls of tension wood.
Munch 1938 speculated that the addition of matter into the cell wall could cause
compression wood. ..
Jacobs 1945 also found that it was commonly the case that the amount of
compression wood developed and the stem angle recovery had a poor relationship.
He suggested maybe it was the normal strain pattern in tension which
correct the lean, the compression wood mealy acted as a pivot, not contributing
a tensile force on the lower side of the stem.

Boyed (1950) developed a new experimental technique in order to investigate the
stress profile further. By cutting a slit longitudinal in the centre of the
log, attaching strain gauges onto the wood inside the slit and successively
shortening the log from both ends he obtained direct extension measurements from
inside the stem. --found that the crossover point is is about 1/3 rad of the log
from the periphery--

Most commonly growth stresses were investigated from the longitudinal direction,
however cells also change dimension in the transverse direction, this leads to a
more complicated three dimensional stress field developing within even a
straight stem.

koehler (1933) showed that a saw cut radially through a disk has a tendency to
close near the periphery suggesting that the peripheral cells are under
tangential compression with the inner cells under radial tension. He suggested this was the
cause of shakes in standing timber.
Jacobs 1945 removed inner circles from disks of a number of species and found
when an inner portion is removed the disks circumference increases. Jacobs again
argued that strain in the sap stream along with cells being wider tangentially
than radially led to the observed lateral stresses. Although he also mentions
the possibility of secondary thickening from the deposition of lignin as a
contributing factor.

Boyed (1950a) developed and experiment whereby he removed a wedge from a disk
and measured the radial expansion, showing the disks were under radial
tension. Further additional species were found to be in agreement with the
results of Jacobs (1945) when the inner circles were removed from disks. Boyd also shows
that the longitudinal stresses manifesting as transverse stresses via Poisson
ratios are only approximately one tenth that of the measured stresses.

The Poisson effect is the revelation that a change of dimension of a material
in one direction will result in a change of dimension in the other directions,
this relationship is characterised by the Poisson ratio (within the elastic
region of deformation). Within growth stress literature there has been some
investigation of this effect as it appears the redistribution of stress through
the Poisson ratio from the longitudinal to tangential direction is not
sufficient to account for the observed tangential strains, which can vary even
for a given longitudinal strain.

Boyd (1950c) provides an in depth rebuttal of the available theories at the
time, arriving at the conclusion the the cell wall development must control the shape
change which results in growth stresses. Further he postulates that cellulose
is primarily responsible with lignin and carbohydrates also playing important
rolls when stresses are formed in normal, compression and tension wood.

Wardrop (1965) commented that a tensile stress generated in the cellulose
transitioning into a crystalline state could be the explanation for cells
contracting during the formation of the secondary wall. Cellulose contraction
aligned well with the observation of the G-layer (which has  a very low
MFA) being common in a number of tension wood producing species, and also gave
the ability for low MFA normal wood to contract. Bamber (1978) further argued
cellulose contraction claiming turgor pressure in normal wood cells remained
high enough that the cells did not contract before the lignin was deposited,
once/during lignin deposition the cellulose became crystalline and shrunk,
causing the cell to become shorter, the mechanism for tension wood is
essentially the same. Compression wood on the other had was explained by the
cellulose being laid down and then the turgor pressure decreasing, causing the
cell to contract before lignin was deposited. In turn the cellulose was under
compression, resulting in the tendency for the compression wood cells to
expand.

Boyd (1972) presented (or rather popularised) the alternative (more widely
accepted) hypothesis of lignin swelling (first conceived by Munch 1938). Tensile
stress is gained in cells of low MFA by lignin deposition into the cell wall,
pushing the cellulose fibrils apart, which in tern shrinks the longitudinal
length of the cell and increases the tangential width. When MFA is high, the
opposite occurs, lengthening the cell and reducing its tangential width. This
shape change is not readily apparent in compression wood (characterised as short fat
tracheids) until the release of the stress acting on the CW, where by the cells
become longer and skinnier.

Around the same time two other lesser known hypothesis were presented, Hejnowicz
1967 and brodzki 1972. Hejnowicz (1967), argued that the stresses in compression
wood are related to the inhibition of water by the cell walls, which results in
swelling, because the expansion of compression wood is equal to the shrinkage
due to drying. --paper disproving this--

Brodzki (1972) hypothesised strains due to 1,3-linked glucan (laricinan)
deposition within the helical checks of the S2 cell wall layer could be the
most significant factor in longitudinal growth stress generation. Boyd (1978)
refuted this idea arguing (along with other issues) that the laricinan would
expand into the cell lumen not causing any stresses in the cell wall, unless a
(non-observed) constraining median restricted the expansion.

-- Gills 1973 --

Through the late 70's and 80's archer produced a number of papers in two series,
`on the distribution of growth stresses' --refs-- mainly concerning the
mathematical treatment of the stress fields within trees. --- and `on the
origin of growth stresses' ---refs--- primarily concerned with the underlying
mechanisms generating growth stresses.

The `on the distribution of growth stresses' series presented a comprehensive
mathematical framework for the treatment of the stress field within living
trees. Advancing on Kublers work Archer introduced orthotropic solution which
allowed for each new growth increment to alter the stress distribution within
the stem in a self equilibrating fashion. The other advancement made was the
increased accuracy from the crossover point from compression to tension now being
governed by the moduli in both the radial and tangential directions.
Archer went on to develop a numerical approximation to the stress fields
generated by asymmetric growth strains and inclined grains, allowing for
variation within growth stresses. Finely he used the developed methods to
present solutions for a number of hardwood species.

Archer followed up his series on growth stress distribution with `on the origin
of growth stresses' where he attempted to mathematically investigate individual
cells, presenting an explicit relationship between strains and growth
increment of the cell wall. The relationship relates MFA and swelling strain,
he argues that these results are consistent with the lignin swelling hypothesis
for compression wood. In tension wood, by increasing the ratio of area of cell
wall to total cell cross sectional area by adding a G layer could theoretically
(with the parameters Archer used) produce a tensile stress of 36MPa.

 --boyd 1977 Basic cause of differentiation of tension and
compression wood--- Boyd (1977) reinvestigated most of the hypothesis available at the time
regarding the reason for the development of reaction wood. He argued that (as
had been previously rejected) the stress hypothesis, claiming that reaction wood
is a response to imposed stresses is the only available hypothesis which fits
the data, and that previous efforts to reject it had resulted from
misinterpreted results. In doing this he provides reasoning to reject the
hypothesis regarding gravitational responses, intrinsic growth direction, auxin
distribution and turgor pressure.

A common argument that is made for the cellulose contraction hypothesis is the
correlation between cellulose content and strain. Higher proportions of
cellulose compared to lignin correlate to tensile strains, while high lignin
content correlates well to compressive strains --refs--. It has been well
reported that compression wood is partly characterised by an increase in
lignin content --ref--, which has been used as an argument for the lignin
swelling hypothesis. Tension wood however is often but not necessarily
correlated with and increased proportion of cellulose. --tests on tension wood
with no G-layer-- Within tension wood of G-layer producing species tensile
strain and whole cell cellulose content correlate well due to the G-layer
having a very low lignin content --ref-- The proportions of cellulose and lignin
within the cell after the G-layer has been removed do not share this correlation
--ref--.

--- timell 1969 higher conc of lignin in s2 layer when G-fibres present

--timell compression wood in geynosperms--


After Bamber 1979 disputed the reliability of Boyd 1972, Boyd 1985 and Bamber
1987 disputed each others analysis's however no new information was presented,
rather a number of issues around interpreting biological data were highlighted.

Kubler 1987 provided an in depth review of the hypothesis, evidence and
experimental methods at the time, much of which has been discussed above. He
presents a table summarising the literature reporting strains for different
species, highlighting the large intra and inter tree variation even within a
single species.

Yamamoto et al. produced a number of papers entitled `Generation Process of
Growth Stresses in Cell walls` --refs-- where both the lignin swelling
hypothesis and the cellulose contraction hypothesis are considered in detail,
including new experimental evidence for each. -- more on these results--- some
must mention the critical MFA (the point of no longitudinal change) ---

After the experimentation of --refs--- finding the critical MFA to be between
25 and 30 degrees for a number of species --refs--, Okuyama et al 1993 (growth
stresses in tree, in Japanese) and yamamoto et al 1995 (series num 6) suggested the
unified hypothesis. Although the idea of both lignin swelling and cellulose
contraction being responsible for growth stress development had been suggested
before --refs-- it was formalised here. In an attempt to solve the critical MFA
discrepancy --yam et al 1995-- augmented the barber and meylan --ref-- cell wall
model to include a S1 layer. The resulting model was the first to be able to
account for production of both tensile and compressive stresses over a wide
range of MFAs, however this was only achievable using unnatural parameter values, (in
particular --what were they--) The S1 layer introduced utilizes a constant MFA
of 90 degrees, with the S2 layer varying from 0 to 60. Cell wall maturation
occurred in two discrete steps, first the cellulose framework is constructed then
the lignin depositions occur. From the model they showed that with an
increasing S1 layer thickness the critical MFA reduces. Unfortunately they found the model
was unable to produce realistic tangential strains unless unnatural parameters
were used.

---maybe include the cell wall model picture in here some where---

Yamamoto 1998 further refined the idea by introducing a much more rigorous frame
work, incorporating time dependence into the cell wall maturation model. The
work presented shows the failings of each lignin swelling and cellulose
contraction, even when time dependence is include. Time dependence does allow
for good agreement between the modeled unified hypothesis and experimental
values from sugi --ref--. The poor agreement with tangential stresses is
explained as being easy to decrease through stress relaxation in comparison to
the longitudinal stress when inside the trunk.

--Guitard 1999 growths tress generation a new mechanical model of the
dimensional change of wood cells during maturation---
Guitard et al 1999 used a S2 layer model which took the transmission of shear
between fibres into account, resulting in non-zero shear moduli. Previously
integral conditions had been used to govern the longitudinal stresses,
presumably as they satisfy the necessary condition implicit within stress field
equilibrium conditions, Guitard et al 1999 however introduced a local condition on
every elementary volume. They argue that although this approach does not satisfy
the necessary equilibrium conditions it provides better agreement with
experimental results when combined with dimensional changes within the
microfibril bundle. In particular this model provides a much better prediction
of transverse strains while being less complex than previous attempts.

Yamamoto et al (origin of the biomechanical properties of wood related to fine
structure of the multi layer cell wall) 2002 further advance his 1998 model to
include drying stresses and moisture depend Young's modulus, however little
changes were made with regard to the growth stress model.

sassus et al 2004 modellsation des deformations de maturation de la fibre

Almeras et al 2005 (modelling anisotropic maturation strains in wood in
relation to fibre boundary conditions microstructure and maturation kinetics)
although similar made some major advancements over the yamamoto 1998 model,
producing what is currently the most advanced mechanical model for growths
stress production available. Previously fibres had been assumed to either be
free (yamamoto 1998, etc --refs--) or fully restrained archer 1987?. Here
various boundary conditions are investigated, the most realistic arrangement being
displacement fully restrained in the longitudinal and tangential direction while
free in the radial. The virtually isolated fibre conditions were simulated here
and found to be in good agreement, although with some small discrepancies from
yamamoto 1998 (due to the introduction of some second order terms).  Their
investigation showed that differing boundary conditions had only a small effect
on the longitudinal strain, however the tangential strain was significantly
effected. This is explained as the cellulose is considered to already be stiff at
the start of maturation, therefore all of the stress within the cellulose can be
released as strain. However in the tangential direction the stiffness of the
fibre progressively increases as maturation proceeds, resulting in the
releasable strain being only a fraction of the total stress. In order to get good
experimental agreement Almeras et al 2005 used a transverse strain release
parameter in order allow some strain to be released during maturation. They
found that 74\% of the transverse stress needs to be released during maturation
to provide the best agreement with experimental data.

Many of the previous models have used a physical interpreatation of the
reinforced matrix hypothesis (barber and meylan) which describes the cell wall
as a two phase strucure of cellulose fibrils and an isotropic hemicellulose and
lignin matrix. However until Yamamoto and almeras 2007 (mori-tanaka) there was
no basis for this asumption. These authors provide a formal basis for validating
these asumptions.

More recently theories regarding the nature of hemicelluloses and their bonding
have been used in an attempt to remove some of the issues associated with the
cellulose contraction hypothesis. One major issue of cellulose contraction is
that in its initial form it was argued that the crystallisation process of cellulose
shortened its length. --ref-- showed that when cellulose crystallised it became
longer as the chains increased order. Two theories have been advanced to combat
the issue of lengthening during crystallisation in order to retain an updated
version of the cellulose contraction hypothesis.

--- argues that at the edge of the cellulose fibrils the cellulose becomes
disordered and is consequently able to bond with hemicelluloses, which have a
slightly shorter repeat length than the cellulose crystal. These hemicelluloses
bonded to the outside of the fibril cause the fibril to be compressed in the
crystalline centre, while under tension on the surface. An interesting consequence
is the contraction of the cellulose due to the hemicellulose bonding should be
dependent on the area/volume to circumference/surface area ratio.

The second theory put forward in an attempt to correct the issues surrounding
cellulose lengthening during crystallisation is from --- who argues that
hemicelluloses form within the fibrils and push them apart causing the
cellulose fibrils to contract. Interestingly mechanically this is very similar
to the lignin swelling hypothesis. By causing the MFs to no longer run straight,
instead they have to use some of their length to deviate passed a cluster of
hemicelluloses consequently shortening the over all distance the fibril can
cover. One side effect of having these deviations is fibrils should not have a
consistent cross sectional area over their whole length, where the
hemicelluloses have been deposited should result in an increased cross section.

Both of these hypothesis would likely (although not necessarily) result in a
positive correlation between strain and hemicellulose content within the
G-layer, however Muller et al 2006 found low hemicellulose content in the
G-layer --compared to what --

The generation of longitudinal maturation stress in wood is not dependent on
diurnal changes in diameter of trunk --- new info on water pressure hyp

%--maybe speculate on the orentation of lignin alowing for control over the way
%pressure is exerted on the pores causing different shape changes--

It is worth noting that because bark is also formed by cambial cells
differentiating, when the cambium divides to the outside, the cells (typically)
phloem --check-- become bark. The bark is therefore under transverse tension.
Some of this is alevated via the bark peeling as it ages and is forced further
from the cambium, however the remaining ring can still be providing a
significant stress on the stem as it tries to contract. Bark rings contracting
when removed from disks has been observed by kraus 1867 Krabbe 1882. Okuyama et
al 1981 measured 750 micro strains in Japanese cedar. Bark is often observed to
split, indicating the maximum strain that the bark can withstand is often
reached before it can be shed, resulting in a limited amount of stress in the
outer regions of the bark layer.

There are currently a number of outstanding issues associated with all
of the current hypotheses/theories. When and how do the stresses get
generated is still of much debate, over the last couple of decades it has become
fairly widely accepted that the generation of the stresses occurs during or
immediately after the deposition of the secondary cell wall. Most commonly either
the G-Layer or the S2 layer are considered responsible. What the mechanism(s)
is within the cell wall has been hypothesised about at great length (as
discussed above), however no theory presented so far is without country
experimental evidence.

Unfortunately most literature has investigated very few samples and reports high
variability within individuals and species,---

Another outstanding issue, common to many biological problems is why do
particular traits vary so much between individual and species? One of the
more debated topics around growth stress generation is whether the generation
mechanisms for stress in reaction wood are extreme versions of the same
mechanisms in normal wood. The G-layer is not found in normal wood, however not
all tension wood producing species produce G-layers. Lignin swelling could
potentially fit this criteria for normal and compression wood, however
modification of Boyds theory would be needed address the dependence of a MFA
as some wood with lower than 40 degree MFA still produces compressive forces,
and there has been reported to be little lignin within the G-layer, which is
suspected to be responsible or at lest partly responsible for tension
generation. Boyds theory combined with excessive mild compression wood
formation in core wood still allows for the same tensile generation mechanisms
to be used by older cambiums, as long as the MFA is suited to the task.

It is fairly well accepted (although almost by default) that growth stresses
exist because they provide a mechanical advantage for survival. However to
quantify the mechanical advantage with so much variability between individuals,
and no known way of controlling growth stress generation this is very difficult.

Growth stresses studies have been largely confined to model, or common species
however there are a number of species which appear to form intermediates or
`strange` forms of reaction wood. For example Hebe is a angiosperm which appears
to form compression wood rather than tension wood.

\subsection{Cellular modeling not focusing on growth stresses}
A number of mathematical models have been presented from the
molecular to the cellular (and whole organ) level --refs--, unfortunately as
growth stress was not a primary concern to these researches they are neglected.
Luckily these works have made significant advancements in other areas of
understanding which need to be incorporated into growth stress research. --maybe
a little on these advances, ---Ref a review paper----- Breifly discuss the math
tecneques used---Has any one used descrete or hybrid systems?---

--note about caves constitutive realationship cace 1968, anisotropic
elasticity of the plant cell wall, further notes on this formilation in Yamamoto
and almeras 2007 (mori-tanaka)---

Recently molecular dynamics methods have been used to simulate small volumes of
the cell wall in order to investigate what nano structure may be present.

\subsection{Experimentation}
Currently there are three commonly used experimental methods for measuring
surface strains. The Nicholson (1971) method, the `French' method and the strain
gauge method.

After the developments of Boyd and Jacobs in testing for growth stresses it
became apparent there was a need for a rapid testing procedure. Nicholson (1971)
developed the first of these measuring the released strain between two metal pins
on the surface of the sample, cut from the surface of logs. While considered a
rapid method in 1971, updated versions of this test are still used for measuring
surface strains but not practical for (or considered rapid) for testing larger
numbers of stems such as in breeding trials. The `French` method (current
iteration Gerard et al 1995) involves drilling a hole between two reference
points, with a dial measuring the distance change between the two points.

Okuyama et al (1981) adopted the use of strain gauges to measure stem surface
stresses of particular layers of wood. Other methods were also derived around
the same time, Guenau and chardin 1973, Guenau and Kikata 1973, and kikata and
kiwa 1977 investigated drilling holes near strain gauges to release strains.
Saurat and Gueneau 1974, 1976 introduced an apparatus which utilised two knife
blades at a set distance, one knife blade bent as the strain was released via
drilling the strain release was measured on the blade.

Measuring strains inside the stem proved to be more difficult. Kikata (1972)
adopted Jacobs planking method and electric strain gauges for improved accuracy.
Kubler (1959) and Wilhelmy and Kubler (1973) drilled holes of known diameters into stems and
attempted to measure the change in shape of the hole as the log was successively
crosscut closer to the test site, as Boyd (1950) had done. Ploge and Thiercelin
1979 attempted to measure the effect of growth stresses on increment cores,
although they found that the stresses had an effect on the corer its self
squashing it into an oval shape. Ferrand 1982 found a correlation between
longitudinal strain and tangential core diameter between -0.67 and -0.77, showing
they can be used for near non destructive growth stress testing.

--split tests--kens papers--
--the more recent external measuments of eucs and rain forest species ---
--Experimental papers on cell properties at macro/molecular scales--

\subsection{Why growth stresses exist}
Hardwoods typically have much larger growth stress magnitudes than softwoods.
--why-- is this true 'xylem cell development'?--- Some young conifers have been
reported to have larger compressive stress at the stem than at the pith,
this may be attributed to the abundance of compression wood in juvenile
conifers observed by some. Once older they follow the same radial stress profile
as hardwoods.

The commonly accepted argument for the reason of growth stresses existence is
the mechanical hypothesis. The mechanical hypothesis argues that a number of wood
properties, including the development of growth stresses evolved in order to
provide increase mechanical stability of trees in order to increase their
survival. The mechanical hypothesis as applied to growth stresses argues that
because wood is stronger in tension than compression by preloading the outer
edge of the stem in tension it increases the non-destructive bending radius on
the inside of the curve when a force is applied causing the stem to bend.

Tangential stresses have been suggested to resist mechanical failure in times of
frost (when water inside the cells freezes and expands) and drought (when water
tension is very high) Kubler 1983.

The growth stresses produced by reaction wood allow for the tree to correct its
center of gravity, orientate in such a way as to minimise external loads such as
wind and position it's self for optimum light interception. All of these
increase competitiveness.

Typically when attempting to determine the reasons for why wood properties exist
one of four hypothesis are used; mechanical, hydraulic, time dependent and a
combination of the previous three. Initial speculation as the the reason for
growth stresses existence came from Martley (1928) who briefly entertained the
mechanical hypothesis based on self weight. Jacobs (1945) suggested they were a
byproduct of sap tension, which he later retracted Jacobs (196?) when sap
pressures were recalculated at a much lower value than the generally believed
values at the time. .. Growth stresses indubitable have an effect on the
mechanical stability of trees, although it is conceivable that the effect may be
byproduct of another driver.

-- new section on negative gravitropism? ---

\subsection{Issues growth stresses cause }

At harvesting growth stresses are released by the saw cut (and crosscutting etc)
and can ruin structural and veneer logs due to the resulting splitting and
warping. Growth stresses, particularly reaction growth stresses increase the
danger for the faller by effects such as saws binding and `barber chairing`.

End splits, heart checks, and ring shakes all reduce the value of the timber in
a stem. When the stem is felled or cross cut, growth stresses are released
around the saw cuts causing shortening at the periphery and extension in the
centre. The dimension change is maximum at the saw cut, reducing as distance
from the cut increases. When the contraction/extension force exceeds the plastic
limit of the stem splitting occurs.

Prolonged compression at the centre of the stem during growth can exceed the
elastic limit of the wood, resulting in internal defects such as brittle heart.
When the stem is felled these defects have already occurred and hence there is
no way to prevent them during felling, however selection for low growth stress
producing families should significantly reduce the occurrence of internal
defects.

Within mills during processing growth stresses cause a number of issues leading
to reductions in value recovery. Because growth stresses are released when the
stem is sectioned via sawing (plain, quarter etc.) the resulting shape change
can cause the saws to jam. The main value loss at this stage of processing comes
from the need to saw boards multiple times in order to release the stresses
while still allowing for the final board dimensions to be retrieved.
Increasing the number of times the boards are sawed to get their end dimensions
gives not only poor saw use efficiency but the major economic loss comes from the
final yield being as low as 30%.


\section{Proposed theoretical and experimental work}

% \subsection{A proposed modification to the lignin swelling hypotheses}
% The lignin swelling hypotheses (Boyd 1950) argues the deposition of lignin into
% the secondary cell wall forces the cellulose fibrils away from each other, because
% cellulose is very stiff when it bows because of the lignin pushing the fibrils
% apart the cell changes shape based on the MFA. One of the main arguments for the
% use of the cellulose contraction hypothesis is that the G-layer is mainly
% crystalline cellulose and hence is not effected by lignin swelling. However if
% the outer of the cell is constrained transversely by high MFA fibrils (as in the
% P and S1 layers) and surrounding cells, when lignin is deposited in the S1, S2
% and/or S3 layers causing swelling, cell expansion will occur toward the cell
% lumen (as long as the MFA is conducive to transverse swelling). If the G-layer
% is already constructed when this swelling into the lumen takes place it will
% cause a bowing of the cellulose fibrils within the G-layer and consequently
% contraction of the cell.
%
% To create the maximum amount of contraction from the G-layer there will be an
% optimum MFA within the secondary wall (most commonly the S2 layer) dependent on cell
% geometries, around 40 degrees. Further because the G-layer is contracting the
% cell, it requires the rest of the cell to be as flexible as possible (without
% compromising the cells other properties). Flexible cells commonly exhibit higher
% MFAs than stiff cells within the S2 layer. By having a non-stiff structure the
% G-layer can cause more contraction on the individual and surrounding cells while
% under less bowing from lignin swelling. Therefore the optimum MFA of the secondary
% wall (excluding the G-layer) will provide the G-layer with the maximum ability
% to contract the cell when MFA is in the mid range, i.e. non-stiff and maximum
% transverse swelling for minimal longitudinal dimension change.
%
% The qualitative basis of the lignin swelling hypothesis in its current form
% remains unmodified and can still account for normal and compression wood growth
% stresses.

% By including the G-layer (assuming the experimental work shows
% the G-layer is a contributing factor to the production of growth stresses in
% tension wood) within the model (by implementing the hypothesis above), and
% comparing the required chemical make up and cell geometries with the
% tension wood experimentally investigated, light should be shed on the likelihood
% of the lignin swelling hypothesis being extendable to include G-layer type
% tension wood.

% The cell wall constituent study results (when retrieved from sources of a high
% enough accuracy) will be used to make comparisons between the growth stresses
% produced by the stems and the different properties within the cell walls. With
% the anatomy results collected from tension, normal and opposite wood comparisons
% can be made not only within trees but also between trees. By comparing tension
% wood with the G-layer removed with normal wood with similar properties some
% insight into the role of the G-layer should be gained.

% The other set of experiments proposed is to investigate the extent of an effect
% the G-layer has on tension generation within tension wood, and how the G-layer
% generates these tensions.
%
% Proposed experiment one:
% Taking samples with G-layers and applying a vacuum pump to suck an enzyme
% treatment into the fibres and vessels to degrade the G-layer releasing the
% strain which the G-layer is applying to the samples will cause a shape change.
% By comparing the initial and final shape change the strain the G-layer was
% imposing on the samples can be obtained. Further the rest of the growth stresses
% can be released using more traditional techniques such as split tests or
% planking, by releasing the remaining strain the proportion of stress associated
% with the G-layer and other cell wall components (assumed to be the S2 layer) can
% be determined.

% Proposed experiment two:
% Release the stress with a split or planking test, then remove the G
% layer using the same method as above. When the tension caused by the G-layer is
% released relaxation back towards the the initial state should be observed. The
% proportion of G-layer induced strain and S2 induced strain will then be
% evident.

% By splitting the half loops through the centre, separating tension and opersite
% wood the amount of extension force (i.e. what compression wood produces)
% produced by the opersite wood can be gained. This may just be due to geometry.

% Any of the three proposed experiments, if they work will provide the proportions
% of the strain in tension wood which can be attributed to the G-layer.

%potentually repete ??? exp with more samples useing moden tech. I think this
% must be the exp of grozdits 1969 but not sure. This is kind of usless anyway
% without measuring luman diameter and diameter of Sx walls. Although we should
% observe tangential swelling when the lignin is deposeted.
% if we susesivly remove cells from the cambium inward checking them for cellulose
% and lignin presents/quantaty same as in grozdits 1969 but also measure the
% difference in shape of the cell removed compared to the shape of the hole.

% Paper claiming camaldulensis has G-layer, note that it is a S1-G cell
% Chemical and anatomical characterization of the tension wood of Eucalyptus
% camaldulensis L. Mokuzai Gakkaishi

% Potential for the use of modern technologies such as portable NIR to investigate
% older trees (in the Marlborough and Nelson plots) without destructively testing
% them may be of use. NIR has shown promise in predicting longitudinal surface
% strain in Sugi green logs (Watanable 2011). Surface tests such as described by
% --ref-- will be required to either calibrate an NIR predictive model, or if NIR
% is unfeasible to acquire surface strain values on older specimens.
%
% Partially destructive testing may be appropriate on a limited number of older
% trees. Core samples will be taken for other works, however at this stage there
% are no methods for the measurement of growth stresses (directly or indirectly)
% from core samples. Surface strain tests using strain gauges as have been used by
% a number of authors --refs-- could be undertaken if necessary.

%Contact Ruth McConnachie: rgcmcconnochie@xtra.co.nz for DFI details.


\subsection{Theoretical work}
Over the years there have been a number of attempts to mathematically model
cells (usually fibres or tracheids) from cell wall constituents (Mark 1967,
Koponen et al. 1989, Harrington et al. 1998, Yamamoto and Kojima 2002, Kojima
and Yamamoto 2004 are a few example) however very few efforts have used these
techniques to investigate the formation of growth stresses (archer 1987,
Yamamoto 1998, Guitard et al. 1999).

Currently the most advanced model for how growth stresses develop within the
cell wall was presented by Almeras et. al. (2005) using the unified hypothesis
(Okuyama et. al. 1986, Okuyama et. al. 1994, Yammamoto et. al. 1991, Yammamoto
et. al. 1992 and Yammamoto 1998) utilising both the lignin swelling and
cellulose contraction hypotheses. For details see section ---.

Proposed model of the cell:
Modelling of a generic single cell with variable cell wall parameters to
investigate the required geometry and constituents to create maximum
longitudinal and tangential extension and contraction via the lignin swelling
hypothesis. The single cell model should have the capacity to put limits on the
magnitude of stress generation the lignin swelling hypothesis is theoretically
capable of under different constituent and geometric makeups.

Because the proposed experiments (see section --) induce tension wood in species
both with and without G-layers an experimental upper limit of the stress
generation the lignin swelling hypothesis is capable of should be reached and
compared to the theoretical one derived above.

It is expected that the base model and parameters will be similar to those
utilised to describe lignin swelling by Almeras  et. al. (2005) and Yammamoto
(1998). Cell wall layer radii, thickness, S2 layer MFA, moduli of the CMF
bundles and matrix will all be included. Additional variables will be included
as necessary. It is intended to add the standard deviation of the MFA within the
cell wall layers, as in Harrington (1998), pore size (or conversely fibril
aggregate size) (Fahlen 2005, Chang 2014, Salmen 2012, Kim 2012) and cell wall
constituents (Baba 2009, Donaldson et al. 2001) and layer properties/geometries
(Bergander 2002, Grozdits 1982, Almeras 2005, Yammamoto 1995, 1998, Chang 2014,
salmen 2002) to form a model, conceptually similar to the qualitative
architecture presented by Mellerowicz et al. 2012, salmen 2009 and others --try
to find original--. Boundary conditions will be initially derived from those
presented by Almeras et. al. (2005) and further modified for increased realism
and/or usability of later models.

One of the major differences between the model presented here and in previous
literature is the inclusion of fibres intertwining macrofibrils. Recently Chang
et al. (2014) measured the pore size and shape within tension wood and oppressed
wood of poplar during cell wall maturation. With this recent advancement,
reasonable assumptions around how regularly fibrils interact with other fibrils
outside of their host macrofibril can be made. It is thought that these pores
occur between joining fibrils connecting the macrofibrils into the larger
structure that is the forming cell wall. If the deposition of lignin into the
pores forcing the fibrils apart is the mechanism by which growth stresses
develop the quantity of pores and pore sizes are important parameters to
investigate as they will largely affect the ability of the mechanism to cause
stress.

The model is limited to a single unconstrained cell, which differs significantly
from a cell within a tree, due to the constraining effects of the surrounding
wood. Boundary conditions will be used to minimize this problem because modeling
a significantly large volume of wood is infeasible. The minimum resolution which
will be considered is the fibril aggregate, these are thought to be the smallest
structural components responsible for growth stresses under the lignin swelling
hypothesis.


\subsection{Experimental work}
Currently neither lignin swelling or cellulose contraction (described in section
---) have any direct experimental evidence. The tension which cellulose is under
on the stem periphery has been directly measured using x-ray diffraction showing
a strain reduction of 0.2\% in cellulose when the stress is released (Clair
2006).

Experimental evidence of the G-layer providing contraction within tension wood
has been presented by Goswami (2008). Longitudinal extension and tangential
contraction were observed when the G-layer was enzymatically removed from
tension wood poplar samples. The S2 layer was reported to have a high MFA (36
degrees) as has been reported previously and for other G-layer producing species
--refs--. Goswami (2008) suggested lateral swelling of the G-layer caused the
contraction.

The primary goal of the set of experiments which will be presented within this
chapter is to attempt to identify which cell wall constituents are controlling
stress generation and how they are controlling stress generation under different
conditions. In order to evaluate stress generation mechanisms a number of
experimental techniques have been identified.

Basic cell wall anatomy and geometry needs to be investigated for the NZDFI
species involved in this project. Where possible literature values will be used
to approximate values for model parametrization.

The following properties are required, however will only be sort from
experimental techniques when it is deemed there is a significant advantage over
available literature values.

The cell wall anatomy of different wood types (tension, normal and opposite)
needs to be investigated for the various NZDFI species (principally \textit{E.
bosistoana}). The anatomy study will consist of investigating which species
produce a G-layer (microscopy with staining) and the cell wall architecture
(Atomic Force and Electron Microscopy), cellulose, lignin and other constituents
volume fractions (Acid hydrolysis combined with NMR studies)  and the MFA and
the MFA standard deviation in all three wood types (x-ray diffraction). Fibre
diameter, length and lumen size will also be obtained (microscopy). Within
tension wood the removal of the G-layer (in G-layer producing species) will be
needed in order to determine the secondary cell wall properties of tension wood
(enzymatic removal).

Note that growth stresses for a large number of samples will be collected during
the breeding work, however because of the time consuming nature of the
experimental works presented here only a small number of specimens will be
tested as needed.

In order to produce the three types of wood required two different growth
manipulation techniques are suggested:

Technique one; Young stems (less than three month old growth from coppice) will
be restrained to a loop, similar to Jacobs' loops (Jacobs 1945) and allowed to
grow for approximately three months, with regular adjustments of the restraints
to make sure the cambium is not damaged.

Technique two; Straight one year old stems (from coppice, and seedlings of a
mixture of camadulensis, tricarpa and quadrangulata) will be bent and restrained
and allowed to growth for a further three months, with regular adjusting of the
restraints to avoid cambium damage. Normal wood samples can be collected from
these stems from wood produced away from the bend site. These plants will be
selected from camaldulensis (reported to produce S1-G tension wood (Baba 1996)),
quadrangualata and tricarpa depending on the suitability of the plants
available, and there ability to produce a G-layer will be investigated with
microscopy with staining --stain ref--.

The following experiment is proposed in order to investigate the proportion of
the stem reorientation that is due to the G-layer. During growth tension wood
production is induced by forcing curvature into the living stem, as described
above. By introducing an enzyme treatment to the plant while it is still
transpiring to degrade the G-layer and reverse any straightening that was caused
by the G-layer. With the G-layer removed the remaining stress can be released
via planking or splitting.

Due to the expense, time, equipment and expertise required to test all of the
properties needed to parameterise the model presented in section --- a number of
properties will be taken from literature and assumed to be a good approximation
for the current work. It is likely to be the case that a number of these
properties are not from Eucalyptus species.

\subsection{Breading}
Because growth stresses cause a number of issues for harvesting and milling
timber, tree breeding programs can and have be used in order to select for
genetics which reduce these effects. There is no reason to expect  breeding for
growth stresses differs significantly from (conventionally) breeding trees for
any other trait. Over the last few decades many advances have been made in
experimental and statistical techniques which rapidly improve the time and
accuracy of conventional breeding.

It is suspected that the most efficient way to minimise the issues growth
stresses cause during the production of timber is through appropriate genetic
selection. Eucalyptus species, in particular \textit{E. bosistoana} are showing
promise within the NZDFI trials to produce high value naturally durable
structural timber. In order to see the yield efficiency required to make this
product profitable, growth stresses need to be reduced to minimise the effects
discussed in section ---. While within the NZDFI project there are a number of
other concerns for breeders (such as durability, form and growth rate) growth
stresses also need to be considered. Using conventional breeding methods
discussed below, growth stresses will be minimised within the NZDFI genetics.
Currently two trials have been established or will soon be established, these
include:

All trials at Harewood are set out as randomised individual trials. Principally
this work will be concerned with \textit{E. bosistoana} of which there  are two
trials. One with 8 replicates of 20 families, planted in 2012 and coppiced in
2013 (check these), due to be harvested in spring/summer 2015. The other E.
\textit{E. bosistoana} trial has 10 replicates of 20 families, was planted in
(2010) and harvested for the first time in 2012, the plants were then coppiced
and harvested again in December 2014. Four families representing the highest and
lowest growth stress generating genetics were coppiced for a second time and
will be due for harvest in 2016. Preliminary results from the 2012 and 2014
harvests show reasonably high heritability of growth strain generated family
rankings. The same data was collected from \textit{E. argophloia} plants planted
in 2010 measured and coppiced in 2012, with final measurements completed in 2014.

Due to the success of the previous trials NZDFI is currently setting up a 4168
plant trial with 81 families of \textit{E. bosistoana} (with an increase to ~160
expected in spring 2015), 336 seedlings over 13 families of \textit{E.
argophloia}. The trials are set up as alpha lattices and harvest is expected in
late 2016 or 2017. The intention is to have as much overlap as possible with the
existing genetics within the current and previous trials.

Note the term family is used here to mean the same mother, but not necessarily
the same father. If seeds are collected at different times even from the same
tree variability exists due to possibly of different set of fathers. Also some
Eucalyptus self propagate, but it is unknown which ones or what proportion of
seeds are self propagated within the NZDFI genetic material. Any effects of self
propagation are ignored.

Within the structure of the breeding program there is the ability to
statistically check within family genetics results from young stems and mature
stems, although the trees are grown under different environmental conditions.

In order to select for low growth stress producing families experimental tests
need to be undertaken on each plant. The main test to be used to determine the
extent of growth stresses within the breeding population is the split test (also
known as the Pairing Test) as described in Chauhan (2010) The test involves
taking a significantly long section of stem and cutting along the pith to create
a radial split. The diameter of the stem is taken before testing and the width
of the opening measured immediately after splitting. Once the opening is
measured the stem is cut to across the grain to give two samples (one from each
side of the split). Density is measured by measuring the mass (using balances)
and volume using the displacement method on each of the pieces. Acoustics are
also taken using wood-spec to calculate the dynamic modulus, and hence the
stress can be derived Chauhan (2010). Due to the size of the Woodville trial it
may be the case that less tests are carried out. Decisions on the essential
tests will be made near the time of harvesting. Throughout testing the samples
are kept in a green state.

Although the particular statistical techniques best suited to the data set
obtained from the trial results can’t be known before the data is collected, it
is expected that typical breeding statistical techniques will be applicable. The
two objectives to be achieved for the breeding trials are to estimate genetic
parameters, (in particular heritability of growth stresses) and to predict
breeding values for families. Mixed model methods are commonly used within tree
and animal breeding for these purposes.

The breeding work undertaken in the thesis will be limited to the investigation
of growth stresses in the NZDFI Woodville and Harewood \textit{E. bosistoana}
trials. No attempts will be made to predict growth stress magnitudes from
individuals or families, only a ranking of lowest to highest growth stress
producing families will be considered.

The assumption is made that families which rack well at young ages will rank
well at a commercially viable harvest size.

\section{Objectives}
The proposed research has two main objectives; 1) add information to the
debate on the molecular mechanisms forming growth stresses within trees and 2)
increase the quality of the breeding stock within the NZDFI \textit{E.
bosistoana} by reducing growth stresses.

\section{Importance and contribution of research}
Mathematical modeling of how the xylem cells mature after primary
wall expansion will provide insight into the molecular mechanisms
responsible for growth stress production. Information regarding the mechanisms
for growth stress production is useful for fundamental understanding of
wood formation, as well as its potential use in biomimetic materials and
genetic modification of trees.

Rapid testing and genetic selection for growth stresses will continue to develop
tools needed for increasing the quality of the breeding stock of NZDFI
\textit{E. bosistoana} and help improve future breeding trials concerned with
growth stresses. Delivering improved \textit{E. bosistoana} genetics with
reduced growth stresses for propagation.


\section{Costs}
Charges associated with the use of the AFM (\$2000 per year for one or two
years), and other imaging technologies.

Travel expenses for use of equipment which cannot be sourced at UC, potently for
use of NMR, SEM (SFF proposed).

Travel expenses to Woodville as required (SFF proposed).
Travel expenses to 8th Plant Biomechanics Conference - Japan, November
30-December 4th 2015

Bluefern - Generally free under peer reviewed research grants. Potently the use
of other HPC facilities if they are better suited to the task (SFF proposed).

Sundry and operational items such as lab materials, chemicals and expenses for
loop and bending trials at Harewood.


\section{Timeline}
\onehalfspacing
\begin{tabular}{ l l }
\hline
  March 2015 & Set up loop and bending trials \\
   & Proposal due 1 April 2015 \\
   \hline
  April 2015 & Harewood analysis \\
  & Write up preliminary results \\
  \hline
  May-July 2015 & Harvesting loops and bending trials \\
  & AFM, x-ray diffraction and chemical composition experiments\\
  & Enzyme experiment\\
  \hline
  August 2015 & One year report due 1st September 2015\\
  \hline
  September-October 2015&Write up anatomy and enzyme studies\\
  &Cellular modeling\\
  & Harewood harvest\\
  \hline
  November-December 2015 & Japan \\
  &Re-establish loop trials\\
  \hline
  January 2016&Harewood analysis\\
  & Update preliminary results \\
  \hline
  February-April 2016 & Modelling\\
  \hline
  May-July 2016&Harvesting loops\\
  &AFM, x-ray diffraction and chemical composition experiments\\
  &Update experimental paper\\
  \hline
  August-November 2016&Finish Modeling\\
  &Write up modelling\\
  \hline
  December 2016&Woodville harvest\\
  \hline
  January-February 2017&Woodville analysis\\
  &Harewood harvest\\
  &Harewood analysis\\
  &Write up breeding paper\\
  \hline
  March-October 2017&Complete remaining tasks and papers\\
  &Submit thesis\\
  \hline
\end{tabular}
\doublespacing

\section{Health and Safety risk assessment}
Engineering health and safety introduction (completed)

Wood tech lab health and safety introduction (completed)

Wood tech lab health and safety training for particular machines (all machines
that will need to be used have been completed, other machines can be added as
needed)

AFM and micro-nano Fabrication laboratory training (currently in progress)

Chemistry work inductions and training will be undertaken as required.

Woodville health and safety plan (to be prepared)




\section{Proposed Chapters}
Geometric parameters of the cell wall molecular architecture; MFA (from x-ray
diffraction), fibril aggregate arrangement (from AFM) in the  cell wall and the
chemical composition of wood.

Mathematical model investigating the formation of growth stresses on a
supramolecular scale.

The effect of the G-layer on growth stresses released by enzymatic removal.

Analysis of the heritability of growth stresses in Eucalyptus bosistoana through
successive copising.

Analysis of the heritability of growth stresses in Eucalyptus bosistoana at the
family level (Woodville trial).



\section{Preliminary results}
2012/2014 Harewood analysis

anatomy study

enzyme experiment

\end{document}