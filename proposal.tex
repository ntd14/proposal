\documentclass{article}

\usepackage{graphicx}
\usepackage[space]{grffile}
\usepackage{latexsym}
\usepackage{amsfonts,amsmath,amssymb}
\usepackage{url}
\usepackage[utf8]{inputenc}
\usepackage{fancyref}
\usepackage{hyperref}
\hypersetup{colorlinks=false,pdfborder={0 0 0},}
\usepackage{textcomp}
\usepackage{longtable}
%\usepackage{multirow,booktabs}



\begin{document}

\title{proposal}

\author{Nicholas Davies\\ University of Canterbury }

\date{\today}

\bibliographystyle{plain}

\maketitle

\section{Introduction to wood structure and formation}
As trees grow they produce wood in order to become taller and wider. Becoming
taller and increasing canopy size is an effective way to out compete the other
trees and plants for light. With increasing height and width comes increasing
weight, wind drag and internal pressures (for water transport), which requires
either enough redundant strength in the existing structure (such as young monocotyledons) or
for the tree to strengthen its structure as it increases its size. In
dicotyledons and gymnosperms this occurs in two ways, apical and cambial growth
on branches, roots and the stem(s).

Softwoods have a simpler micro structure than hardwoods, consisting mainly of
axially elongated pointed cells named tracheids which serve as both mechanical
support structures and water conduits. Although varying with species, softwoods
may also contain radially orientated tracheids, radially or axially orientated
parenchyma cells and other cell types. Tracheids are the dominant form of cells
within the stems and branches.

Hardwoods contain a more complex micro structure with a number
of different cell types. Fibres provide structural support as their primary
function, while similar to softwood tracheids they differ in some key aspects,
being shorter in the longitudinal direction, more rounded in the transverse
outline, tend to have smaller lumens and have little role in sap ascent. However
the ends do taper to points as in softwood tracheids. Libriform fibres tend to
be longer than fibre trachiads, have thicker walls and are solely for support.
Fibre trachiads function in both conduction and support, as in softwoods,
however their appearance in wood with vessels suggests that they function
primarily for support, and perhaps are an intermediate evolutionary feature
between the softwood trachiad and the libriform fibre. Septate fibres devide
their cell lumens into chambers without crossing the primary cell wall. Septate
fibres are produced in the late stages of division just prior to the death of
the cytoplasm, and appear to resemble axial parenchyma cells, and have been
hypothesised to store starches, oils and resins.

Vessels are the main conduits for sap ascent. Vessels are comprised of multiple
vessel elements being joined at the ends to form long conduits, which can extend short
distances (often less than 200mm) or can be as long as the height of the tree.
These elements are connected through pores or perforations in perforation
plates at the end walls of the cells. The arrangement of vessels into groups is
species dependent and usually described as ring porous (the vessels congregate in early wood)
or diffuse porous (vessels are distributed throughout both early and late wood).

Rays are formed from radially orientated cells often tracheids or parenchyma.
Hardwoods typically contain multisteriate parenchyma rays, but there are a
number of species with unisteriate or a combination of ray sizes, comparatively softwoods rearly
contain multisteriate rays. Parenchyma ray cells are living within sap wood,
however during the transition to heartwood die and are used for storage of
extractives. Rays also provide a mechanical advantage by diverting the
axial force flow reducing buckling and shear stresses between fibres.

Further cell types also exist, such as vasicentric tracheids which have profuse
side wall pitting exhibiting deformation from the expansion of the surrounding
vessels. Axial parenchyma cells are generally abundant and tend to exist in
vertical files and are expected to play a role in the development of heartwood.
More detailed wood anatomy and has little bearing on this project and is
discussed in a number of wood anatomy texts.

In order to reorentat stems and branches of (most) trees produce reaction wood
which provides a force in order to reorentat the tissue. Typicly this
reorentation is toward the light or upwards as is defiend by the negative
gravitrposim hypothesises. Other reasons for reorentation such as reduceing wind
drag have also be sugested. In softwoods this reoretion is caused by the
production of compression wood. Compression wood forms on the out side of the
stem or branch and (expands? so that it is under compression? causeing a
restoring force). Hardwoods on the other hand produce tension wood on the inside
of the desired curve which (contracts?) resulting in a curve forming.
Tradtionly the galaterness layer (G-layer), a layer primerally consisting of low
angle cellulose fibrils on the inside of the fibre tracheids, is credited with forming growth
stresses within the tension wood. However some hardwoods produce tension wood
without producing a G-layer such as E Nitens.

Primarily, at different resolutions this work focuses on the fibre tracheids as
they are the structual cells expected to be responcable for growth stresses in
normal and reaction wood within hardwoods. The fibre tracheids consist of a
number of cell wall layers depending on the species, the particualar cell and
its primary function. Normal wood fibres within Eucalyptus species (CHECK THIS)
consist of a middle lamaner (conecting the fibre to the sourding cells) a
primary cell wall and a secondary cell wall consisting of S1, S2 and S3 layers
(produced in coronalogical order so the exact composition will change depending
on the cells developmental stage). The S2 layer is the largest layer and
consists of cellullose macrofibrils wraped helicly around the cells longitudinal
axis. This cellulose is contained within a matrix of hemicelluloses (examples)
and lignin giving the cell wall properties of a fibre reninforced matrix. --how
does this provide structure--

In order for the living cambrial cells to produce wood, each cell must go
through devision from its perant cell, growth and death. Because the
cambriam (and apical merastem) are continually deviding it alows for the tree to be a dynamic
structure changing its form to become better adapted to its current enviromental
setting even though large portions (ie the wood) are dead. The transition form
devision through elongation and development to death is expected to play a role
in the development of growth stresses within the stem.

\subsection{Basic cell division}
Dicotyledons and gymnosperms grow in two main ways, upward apical growth and
outward cambial growth. Note monocotyledons (for example palms) do not produce
secondary growth and instead diameter forms as part of primary growth.

As the cambium is forming, fusiform and ray initials are created.
(how are the initials created)
Fusiform initials are short radially and tangetialy with tapered
ends. From the cambial initials, cells to the inside create the vertical
elements of xylem (tracheids, vessels, fibers, parenchyma, etc.), while cells outside become phloem.
Ray initials produce horizontal elements (rays).

Cambial cells divide in two ways, periclainal and anticlinal.
Periclainal cell division occurs to the inner and outer of the cambial layers.
As the cell division to the inside occurs the volume of secondary xylem that is being
formed increases the tangential stress on vascular cambium resulting in an
extention of the cambial circumference. Although over time many plants show an
increase in the longitudinal and tangential dimensions of the cambial initials it is
likely that this expansion is mainly facilitated by anticlinal division followed
by the expansion of the daughter cells next to the pedant.



\subsection{Cell formation, elongation and death}
During primary wall formation rapid elongation occurs. When the cells devdie
from their perants they remain fixed to their nabiours via the midel lamina. The
intenal hydrostatic (turgor) pressure causes cell expantion. The osmotic flow of
water from the outside the cell to the inside (due to a lower solute concentration
outside the cell than in) which is constrained by the primary cell wall, the
primary cell wall becomes under increasing tension as more water flows into
the cell. Because the centre of the cell has restricted movment, in order for
elongation (to disipate the increaseing tensile forces generated from the
inflow of water) to occour the cell turns the biosythesis of cell wall
constituants to produce tip growth. Growth at the tips of the cells allows for
the cells to remain a cosntnat thickness, so no streching is needed during the
elongation phase, as has been sugested previously. The expantion of the cells is
suspected to be controld via modulation of the primary cell wall rather than via
turgor pressure. -- note that primary wall has randomly orentated MFs embeded
in hemicellulose and pectic compounds and becomes lignified after S layer added
, ML is non lignifed, note often compound middle laminer is used to describe
the ML and P at once as are hard to distinguish--- Once the cell has reached its
full size biosynthises of the S1 starts.
Typicly the S1 layer is thin and comprises of very high angle microfibrials,
within the layer many laminates are found. Within each laminate the MFs are
closly aligned, however between each laminate they can (but do not nessassery)
differ greatly, or even reverse the direction of the helix the MFs form around
the cell, although lower right to upper left orentation tends to be favered.
Close to the S2 layer the MFA decreases repidly.
The S2 layer bound to the inside of S1 is typicly much thicker and has more verticly orentated
micorfibrils compeared to the primary, S1 and S3 layers, these MFs circle the
cell axis from lower left to upper right. S2 contains the majorty of the lignin
within the cell. In some cases, most commonly in late wood a thin S3 layer is
also produced with high MFA, reversing the direction of the MF helixs to lower
right to upper left.

Finally if tension wood is being produced a Gelatonus layer
may be produed on the inside of the inner most wall (S2 or S3). The G-layer has
near verticly orentated microfibrils and very little lignification. It is suspected that the
G-layer plays an important role in the generation of reorentation stresses.

At some point during the formation of the seconday cell wall, or soon after the
cell shrings verticly and expands tangentially. Because of the connectivness
between cells this results in growth stresses forming within the stem, this
phenomonan is descussed in greater detail in ----. After the seconday wall
formation cell `death` occours as part of the transition from sap wood into
heartwood. While the hollow, dead cells play an importnat role in water
transport and mechanical support of the tree, over time any residual nutrant
that can be used by living cells--- heatwood stuff----

What is the deal with Rays----

\subsection{Cells and wood in the context of a whole tree}
Wood as a materal within the tree has three major functions to achive; water
transport, nutrant transport and mechanical struture. Softwoods achive
water transport and mechancial struture within trachieds, while parenchima cells
are used for nutrant transport. Hardwoods have evolved a more complicated
internal structure of vessels and fibre tracheids in order to seperates out the
functions of water transport and mechanical support respectivly.

--advantages and disavantages of this--

The growth stresses that form as part of cell formation are throught to provide
a superiour mechanical structure. Because of the continual formation of new
cells providing growth stresses on the perifery of the stem the older wood which
has completed its formation and cell death must be contracted further with each
new layer of cells atempting to contract. The result of this is the older wood
near the centre of the stem becomes compressed while the newer cells can not
contract to the extent that would leave them in their lowest energy state
remain in tension, until the bond between the old wood and new is seperated
releasing the forces restricting this contraction (and extention in the centre)

Reaction wood as described above provides the ability for the stem to reorentate
in order to be best adapted to its enviroment at any given time.

These properties of wood allow for an adaptive organism to survive..

\section{Introduction to growth stresses}

It is suspected that growth stresses develop within trachieds during the
formation the secondary cell wall, although the exact timeing and mechanisum for
developing growth stresses is still of much debate. The most current thoery is
a hybrid of the older cellulose contraction and lignin swelling hypothesis.

Wood workers have unintenionally known of growth stresses within trees for
centries. Usually refered to as `a pull towards the sap` when cuting boards good
craftsmen would section the log in such a way as to get a stright board once it
is removed from the log (and the growth stresses relesed). Most work early on in
the study of growth stresses sorounded investigating how/why boards changed
shape when cut from an intact stem.

Martley (1928) was posibly the first to study growth stresses in a scientific
manner. Initally he argued that the curvature of planks sawn from logs was due
to the current growth not being able to support the dead weight of the tree until
lignification was complete. As a result the centre is under compression while
the perfifery had zero stress. However calculations showed that the self weight
was not sufficent to cause the observed longitudinal dimention changes of the
timber.

After Martley's work a small number of authors investigated growth stresses
through the 30's and 40's. Jacobs, although testing 34 hardwood species, focused
mainly on Eucalyptus and in 1938 argued that (longtudinal) tension successively develops in the outer layers of
the stem as it grows, and as a concequnce of of the tension, compression
must form in the centre of the stem. Jacobs later used E. gigantea to descibre a strain
gradient developing during growth. Experementally Jacobs made use of strip
placking, measuring the deflection of the board after removal from the log, and
the length change when the planks were foced back straight. He showed that wood tends to shrink in the
longitudinal direction at the perifferly while extend near the pith
(indicating in the log the planks are under compression in the centre and
tension at the extremities).

Further Jacobs put foward a number of hypothesis to explane how the growth
stresses were forming. First arguing that it is very unlikly that dead cells
(wood) could extend within the core in order to create the observed stress
gradiant. Instead sugesting the causes of; weight of the tree, surface tension
and sap stream forces, cellulose and collodal complexes, lignin intercellular
substances and the primary or secondary cell wall. Although without any evidence
did not claim any of these to be the major cause.

Stresses relating to reaction wood recived more attention through the 30s and
40s for both soft and hardwoods. Jacobs 1945 stated that the reorentation of
stems is caused by a modification to the already existing stress gradiant
throughout the stem. One option he presented was simply that the eccentric
growth causes larger number of cell sheves to be added to the upper side of the
curve each providing the same amount of contraction force, this results in a
angle correction even with identical cells. Sap tension is also considered, but
more importantly Jacobs notes the posability of tensions being formed within the
cell walls of tension wood.
Munch 1938 specualted that the addition of matter into the cell wall could cause
compression wood. ..
Jacobs 1945 also found that it was commanly the case that the amount of
compression wood developed and the stem angle recovery had a poor relationship.
He sugested maybe it was infact the normal strain pattern in tension which
correct the lean, the compression wood mearly acted as a pivot, not contributing
a tensile force on the lower side of the stem.

Growth stresses in woody stems, Munch
1937 clarke 1939, luxford 1937.

Boyed 1950 new exp techneque- found that the crossover point is is about 1/3 rad
of the log from the perifphery-- first attempt to mathematicly derive a
relationship
varation by kubler 1959

Mayer-wegelin 1955-- review of checking and spliting also lenz and strassler
1959 giordano 1969

some comments on use of hookes law etc crandall 1978, lenz and strassler 1959

koehler 1933 focus on transverse stress, jacobs and boyed as well

see archer for review of models produced up until this time
Through the late 70's and 80's archer produced a number of studies
investigiating growth stress occourance and formation. ...
Cave paper somewhere around here?

The first major advancment in the understanding of the formation of growth
stresses came from wardrop (1965) where the cellulose contraction hypothesis was
first thearised. ---- Bamber (1978) refined this hypothesis further.---

wardrop 1965, bamber 1978 A tensile stress generated in teh cellulose MF forces
the TW and NW fiber with low mfa to shrink.

Boyd (1972) presented the alternative hypothesis of lignin swelling
compressive stress originating in the matrix pushes the CW
tracheid whith a high MFA to expand

Around the same time two other lesser known hypothesis were presented,  strains
due to changs in water content Hejnowicz 1967, argued that the stresses in
compression wood are related to the inhibition of water by the cell walls,
which results in swelling, because the expansion of compression wood is equal to
the shrinkage due to drying. --paper disproving this--

brodzki 1972 hypothesised strains due to 1,3-linked glucan (laricinan)
deposition could be the most significant factor in longitudinal growth stress
generation. .....
boyd 1978 later refuted this idea showing that , Brodzki's theory may have
morfed, or inspired the hemicellulose based hypothoses.

More recently theories regarding the nature of hemicelluloses and their bonding
have been used in an atempt to remove some of the ussues assocated with lignin
swelling and cellulose contraction theories. .. ..

The most resent attempts made to describe the formation of growth stresses have
been made by yamamoto and his team. They argue a combination of both lignin
swelling and cellulose contraction is needed, called the unified hypothesis.
By unifing the hypothesises they follow the current thinking of a number of
authors in other areas of wood science that both tension and compression wood
are not distictly different types of wood and are instead extreme versions of
normal wood.

issues with current understanding:
When and how do the stresses get generated is still of much debate, over the
last couple of decades it has become fairly widly accsepted that the generation
of the stresses occours during or imediatly after the deposition of the seconday
cell wall. Most commanly either the G-Layer or the S2 layer are considered
responcable.
why do they varry so much between indervidualls and species?
One of the more debated topics around growth stress generation is whether the
generation mechanisums for stress in reactionwood are extreme versions of the
same mechanisums in normal wood. The G-layer is not found in normal wood,
however on rear ocastions .. .. lignin swelling could potentually fit this
criteria for normal and compression wood, hopwever modification of boyds theory
would be needed due to the depence of a MFA as some wood with lower than about
40 degree MFA still produces compressive forces. Boyds theory combinded
with excessive mild compression wood formation in corewood still alows
for the same tensile generation mechanisums to be used by older cambriams, as
long as the MFA is suited to the task.
how can we quantify the mechanical advantage with so much varablity
what about the other unusuall forms of reaction wood, hebe etc


\subsection{Why growth stresses exist}
Hardwoods typicly have much larger growth stress magnitudes than softwoods.
--why-- is this true 'xylem cell development'?--- some have claimed conifers
have compression throughout the stem when young, not until old that they follow the same trend as
hardwoods

Prehapse the leading argument for the reasion of growth stresses existance is
the mechanical hypothesis. The mechanical hypthesis argues that a number of wood properties,
inclusing the development of growth stresses evolved in order to provide
increase mechancial stability to trees in order to increase their servival. The
mechanical hypotheisis as applied to growth stresses argues that because wood is
stronger in tension than compression by preloading the outer edge of the stem
in tension it increaseing the non-destructive bending radius on the inside of
the curve when a force is applied causing the stem to bend. --dosnt explain why
hardwoods have larger GS than softwoods, or why young softwoods exibit
compression.-- This hypothesis struggles to explane the differences between hard
and softwoods, particually at young ages. If mechancial stability is infact the
driver for growth stress generation at young ages, why do young angiosperms and
genosperms produce esentually opersite solutions?.

If the reason conifers have compressive forces when young is excessive
compression wood to enable reorentation, maybe this idicates that normal wood is
more colosly realated to tension wood than compression wood. What forms of mild
tension wood are known of?

speculation from various authors
Typicly when atempting to determain the reasons for why wood properties exist
one of four hypothesis are used; mechanical, hydrolic, time dependent and a
combonation of the previous three. Inital speculation as the the reason for
growth stresses existance came from Martley (1928) who breifly entertained the
mechancial hypothesis based on self weight. Jacobs (1945) sugested they were a
biproduct of sap tension, which he later retracted Jacobs (196?) when sap
pressures were recalculated at a much lower value than the genrally beleived
values at the time. .. Growth stresses undobutable have an effect on the
mechancial stability of trees, although it is consevable that the effect may be
biproduct of another driver.


\subsection{Intro to the issues growth stresses cause }

for harvesting
dangerous for feller
splitting of log at felling
internal checking
Mayer-wegelin 1955-- review of checking and spliting also lenz and strassler
1959 giordano 1969

within mills
bending/bowing/warping during cutting jaming up saws, large proportion of waste,
multiple cuts required to get boards to desired dims

lost revenue final yield less than 30%


\section{Theoretical and experimental understanding of growth stresses}

\subsection{Background of breading}
Because growth stresses cause a number of issues for harvesting and milling
timber tree breading programs can be used in order to select for genetics which
reduce these effects. --previous breading for GS-- There is no reasion to
expect  breading for growth stresses differes significanly from (convetanally)
breading trees for any other trait, which is process which has been developed
over centries. Over the last few decades many advances have been made in
experemental and statistical techneques which rapidly improve the time and
acuracy of conventinal breading.

 field techniques
Typically breading trials, like any sciantific exprements are designed in order
to minamise noise from uncontroled varables,

laboratory techniques


stat techniques

mention tradeoff with durability etc

\subsubsection{Beading work in this thesis}
What we actually have:

Harewood trial:
dec 2014 has bosistoana and argophloia copiced from old planting that mon has GS
data from. --from data can show (kind of as could argue same enviromental
effects caused it) genetic relationship.
New Harewood trial, 2016 harvest, will
have a number of species potential to copice bos again if needed.

Woodvile, 2016/2017 harvest will have Bosistoana, argophloia and possibly globoidea.
May or may not be the same families as the various drylands trials.

NOTE family means same mother, not same father.
If collected at different times even from the same tree variability exists due
to possibly of different set of fathers. Also some self propagate, but we don't
know which ones or what proportion, so ignore this.

Progeny trials are alpha latauses, harewood is a standard randomised individual
trial.

Contact Ruth McConnachie: rgcmcconnochie@xtra.co.nz for DFI details.

slit tests
Pairing Test and Longitudinal Growth Strain: Establishing the Association 2008
is the earlyest paper I can find on the split/paring test. note kens papers from
mon

surface tests

Potentually use NIR
http://www.afs-journal.org/articles/forest/pdf/2002/05/05.pdf
Has some useful info on wave lengths associated with bonds ic cellulose

Non-destructive evaluation of surface longitudinal growth strain
on Sugi (Cryptomeria japonica) green logs using near-infraredspectroscopy

statistics

Progeny trials are alpha latauses, harewood is a standard randomised indervidual
trial.

PLSR etc for NIR work

normal breading stats?

\subsection{Background of chemistry work}

lignin swelling

cellulose contraction

what has been done in the past?
that xray syncotron experement etc

\subsubsection{Chemistry work in this thesis}

Do all of the DFI species have a G-layer?
Maybe include some Nitens tests if they don't.
check MFA and SD for S\_2 in tension, normal and compression/opposite wood
Get cellulose lignin and hemicellulose(s) contests for tension normal and
compression/opposite wood Split hemicelluloses where possible, eg xyloglucan
etc.
Torsion tests on individual cells, again for tension, normal and compression. Maybe remove G-Layer
in tension wood and compare to normal and compression wood of similar MFA and compounds etc.

Could we somehow measure growth stress release on a single cell?
Ideally, grow disordered cells invitro, and separate them from the parent cell
as soon as possible, then record when in their formation they undergo what
dimension changes. Is there some non-destructive test to check what is going on in the cell? or if we have multiple cells in the same conditions maybe we can destructivly test some during the growth phase, under the assumption they are all growing at the same time.
OR
remove the cambial layer leaving top and bottom of cell attached to the stem on a large sample,
then somehow remove the connection to the cells behind it, then release the top and measure the contraction.


\subsection{Background of modeling}

yammamotos most resent attempt

possible different methods > FEM, DEM, molecular dynamics, gemomentry of stem
and cells

\subsubsection{Modeling in this thesis}
cells as partials in relaxed state

apply body force, ie the growth stress field

get original/non cut stick back

take take groups of repressive cells and use composite theory and position dependent body force
(growth strain field) from the sub domain above

introduce time dependence to see how the stress field develops during
maturation, composite scale still -- each cell can have its own clock so that it
has a maturation rate to change its field variables.

take individual cells at macromolecular level and try to produce stress field
above during a time dependent maturation function

Molecular dynamics simulations to work out the molecular mechanisms developing the growth stresses

Using the MD sims paramertorize a cell model

Using the cell model develop a time dependent field function

from the field function create representative cell blocks

put the cell blocks together into a stick

cut the stick > do we get out what we put it?

\section{Intentions}
to improve breeding stock for NZ dryland forestry with respect to eucs being used for structural timber

to increase understanding of growth stress formation particularly in eucalyptus
by chemical analysis and computer modeling

\section{Objectives}

to create a mathematical model and computer simulation of a piece of cambium forming growth

stresses at the macromolecular level

to investigate the chemical causes of GSs by chemical analysis >> how?

to improve breeding stock for eucs wrt growth stresses from field and lab testing to select appropriate families.

\section{Costs}

\section{Timeline}

\end{document}