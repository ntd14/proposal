\documentclass{article}

\usepackage{graphicx}
\usepackage[space]{grffile}
\usepackage{latexsym}
\usepackage{amsfonts,amsmath,amssymb}
\usepackage{url}
\usepackage[utf8]{inputenc}
\usepackage{fancyref}
\usepackage{hyperref}
\hypersetup{colorlinks=false,pdfborder={0 0 0},}
\usepackage{textcomp}
\usepackage{longtable}
%\usepackage{multirow,booktabs}



\begin{document}

\title{proposal}

\author{Nicholas Davies\\ University of Canterbury }

\date{\today}

\bibliographystyle{plain}

\maketitle

\section{Intro to wood structure and formation}
As trees grow they produce wood in order to become taller and wider. Becoming
taller and increasing canopy size is an effective way to out compete the other
trees and plants for light. With increasing height and width comes increasing
weight, wind drag and internal pressures (for water transport), which requires
either enough redundant strength in the existing structure (such as young monocotyledons) or
for the tree to strengthen its structure as it increases its size. In
dicotyledons and gymnosperms this occurs in two ways, apical and cambial growth.

basic architecture

softwood and hardwood structure in stems

compression and tension wood in stems and branches

detailed hardwood anatomy, euc focus

detailed hardwood fibre anatomy

segway into cell formation for below

\textbf{basic cell division}
Dicotyledons and gymnosperms grow in two main ways, upward apical growth and
outward cambial growth.

Note monocotyledons (for example palms) do not produce secondary growth and instead diameter
forms as part of primary growth.

As the cambium is forming fusiform and ray initials are created.
(how are the initials created)
Fusiform initials are short radially and tangetialy with tapered
ends. From the cambial initials cells to the inside create the vertical elements
of xylem (tracheids, vessels, fibers, parenchyma, etc.), while cells outside become phloem.
Ray initials produce horizontal elements (rays).

Cambial cells divide in two ways, periclainal and anticlinal.
Periclainal cell division occurs to the inner and outer of the cambial layers.
As the cell division to the inside occurs the volume of secondary xylem that is being
formed increases the tangential stress on vascular cambium resulting in an
extention of the cambial circumference. Although over time many plants show an
increase in the longitudinal and tangential dimensions of the cambial initials it is
likely that this expansion is mainly facilitated by anticlinal division followed
by the expansion of the daughter cells next to the pedant.

\textbf{cell formation}

\textbf{cell elongation/shape change}

note GS

\textbf{cell death > final cell shape change and chemical constituants}

Note GS

\textbf{cells/wood in context of wood and whole tree}

Note GS

relate back to first couple paragraphs


\section{intro to what growth stresses are}

ref to above for cell elongation and death

early work in 20s and related models/theories

lignin swelling

cellulose contraction

hemicelluose theories

yamamotos recent model

issues with current understanding


\section{why growth stresses exist}

hardwoods v softwoods

speculation from various authors

mechanical hypotheses

\section{intro to the issues growth stresses cause }

for harvesting

within mills

\section{background of breading}

field techniques

laboratory techniques

stat techniques

mention tradeoff with durability etc

\subsubsection{beading work in this thesis}
What we actually have:

Harewood trial:
dec 2014 has bosistoana copiced from old planting that mon has GS data from.
New Harewood trial, 2016 harvest, will have a number of species potential to
copice bos again if needed.

Woodvile, 2016/2017 harvest will have Bosistoana, argophloia and possibly globoidea.
May or may not be the same families as the various drylands trials.

NOTE family means same mother, not same father.
If collected at different times even from the same tree variability exists due
to possibly of different set of fathers. Also some self propagate, but we don't
know which ones or what proportion, so ignore this.

Progeny trials are alpha latauses, harewood is a standard randomised individual
trial.

Contact Ruth McConnachie: rgcmcconnochie@xtra.co.nz for DFI details.

slit tests
Pairing Test and Longitudinal Growth Strain: Establishing the Association 2008
is the earlyest paper I can find on the split/paring test.

surface tests

Potentually use NIR
http://www.afs-journal.org/articles/forest/pdf/2002/05/05.pdf
Has some useful info on wave lengths associated with bonds ic cellulose

Non-destructive evaluation of surface longitudinal growth strain
on Sugi (Cryptomeria japonica) green logs using near-infraredspectroscopy

statistics

Progeny trials are alpha latauses, harewood is a standard randomised indervidual
trial.

PLSR etc for NIR work

normal breading stats?

\section{background of chem work}

lignin swelling

cellulose contraction

what has been done in the past?


\subsubsection{chem in this thesis}

Do all of the DFI species have a G-layer?
Maybe include some Nitens tests if they don't.
check MFA and SD for S\_2 in tension, normal and compression/opposite wood
Get cellulose lignin and hemicellulose(s) contests for tension normal and
compression/opposite wood Split hemicelluloses where possible, eg xyloglucan
etc.
Torsion tests on individual cells, again for tension, normal and compression. Maybe remove G-Layer
in tension wood and compare to normal and compression wood of similar MFA and compounds etc.

Could we somehow measure growth stress release on a single cell?
Ideally, grow disordered cells invitro, and separate them from the parent cell
as soon as possible, then record when in their formation they undergo what
dimension changes. Is there some non-destructive test to check what is going on in the cell? or if we have multiple cells in the same conditions maybe we can destructivly test some during the growth phase, under the assumption they are all growing at the same time.
OR
remove the cambial layer leaving top and bottom of cell attached to the stem on a large sample,
then somehow remove the connection to the cells behind it, then release the top and measure the contraction.


\section{background of modeling}

yammamotos most resent attempt

possible different methods > FEM, DEM, molecular dynamics, gemomentry of stem
and cells

\subsubsection{modeling in this thesis}
cells as partials in relaxed state

apply body force, ie the growth stress field

get original/non cut stick back

take take groups of repressive cells and use composite theory and position dependent body force
(growth strain field) from the sub domain above

introduce time dependence to see how the stress field develops during
maturation, composite scale still

take individual cells at macromolecular level and try to produce stress field
above during a time dependent maturation function

Molecular dynamics simulations to work out the molecular mechanisms developing the growth stresses

Using the MD sims paramertorize a cell model

Using the cell model develop a time dependent field function

from the field function create representative cell blocks

put the cell blocks together into a stick

cut the stick > do we get out what we put it?

\section{intentions}
to improve breeding stock for NZ dryland forestry with respect to eucs being used for structural timber

to increase understanding of growth stress formation particularly in eucalyptus
by chemical analysis and computer modeling

\section{objectives}

to create a mathematical model and computer simulation of a piece of cambium forming growth

stresses at the macromolecular level

to investigate the chemical causes of GSs by chemical analysis >> how?

to improve breeding stock for eucs wrt growth stresses from field and lab testing to select appropriate families.

\section{Costs}

\section{Timeline}

\end{document}